%
% File: abstract.tex
% Author: Oliver Feighan
% Description: Contains the text for thesis abstract
%
% UoB guidelines:
%
% Each copy must include an abstract or summary of the dissertation in not
% more than 300 words, on one side of A4, which should be single-spaced in a
% font size in the range 10 to 12. If the dissertation is in a language other
% than English, an abstract in that language and an abstract in English must
% be included.

\chapter*{Abstract}
\begin{SingleSpace}
\initial{T}he challenge with modelling the excited states of light harvesting complexes
lies in scaling accurate approaches for large system sizes. Many common approaches 
to excited states are not viable on this scale, creating the need for more efficient
methods. These methods come with compromises either in accuracy, extendability or simplicity.
In spite of this these models have found great success in being able to predict 
and explain the nuanced mechanisms of light harvesting complexes. 

This work explores alternatives to these existing methods in order to overcome some
of the compromises. The design of these new methods required an understanding the
key components of a useful methods, which was achieved by comparison to high-level
data and other existing methods. The outcome of this is a set of novel response 
method approximations and workflow for reparameterisating Grimme's GFN-xTB methods.
These were applied to create a specific method, referred to as chl-xTB, parameterized
for the $Qy$ transition in LH2 BChla molecules.

The accuracy of chl-xTB transition properties for monomer, dimer and oligomer chlorophyll
systems is established with benchmarking and comparison to existing literature,
using a Frenkel-Davydov exciton framework for systems beyond chlorophyll monomers.
chl-xTB is found to perform very well in these cases, giving confidence that it 
can be reliably used to predict light harvesting complexes properties.

This is showcased by calculating novel properties of the LH2 complex, such as spectral
densities and rate of transition between charge-separated and vertical-transition
excited states. These properties further explore the role of the LH2 protein scaffold
and its manipulation of the exciton system. Both this and discussion of further
applications of both the novel efficient response framework and the specific chl-xTB
method demonstrate a possible solution to the issues of efficiently modelling light 
harvesting complexes.

\end{SingleSpace}
\clearpage