%
% File: abstract.tex
% Author: Oliver Feighan
% Description: Contains the text for thesis abstract
%
% UoB guidelines:
%
% Each copy must include an abstract or summary of the dissertation in not
% more than 300 words, on one side of A4, which should be single-spaced in a
% font size in the range 10 to 12. If the dissertation is in a language other
% than English, an abstract in that language and an abstract in English must
% be included.

\chapter*{Abstract}
\begin{SingleSpace}
\initial{T}he challenge with modelling the excited states of light harvesting complexes
lies in scaling accurate approaches for large system sizes. Many common approaches 
to excited states are not viable on the scale of thousands of atoms, creating the 
need for more efficient methods. These methods come with compromises either in accuracy,
extendability or simplicity. In spite of this, some models have found great success 
in being able to predict and explain the nuanced mechanisms of light harvesting
complexes. 

This work explores alternatives to existing methods that overcome some of the compromises.
The design of these new methods requires an understanding of the key components of 
existing methods, which is achieved by comparing transition properties to high-level 
data from established methods. By introducing a new set of approximations to the
standard response method framework and re-optimizing key parameters in the recently
developed semi-empirical GFN-xTB method, a new approach is derived, referred to 
as Chl-xTB, that efficiently and accurately describes the \Qy transition in chlorophyll,
the most important transition to model for light harvesting complexes.

The accuracy of Chl-xTB transition properties for monomer, dimer and oligomer chlorophyll
systems is established with benchmarking and comparison to existing literature,
using a Frenkel-Davydov exciton framework for systems beyond chlorophyll monomers.
Chl-xTB is found to perform very well in these cases, giving confidence that it 
can be reliably used to predict light harvesting complex properties.

The successful application of the Chl-xTB method is showcased by calculating properties 
of the LH2 complex that are prohibitively expensive using existing techniques, such 
as electron transfer rates between charge-separated and vertical-excitation states,
as well as spectral densities of properties that encompass the entire exciton network. 
These properties shed light on the role of the LH2 protein scaffold and its manipulation
of the exciton system. For example it can be explicitly shown how the LH2 scaffold 
quells transition between vertical excitation to charge transfer states in chlorophyll 
dimers, a key mechanism in concentration quenching. Additionally, the spectral densities
of exciton states show how environmental coupling to off-diagonal elements of the
exciton Hamiltonian have little effect on the overall coupling to exciton states. 
The calculation of these novel properties and discussion of further applications 
of both the novel efficient response framework and the specific Chl-xTB method demonstrate
a possible solution to the issues of efficiently modelling light harvesting complexes.

\end{SingleSpace}
\clearpage