%
% File: chap06.tex
% Author: Oliver J. H. Feighan
% Description: Discussions 
%
\let\textcircled=\pgftextcircled
\chapter{Conclusions}
\label{chap:discussion}

\initial{T}he discussion of results in the previous chapters have analysed whether
the novel approach to designing a novel response method for LHCs has worked. In 
the first chapter it was found that while a the more efficient \dscf approach is 
viable for DFT based methods, the tight-binding \dxtb methods would not be accurate
enough for LHC models. This was argued to be due to the parameterisation of the 
GFN-xTB methods used, as other tight-binding schemes do work well. From this the
more specific chl-xTB method was designed and parameterised for \Qy transition properties.
The specificity of this model is similar to the machine-learning models discussed 
in the introduction. chl-xTB proved to be highly accurate against the selected training
data, although choices in the constructing the training data has shown some signs
of limiting this method. This is discussed in below. 

Applying this efficient response method to LHCs also proved successful. Dimer transition 
properties as well as the LH2 coupling values were well recovered by the chl-xTB 
framework, implying that conclusions about LHC (or other oligomer chlorophyll system) 
properties would be reliable for discussion. 

Having established the usability of this efficient response method for LHCs, it
was then possible to calculate novel properties. These were chosen to highlight 
how the chl-xTB exciton framework answers some of the issues discussion in the introduction
with existing LHC models. 

The first application was the rate of transition to the charge transfer state for
chlorophyll dimers, for which is was necessary to have a model that could include 
embedding effects from different environments. This was straightforward to achieve 
with chl-xTB as the point-charge embedding terms can be added into the Fock matrix. 
This is in contrast to the machine-learning models where QM/MM effects must be present 
in the training data. 

Calculating the spectral density of whole exciton state was the second application.
This relied on the accuracy of chl-xTB with respect to small changes in chlorophyll
geometry, which may not be found in more general tight-binding methods (as shown
in figure \ref{fig:energy_training_scatter}). It also required a highly efficient
response method due to the large number of individual LH2 geometries needed for 
the spectra.

Both of these applications resulted in novel explanations of LHC phenomena. The
rates of transition show how the LH2 protein purposefully inhibits the charge-transfer
quenching mechanism of excited states when compared to a more stochastic environment
such as diethyl-ether solvent. The spectral densities then investigated further 
how the environment couples to the exciton system, finding that intra-chlorophyll
variations are far more important than inter-chlorophyll interactions, a surprising
outcome given the delocalisation of exciton states across the site basis.

The investigation of novel efficient response methods and their applications reported
here was not intended to be (and is not) an exhaustive search but more a proof-of-concept
of how future models could be constructed. Some considerations for other models 
have been highlighted in the previous chapters, such as how the prediction of transition
properties from chl-xTB is highly dependent on the training data. Other considerations
are proposed below, sketching out possibilities for future work, based on some observations
that have not been part of the discussion so far. This includes how more properties 
for LHCs could be calculated using the chl-xTB exciton framework. A workflow for
designing similar methods for systems other than LH2 BChla chlorophyll is proposed,
which may be of use in studies on systems similar to LHCs.

\section{Efficient Response Methods}
\label{sec:future_response_methods}

Testing the chl-xTB response method approximations on a different system would be
the logical extension of this type of method. In chapter \ref{chap:chl_xtb} it was
posited that the success of this method was due to strong single-character transition,
a feature that has been explored in other work as well. This is also corroborated 
by the accuracy of the DFT based \dscf methods from section \ref{subsec:dscf_chl_tests}
as these are also single transition methods.

It is argued then that the response approximations and reparameterisation of GFN1-xTB
may give good results for other single-character systems and transitions. These 
investigations would have a very similar method to that used in chapter \ref{chap:chl_xtb},
constructing a set of training data from a high level method and fitting the response
method parameters. Due to the high level of accuracy achieved in chl-xTB from a
relatively small set of training data, this reparamterisation could be easily done
for other systems. If these systems are smaller then the training data could be 
made from higher level methods or from more data. 

A logical candidate for this would be the $Q_x$ transition in chlorophyll. This 
transition mostly characterised by the HOMO-1 - LUMO+1 transition so the approximations
used in chl-xTB response would be applicable. The changes to the GFN1-xTB Hamiltonian
parameters would also be the same as the transition density of this transition is
still centered on the porphyrin ring similar to the \Qy transition. Retraining for
this transition then would be a straightforward investigation to do. This would 
be a useful method to construct for some models of LHCs, especially for those more
complex than those looked at in this work. Some exciton systems expand the basis
sites and transitions to include the $Q_x$ transition, although this is only necessary
to create a more detailed image of the higher energy exciton states.

Reparameterising for other chlorophyll systems would also be a good area for further
investigation. As the LH2 protein only includes BChla molecules it was not necessary
to construct models for other types of chlorophyll. However this would no longer 
be the case when looking at other chlorophyll systems such as solvated systems or
other types of LHCs. $Q$ band transition in other chlorophyll systems are similar
to those found in BChla with changes found in the transition energies and the amount
of HOMO-LUMO and HOMO-1 - LUMO+1 character. This is dependent on the functional 
groups that are attached to the porphyrin ring.

There would be two possible approaches to alter the chl-xTB as it is reported here 
for use on other chlorophyll systems and transitions. The first approach would be 
simple reparameterisation using the parameters already reported. Not altering which
GFN1-xTB parameters are optimised would probably work best for a model that predicts
a single system and transition as has been investigated here. A more encompassing 
approach would be to include $Q_x$ transitions as well as other chlorophyll systems
into one set of training data and optimise on model. Due to the diversity of this
training data it would probably be necessary to reoptimise additional parameters,
such as a better treatment of the atom types in the porphyrin ring functional groups 
(i.e. specific C and O parameters). The range of transition energies may also cause
issues when using the objective function reported in section \ref{subsec:obj_func}.
Both the RMSE and $R^2$ values would increase with a more varied set of training 
data. It could be that the values of transition energies would form clusters around
some set of means which would artificially raise the $R^2$ value for the whole training
set and obscure the more detailed correlation for transition properties in one system.
As stated in chapter \ref{chap:chl_xtb} the correlation is an important metric for
getting good response in predicted transition properties to geometry variations.
Investigating the weights $\lambda_n$ in the objective function (eq. \ref{eq:obj_func})
may solve this issue by getting the optimisation procedure to apply higher pressure
to this metric. It is hard to argue whether a similar level of accuracy would be
achieved by including differing systems and transitions however the benefits of 
having a method that could be applied to more than one specific case are obvious.

Going beyond chlorophyll, it would be possible to test this procedure for other 
systems. One candidate relevant to LHC studies would be carotenoid units (present
in LH2) that participate in some electronic energy transfer to the chlorophyll system.
Similar to the inclusion of $Q_x$ transitions in the exciton framework some models
also include carotenoids as basis sites again for a more detailed description of
high energy states. It may be that these models would only be successful if the 
transitions are mostly single-character and have well defined transitions that can
be used in the objective function. 

Expanding the chl-xTB method to other systems as transitions would be useful for
calculating properties for more systems. For example, the spectral densities of 
other chlorophyll systems could be calculated and compared to those reported here.
Using the chl-xTB exciton framework, as well as some extensions, for further LHC
applications are discussed below.

\section{Further Investigations into LHCs}
\label{sec:lhc_investigations}

As stated above the exciton framework could be extended with additional basis sites
and transitions such as the carotenoid sites and $Q_x$ transitions. These states 
are usually higher in energy and so many not couple much to the lower exciton states
based on \Qy transitions. However they are important when considering all the mechanisms
for electronic energy transfer between states. However to achieve these models with
the similar methods used in this work would require reparameterisation as discussed
above.

The chl-xTB exciton method could be used for many more LHCs studies than those reported
here. One application would be calculating spectral densities of other LHC complexes,
such as the FMO complex, to find whether these spectra are also be dominated by
intrinsic chlorophyll properties or whether there is more effect from the protein 
manifold. The only objection to this investigation would be that, similar to the 
conclusion from section \ref{subsec:absorption_spectra}, the exclusive use of LH2
geometries in the training set data may have biased the model. This effect is almost
negligible as the agreement of predicted spectra shown in figure \ref{fig:chl_diethyl_ether}
is very good and it is only the smaller \Qy peaks that are missing, in both the
DFT reference methods as well as chl-xTB. Expanding the training set is achievable
with some considerations already discussed above.

Another key benefit of the chl-xTB response approximations, and a design choice 
that has only lightly been discussed, is that the formalism is fairly simple compared 
to full linear-response or other tight-binding methods such as sTDA-xTB. Specifically
the gradient of the response properties with respect to atomic positions would be
substantially easier to calculate than these methods, as TD-DFT gradients would 
require solving couple-perturbed equations and sTDA-xTB gradients would have a gradient
term in the basis functions due to their coordination number dependence. The gradient
of the exciton system is constructed from these site gradients similar to the Hamiltonian.
Whilst dynamics simulations of the LH2 exciton system have been done before with
gradients, several approximations were required that would not be needed if using
the chl-xTB framework. Additionally this approach required DFT level data which 
limits what could be calculated within reasonable expense. Using the chl-xTB framework 
would give a much more \emph{ab initio} approach. Calculating the gradients of chl-xTB
transition properties would be much easier than other methods. The gradient of the
transition charges, for example, would require gradients of the MO coefficients 
from the altered GFN1-xTB method. However as this method is optimised for analytic 
geometry optimisations which also require this gradient term, many programs would
readily be able to provide this. Also as the other terms in calculating the transition
energy are based on interatomic distances the gradient terms should be relatively 
easy to calculate.

Additionally as the chl-xTB exciton method is extremely efficient future investigations
could be similar to previous studies where it would have been beneficial to just 
calculate a larger volume of properties. For example a larger model of the entire
light-harvesting apparatus present in purple bacteria could be achieved, or investigating
the effects on EET when multiple LH2 proteins are present. These investigations 
would not add anything more to the model but highlight how increased efficiency 
allows larger scale models to become feasible.


These suggestions for future work highlight the useful features of the chl-xTB monomer
method as well as the chl-xTB exciton framework. As there are many potential areas
for further research it is argued that these novel solutions to the problem of scale
in LHC models introduced in chapter \ref{chap:intro} will be useful for inspiring
similar approach in the future. The efficiency of tight-binding methods has been
well utilised and would allow for larger models to be possible. The more extendable
formalism, achieved by being based on electronic structure methods and requiring 
few parameters, would allow for more many more properties and effects to be taken
into account without the need for large change. Finally many of the transition properties
are reliable due to the decent accuracy achieved by optimisation to a more specific
training set than other methods, a feature which should also be true for cases beyond
the \Qy chlorophyll transition. These approaches should therefore be useful tools
for the future investigations in LHCs that are necessary to fully explore photosynthesis
and light harvesting mechanisms. 