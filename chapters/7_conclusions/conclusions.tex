%
% File: chap06.tex
% Author: Oliver J. H. Feighan
% Description: Discussions 
%
\let\textcircled=\pgftextcircled
\chapter{Conclusions}
\label{chap:discussion}

\initial{T}he discussions of results in the previous chapters analyzed whether novel
response methods for LHCs are viable alternatives to other methods proposed in the
literature. In the first chapter it was found that while the more efficient \dscf 
approach is accurate when using DFT based methods, \dxtb methods are not accurate 
enough for LHC models. This shortcoming was argued to be due to the ground-state
property parameterization of the GFN-xTB methods as other tight-binding schemes 
do work well when parameterized for transition properties. Running with this conclusion, 
it followed that a bespoke parameterisation of the tight-binding electronic structure
theory and excited state formalism may be beneficial, giving the chl-xTB method 
designed and parameterized specifically for \Qy transition properties. The specificity
of this model is similar to the machine-learning models discussed in the introduction, 
but relies on electronic structure theory rather than any machine-learning method.
It is also much easier to parameterize. Chl-xTB proved to be highly accurate against 
the selected training data, although choices in constructing the training data set 
have shown some signs of limiting this method.

While targeted parameterization gives Chl-xTB a very specific application, the underlying
physical model means it can be more easily extended or interfaced into larger multiscale
simulations than machine learning models, which are strictly limited to describing 
the properties and situations they were trained for. This protocol also makes far 
more efficient use of parameters than a typical ML model, requiring only a handful 
of constants that all have specific chemical meaning (compared to tens of thousands
used in a supervised learning model such as the one by H\"{a}se \emph{et al.})\cite{AspuruGuzik2016}.

Using this efficient response method for work with LHCs also proved successful. 
Dimer transition properties as well as the LH2 coupling values were well recovered 
by the Chl-xTB framework, implying that conclusions about LHC (or other oligomer
chlorophyll system) properties would be reliable.

Having established the usability of this efficient response method for LHCs, it
was then possible to calculate novel properties. The free energy surfaces and spectral
densities posed a challenge to established methods for excited states due to the
massive volume of calculations required, coupled with the accuracy and extensive 
detail necessary.

Both of the applications of the Chl-xTB exciton framework have overcome these challenges
and resulted in novel observations of LHC phenomena. The rates of ES-CS transition 
show how the LH2 protein purposefully inhibits the charge-separation quenching mechanism
of vertical excited states when compared to a more disorganised environment such as
 diethyl-ether solvent. The spectral densities then investigated further how the
environment couples to the exciton system, finding that intra-chlorophyll variations
are far more important than inter-chlorophyll interactions. This outcome is surprising 
given the emphasis on accuracy in inter-chromophore and environment-chromophore coupling.

The investigation of novel efficient response methods and their applications reported
here was not intended to be (and is not) an exhaustive search but more a proof-of-concept
of how future models could be constructed. Some considerations for other models 
have been highlighted in the previous chapters, such as how the prediction of transition
properties from Chl-xTB is highly dependent on the training data. Other considerations
are proposed below, sketching out possibilities for future work based on some observations
that have not been part of the discussion so far. This includes how more properties 
for LHCs could be calculated using the Chl-xTB exciton framework. A workflow for
designing similar methods for systems other than LH2 BChla chlorophyll is proposed,
which may be of use in studies on systems similar to LHCs.

\section{Efficient Excited State Methods}
\label{sec:future_response_methods}

Testing the Chl-xTB excited state approximations on different systems would be
the logical extension of this type of method. In chapter \ref{chap:chl_xtb} it was
posited that the success of this method was due to a strong single-character transition,
a feature that has been explored in other work as well. This is also corroborated 
by the accuracy of the DFT based \dscf methods from section \ref{subsec:dscf_chl_tests}
as these are also single transition methods.

It is argued then that the response approximations and reparameterisation of Chl-xTB
may give good results for other single-character systems and transitions. These 
investigations would have a very similar method to that used in chapter \ref{chap:chl_xtb},
constructing a set of training data from a high level method and fitting the response
method parameters. Due to the high level of accuracy achieved in Chl-xTB from a
relatively small set of training data, this reparameterisation could be easily done
for other systems. If these systems are smaller, then the training data could be
made from higher level methods or from more data. 

A logical candidate for strategy this would be the $Q_x$ transition in chlorophyll.
This transition is dominated by the HOMO-1 - LUMO+1 transition so the approximations
used in Chl-xTB response would be applicable. The changes to the GFN1-xTB Hamiltonian
parameters would also be the same as the transition density of this transition is
still centered on the porphyrin ring similar to the \Qy transition. Retraining for
this transition then would be a straightforward investigation to do. This would 
be a useful method to construct some models of LHCs, especially for those more complex
than those looked at in this work. Some exciton systems expand the basis sites and
transitions to include the $Q_x$ transition, although this would only be necessary
to create a more detailed image of the higher energy exciton states.

Reparameterizing for other chlorophyll systems would also be a good area for further
investigation. As the LH2 protein only includes BChla molecules it was not necessary
to construct models for other types of chlorophyll. However this would no longer 
be the case when looking at other chlorophyll systems such as solvated systems or
other types of LHCs. $Q$ band transitions in other chlorophyll systems are similar
to those found in BChla with changes found in the transition energies and the amount
of HOMO-LUMO and HOMO-1 - LUMO+1 character. This is dependent on the functional 
groups that are attached to the porphyrin ring.

There would be two possible approaches to reoptimize the Chl-xTB parameters for 
uses on other chlorophyll systems and transitions. The first approach would be 
simple reparameterization using the parameters already reported. Not altering which
GFN1-xTB parameters are optimized would probably work best for a model that predicts
a single system and transition as has been investigated here. A more encompassing 
approach would be to include $Q_x$ transitions as well as other chlorophyll systems
into one set of training data and optimize one model. Due to the diversity of this
training data it would probably be necessary to reoptimize additional parameters,
such as a better treatment of the atom types in the porphyrin ring functional groups 
(i.e. specific C and O parameters). The range of transition energies may also cause
issues when using the objective function reported in section \ref{subsec:obj_func}.
Both the RMSE and $R^2$ values would increase when a more varied set of training 
data is used but this increase would not be reflective of greater accuracy. The 
values of transition energies would form clusters around some set of means which
would artificially raise the $R^2$ value for the whole training set and obscure
the more detailed correlation for transition properties in one system. As stated
in chapter \ref{chap:chl_xtb} the correlation is an important metric for getting 
a good description of the relationship between transition properties and geometry
variations. Investigating the weights $\lambda_n$ in the objective function (eq. \ref{eq:obj_func})
may solve this issue by getting the optimization procedure to apply higher pressure
to this metric. It is hard to argue whether a similar level of accuracy would be
achieved by including differing systems and transitions, however the benefits of 
having a method that could be applied to more than one specific case are obvious.

Going beyond chlorophyll, it would be possible to test this procedure for other 
systems. One candidate relevant to LHC studies would be carotenoid units (present
in LH2) that participate in some electronic energy transfer to the chlorophyll system.
Similar to the inclusion of $Q_x$ transitions in the exciton framework, some models
also include carotenoids as basis sites, again for a more detailed description of
high energy states \cite{Polli2006, Andreussi2015}. 

Future applications to molecules other than chlorophyll is facilitated by the deliberate
use of widely available computational tools in the parameterization workflow (the 
open-source GFN1-xTB implementation and the SLSQP optimization procedure, which 
is available in most standard statistical software packages, such as Python's \texttt{SciPy}). 
The approximation of a diagonally dominant $\textbf{A}$ matrix used to derive transition
energies limits the approach to transitions that are dominated by a single orbital 
excitation. While scaling the transition density can recover a limited amount of 
mixed-transition character, we would not expect transitions consisting of a mix
of degenerate excitations to work well. However, transitions with single-excitation
character are relevant to a num- ber of interesting biological molecules. For example, 
both the S2 excited state in carotenoids and the Q band in bilins (two of the other
important pigments in LH2 complexes) are almost entirely HOMO $\rightarrow$ LUMO transitions 
in character\cite{Tracy2022, Matute2010, Scholes2013, Coccia2014}. In general, larger
molecules that are too expensive to treat with TD-DFT (and there- fore require a 
cheaper method) are often not symmetric enough to have much degeneracy, so are more 
likely to have transitions that would be suitable for describing with the type of 
protocol outlined here.

Another obvious extension of the Chl-xTB protocol is to retrain it to a different 
reference method, such as the long-range corrected hybrid functional CAM-B3LYP, 
although this may require changes to equation \ref{eq:gammaJ} and \ref{eq:gammaK} 
to account for bifurcating the Coulomb and exchange interactions into short- and
long-range terms. Altering the formalism to reflect the structure of more complicated 
functionals would not preclude the approximations that make Chl-xTB fast and efficient. 
Thus we anticipate that, with only minor adjustments, the Chl-xTB protocol can be 
adapted to different levels of theory to suit the system of interest.

Expanding the Chl-xTB method to other systems as transitions would be useful for
calculating properties for more systems. For example, the spectral densities of 
other chlorophyll systems could be calculated and compared to those reported here.
Using the Chl-xTB exciton framework, as well as some extensions, for further LHC
applications are discussed below.

\section{Further Investigations into LHCs}
\label{sec:lhc_investigations}

As stated above the exciton framework could be extended with additional basis sites
and transitions such as the carotenoid sites and $Q_x$ transitions. These states 
are usually higher in energy and so may not couple much to the lower exciton states
based on \Qy transitions. However they are important when considering all the mechanisms
for electronic energy transfer between states \cite{Polli2006}. To achieve these
models with the similar methods used in this work would require reparameterization 
as discussed above.

The Chl-xTB exciton method without modifications could still be used for many more
LHC studies than those reported here. One application would be calculating spectral 
densities of other LHC complexes, such as the FMO complex, to find whether these 
spectra are also dominated by intrinsic chlorophyll properties or whether there 
is more effect from the protein manifold. The only objection to this investigation 
would be that, similar to the conclusion from section \ref{subsec:absorption_spectra}, 
by only training to LH2 bacteriochlorophyll structures, which are constrained by
protein binding pockets, the current form of Chl-xTB has the potential to misrepresent
other geometries that would occur in different environments. This effect is almost 
negligible as the agreement of predicted spectra shown in figure \ref{fig:chl_diethyl_ether} 
is very good and it is only the smaller \Qy peaks that are missing, in both the 
DFT reference methods as well as Chl-xTB. Expanding the training set is achievable 
with some considerations already discussed above.

Another key benefit of the Chl-xTB excited state approximations, and a design choice 
that has only been lightly discussed, is that the formalism is fairly simple compared 
to full linear-response or other tight-binding methods such as sTDA-xTB. Specifically
the gradient of the response properties with respect to atomic positions would be
substantially easier to calculate than other methods. For example TD-DFT gradients 
would require solving couple-perturbed equations and sTDA-xTB gradients would have 
basis function gradient terms due to their coordination number dependence. The gradient
of the exciton system is constructed from these site gradients similar to the Hamiltonian.
Whilst dynamics simulations of the LH2 exciton system have been done before with
gradients, several approximations were required that would not be needed if using
the Chl-xTB framework \cite{Sisto2017}. Additionally this approach required DFT 
level data which limits what could be calculated within reasonable expense. Using 
the Chl-xTB framework would give a much more \emph{ab initio} approach. Calculating 
the gradients of Chl-xTB transition properties would be much easier than other methods.
The gradient of the transition charges, for example, would require gradients of 
the MO coefficients from the altered GFN1-xTB method. However as this method is 
optimized for analytic geometry optimizations which also require this gradient term,
many programs would be able to readily provide this. Also as the other terms in 
calculating the transition energy are based on interatomic distances the gradient 
terms should be relatively easy to calculate.

Additionally, as the Chl-xTB exciton method is extremely efficient, future investigations
could be similar to previous studies where it would have been beneficial to just 
calculate a larger volume of properties. For example a larger model of the entire
light-harvesting apparatus present in purple bacteria could be achieved, or investigating
the effects on EET for multiple LH2 proteins. These investigations would not add
anything more to the model but highlight how increased efficiency allows larger
scale models to become feasible.

These suggestions for future work highlight the useful features of the Chl-xTB monomer
method as well as the Chl-xTB exciton framework. As there are many potential areas
for further research it is argued that these novel solutions to the problem of scale
in LHC models introduced in chapter \ref{chap:intro} will be useful for inspiring
similar approach in the future. The efficiency of tight-binding methods has been
well utilized and would allow larger models to be possible. The more extendable
formalism, based on electronic structure methods and requiring few parameters, would
allow many more properties and effects to be taken into account without the need
for large change. Finally, many of the transition properties are reliable due to
the decent accuracy achieved by optimization to a more specific training set than
other methods, a feature which should also be true for cases beyond the \Qy chlorophyll
transition. These approaches should therefore be useful tools for the future investigations
in LHCs that are necessary to fully explore photosynthesis and light harvesting
mechanisms. 