%
% File: chap05.tex
% Author: Oliver J. H. Feighan
% Description: LHII Spectral Density
%
\let\textcircled=\pgftextcircled
\chapter{Light Harvesting Complexes}
\label{chap:LH2}

\subsubsection*{Previous Published Work}
All of the work presented in this chapter are also included in a paper published 
with Dr. Susannah Bourne-Worster and Prof. Fred Manby in January 2023 \cite{Feighan2023}.

\initial {I}n this chapter further large scale properties of LHCs are explored with
the Chl-xTB exciton framework. As opposed to the previous chapter where only the 
chlorophyll dimer case was investigated, this chapter examines the interplay between
the entire chlorophyll exciton network and the protein environment.

As seen in the previous chapter, the protein scaffold and chromophore geometry conformation
have direct effect on the rates of EET. Similarly dynamic properties of these systems
will effect EET, as the properties of chlorophyll chromophores in the LH2 complex
are not static. Motion (e.g., vibrations) of the chromophores themselves and of 
the protein complex constantly changes the conformational energy of the chromophores,
their interactions with each other, and their interactions with their protein environment, 
creating continual fluctuations in the excitonic energies of the chromophore network. 
These fluctuations drive relaxation processes, which lead to spectroscopic line-broadening 
and heavily influence the ability of the chromophore antenna network to efficiently 
transport energy.

This important environmental influence can be encapsulated in a single parameter 
called the spectral density, which describes the coupling between vibrations (of 
the chromophores and environment) and excitonic energies.

The spectral density describes how the oscillations in the surrounding environment 
can effect a property. Calculating spectral densities for transitions energies is
usually used when looking at light harvesting complexes to assign functions to the 
protein structure, as well as necessary for population dynamics \cite{Mallus2018}.

This important environmental influence can be encapsulated in a single parameter 
called the spectral density, which describes the coupling between vibrations (of 
the chromophores and environment) and excitonic energies. Spectral densities are
commonly defined as the half-sided Fourier transform of the autocorrelation function 
of a time series for the property of interest (usually a specific energy). For example,
the spectral density for a  transition energy $\Delta E$ is 
\begin{equation}
    J\left(\omega\right) = \frac{\beta \omega}{\pi} \int^\infty_0 dt \, C\left(t\right) \cos \left(\omega t\right), \label{eq:specden}
\end{equation}
%
where $\omega$ is frequency and 
\begin{equation}
    C\left(t_j\right) = \frac{1}{N-j} \sum^{N-j}_{k=1} \Delta E \left(t_j + t_k\right) \Delta E \left(t_k\right)
\end{equation}
%
is the autocorrelation function in the time domain. The indices $j,k$ label frames
in the time series of $\Delta E$ (usually a series of energies calculated along 
consecutive frames of a molecular dynamics trajectory) and $N$ is the total number 
of frames. The prefactor $\frac{\beta \omega}{\pi}$ in equation \ref{eq:specden} 
is required to maintain detailed balance, with $\beta = \frac{1}{k_B T}$ and $T= 300 K$.

The ability to capture coupling to vibrational modes of a particular frequency depends 
on the length ($N$) and frequency ($1 / \left(t_k-t_j\right)$) of sampling across 
the time series. Specifically, the upper frequency limit is $\frac{1}{2\left(t_k-t_j\right)}$, 
whilst the lower limit is $\frac{1}{2N}$. In order for the spectral density to cover 
high-frequency vibrations (e.g. vibrations of hydrogen bonds), as well as low frequency 
vibrations (e.g. large-scale motion of the protein complex), transition energies 
have to be sampled over the long time scale of protein motion (10s -- 100s of picoseconds)
at a femtosecond or even sub-femtosecond resolution.

With current electronic structure methods and computing power, this is an almost 
prohibitively large volume of data to acquire. Consequently, most spectral densities
calculated from MD trajectories only cover a small frequency range. Recent examples 
for bilin \cite{Blau2018} and Nile red chromophores \cite{Zuehlsdorff2022} (both 
smaller than bacteriochlorophyll), calculated at a TD-DFT level of theory (B3LYP/aug-cc-pVDZ 
and B3LYP/cc-pVTZ for bilin chromophores and CAM-B3LYP/6-31G* for Nile red chromophores), 
sample high frequency molecular motion with a sampling rate of 0.5 $\mathrm{fs}^{-1}$ 
but do not simulate beyond a few picoseconds (6 ps and 40 ps for bilin and Nile red 
simulations respectively) so cannot capture lower frequency modes.

They nevertheless required months of GPU time. Switching to lower level electronic
structure methods, such as ZINDO, give access to longer timescales (Mallus \emph{et al.}
achieve 300 ps at 2 fs intervals\cite{Mallus2018}) but still remain too expensive 
to treat the entire exciton framework collectively. In the absence of affordable 
\emph{ab initio} spectral densities, a popular alternative is to use experimentally 
parameterised model spectral densities\cite{Ishizaki2009,Renger2002,Padula2017}, 
but even these can be difficult to obtain and only offer an approximate, coarse-grained 
picture. 

The speed and accuracy of Chl-xTB, particularly in describing the relative excitation 
energies of closely related geometries (see figures \ref{fig:mode_83} through to \ref{fig:mode_135}), 
addresses this issue by making it feasible to sample transition energies over long
timescales at high resolution. Using this method it becomes possible to calculate 
spectral densities that not only cover a broader frequency range than was previously
possible but also for the exciton state energies of the entire LH2 antenna complex 
(rather than just the transition energies on individual chromophore sites).

This chapter tries to answer two questions about environmental frequency coupling
to the exciton states. The first is whether low frequency modes are present in the
spectral density, which would indicate a large scale motion of the protein scaffold.
It can be seen on AFM images of the LH2 protein reveal that there can be some deformations
of the circular symmetry, which could be due to ring-breathing vibrational modes 
in the chlorophyll site positions. It is not fully understood whether this would
have an effect on the exciton states. The second questions would be by how much 
is the exciton state spectral density different to that of individual chlorophyll 
molecules (inside the LH2 protein). This would be an indicator of whether the protein
structure promotes some special interaction between sites, or if the protein only
affects intra-chromophore geometry. The amount of intra- verses inter-site difference
would inform how much detail is needed for future studies on LH2. It would also 
help in the design of artificial biomimetic systems.

\section{Calculating LH2 Excitons}
\label{sec:MD}

\subsection{Molecular Dynamics}
\label{subsec:specdens_MD}

Exciton states were calculated from a series of structures of LH2 and chlorophyll
generated by molecular dynamics simulations run with the OpenMM package. The force-field 
and geometry were the same as used by Ramos \emph{et al.} \cite{Mennucci2019}. The
system was equilibrated for 60 ps, before starting a production workflow of 300 ps, 
with frames taken every 2 fs similar to the work done by Mallus \emph{et al.} \cite{Mallus2018}.
A Langevin integrator set at 300 K was used with a timestep of 2 fs. Non-bonded
interactions were treated with a particle mesh Ewald method.

\subsection{Scaling computational performance}
\label{subsec:specdens_scaling}

Exciton states were calculated with the same method as used in the previous chapter
(encapsulated in equation \ref{eq:exciton_H_matrix}), but extending beyond the dimer 
system to include all chlorophyll sites. The elements in the exciton Hamiltonian 
were calculated using Chl-xTB properties. It was necessary to implement a highly
parallelized version of the code to calculate Chl-xTB properties efficiently. It
was found that using a parent program to partition each chlorophyll site to run 
in serial on a single core had the best performance, with the data then being collected
back into a separate program to construct the exciton Hamiltonian. As the high performance
computing machines available had around 20 cores, the compute wall-time of each 
exciton run was on the order of a single chl-xTB calculation.

The main bottleneck in analysing these states was the volume of Chl-xTB properties
needed, as it was found construction and diagonalization of the exciton Hamiltonian
was negligible. A 300 ps MD simulation, saving frames every 2 fs, generates 150,000
individual frames requiring 4,050,000 individual Chl-xTB calculations on distinct 
geometries to construct the exciton Hamiltonian. The time for each Chl-xTB calculation
is $\approx 0.5$ s (see figure \ref{fig:method_timing}), so this presents about
46 days of serial CPU time. This was significantly reduced by using the highly parallelized
program, to $\sim$ 30 hours. Whilst have this level of detail is necessary for spectral
density investigations, other properties of the LH2 system that can be obtained 
with more statistical methods may be a more efficient use of resources than the 
approach used here.

\subsection{Coupling Values}
\label{subsec:coupling_values}

A summary of the coupling values, broken down by ring-interaction type, are shown
in figure \ref{fig:coupling_all}. The coupling values obtained from the chl-xTB 
exciton framework match those of previous studies well. Coupling values are highest 
for B850-B850 interactions, distributed around a centre of 350 $\text{cm}^{-1}$,
which corresponds well with previously used values between 238 and 771 $\text{cm}^{-1}$ \cite{Cogdell2006}.
Intra-dimer (where the dimer is a subunit of LH2 chlorophylls consisting of a B850a,
B850b and B800 chlorophyll) couplings are markedly stronger than inter-dimer couplings, 
which also corresponds well with previous estimated of inter-dimer couplings being
~100 $\text{cm}^{-1}$ less than intra-dimer couplings \cite{Koolhaas1997, Jimenez1996, Chachisvilis1997, Scholes2000}. 
Inter-ring couplings are much lower, with distributions around 35 $\text{cm}^{-1}$
and 10 $\text{cm}^{-1}$, consistent with previous estimates of between 24 and 31 
$\text{cm}^{-1}$ \cite{Krueger1998}.

\begin{figure}
    \centering
    \includegraphics[]{../../Year_3/SpectralDensity/images/coupling_all.png}
    \caption{a) a diagrammatic representation of coupling strengths and interactions
    between sites specified by ring labels, with thicker lines indicating stronger
    exciton coupling interactions. b) Exciton coupling values (in $\text{cm}^{-1}$)
    plotted as a function of distance between chlorophyll sites, illustrating the
    distance-dependent cutoff. c) Probability distributions of B850 coupling values,
    showing the difference in strengths between intra and inter dimer pairs. d) 
    Probability distribution of B800-B850 interactions.}
    \label{fig:coupling_all}
\end{figure}

\subsection{Screening and Embedding}
\label{subsec:screening}

Recent studies that calculate exciton models for light harvesting systems employ
screening factors as well as point charge interactions to compensate for some of
the embedding effects of the protein scaffold. These were also tested, but ultimately
not used for the spectral density in favour of a vacuum model.

The point charge embedding used in the previous chapter for LH2 dimers was implemented 
with the required PME periodic terms using the open source \code{helPME} library.
The CPU time of the PME terms was profiled using the LH2 MD frames, and it was found
that due to implementation problems with re-using splines and grid positions the
time for each chl-xTB calculation increased from $\sim 1$ second to $\sim 40$ seconds. 
Additionally the PME functions required multiple cores in order to get the best 
performance, which competed with the parallelization of chl-xTB runs. As improving 
PME implementations are out of the scope of this work (beyond what was already done
to achieve the work in the last chapter), and considering that the embedding only
marginally changes the \Qy transition, this embedding scheme was not applied to 
the exciton system used in this chapter.

A screening term was also implemented. This followed the same form as that reported
by Mallus \emph{et al.} \cite{Mallus2018}, where any point-charge interaction has
a prefactor screening term that is dependent on the distance between point charges.
For example for a coupling term between two exciton states, a single element in 
the sum of chromophore-chromophore interactions, would be given by

\begin{equation}
    V_{\left(m, 1\right), \left(n,1\right)} = \sum_{A \in m, B \in n} \frac{f}{4\pi\epsilon_0} \frac{q^{\text{tr},A}_m q^{\text{tr},B}_n}{r_{AB}}
    \label{eq:pc_exciton_coupling}
\end{equation}
%
where the definitions of variables are the same for equation \ref{eq:exciton_coupling}.
The scaling factor $f$ is given by

\begin{equation}
    f = A \text{exp}\left(-B R_{ij}\right) + f_0
\end{equation}
%
where $A$, $B$ and $f_0$ are constants. 

When employing this screening factor, it was found the only effect was to decouple
the B800 and B850 exciton states. This is best demonstrated in the simulated absorption
spectra for LH2 with and without the screening factor, as well as the breakdown 
of exciton states into the density contributions from each site.

\subsubsection{Absorption Spectra}
\label{subsubsec:abs_spec}

Absorption spectra of LH2 are usually simulated by calculating the intensities of 
transitions for each exciton state and plotting these against the wavelengths of
transitions to these states. 

The intensity of transition $I_k$ from the ground exciton state $\Psi_0$ to a (one)
exciton state $\Psi_k$ is given by

\begin{equation}
    I_k \varpropto E_k \left\lvert \braket{\Psi_k| \hat{\epsilon} \cdot \mathbf{\mu} | \Psi_0} \right\rvert^2
\end{equation}
%
where $E_k$ is the energy of the state $\braket{\Psi_k|H|\Psi_k}$, $\hat{\epsilon}$
is a unit vector in the direction of the polarisation of the incident light, which
to mimic sunlight should follow a random distribution. As the one-exciton states 
are mostly at the same energy the energy factor $E_k$ can be neglected. The overlap
term can then be expanded into the monomer basis giving

\begin{equation}
    I_k \varpropto \left\lvert \sum^N_{j=1} c_{kj} \hat{\epsilon \cdot \mathbf{\mu}_j }\right\rvert^2
\end{equation}
%
where $j,N$ are the index and total number of chlorophyll sites respectively, $c_{kj}$
is the eigenvector coefficient of state $k$ at site $j$, and $\mathbf{\mu_j}$ is 
the transition dipole moment of chlorophyll $j$ 

\begin{equation}
    \mathbf{\mu}_j = \braket{\phi_j^{\left(1\right)}|\mathbf{\mu}|\phi_j^{\left(0\right)}}
\end{equation}
%
As this intensity is dependent on the orientation of the unit vector $\epsilon$,
the average intensity can be found by either calculating the intensities for a large
distribution of randomized unit vectors and taking the average, or by taking the
analytic spherical average. In the large number limit, the statistical method converges
to the analytic answer.

The simulated LH2 spectra with and without screening factors, alongside the experimental 
spectrum, are shown in figure \ref{fig:LH2_abs_spec}. It can be seen that including
the screening factor does produce better splitting of the B800 and B850 peaks. The
poor fit of the B800 peak to the experimental spectrum has been well discussed in
the literature \cite{Stross2016, Mennucci2019}. A lack of deformation of the ring 
structures is the most probable reason for the poor fit of the B800 peak in both 
screened and unscreened spectra.

\begin{figure}
    \centering
    \includegraphics[scale=0.6]{../../Year_3/SpectralDensity/images/absorption_spectra.png}
    \caption{Simulated absorption spectra of LH2, with (dashed line) and without 
    (solid line) a screening factor for point charge interaction. An experimental
    line reconstructed from Strain \emph{et al.} \cite{Strain1963} is plotted in
    black. The simulated spectra are shifted to match the position of the 850 nm
    peak.}
    \label{fig:LH2_abs_spec}
\end{figure}

The lack of different features in the absorption spectra implies that the only effect
of using the screening terms is to lower the B800 energy states to better fit the
experimental spectrum. This would be due to the inclusion of B850 site character
into exciton states localized in the B800 ring, but the lack of significantly differing
features implies this mixing may be minimal. The next section takes a more detailed
look at the density contributions of each site to the exciton states.

\subsubsection{Site Contributions to Exciton Density}
\label{subsubsec:site_dens}

The eigenvector solutions contain the amount of site character in each state, with
the square of this being equal to the exciton density on a specific site. For example,
the second element in each eigenvector is the corresponding amount of character 
from an excitation on the first chlorophyll site (the first element corresponding
to the ground state, with no excitation on any chlorophylls). The average of density
values is shown in figure \ref{fig:LH2_density} for both exciton states calculated
with and without the screening factor, as well as the difference between them. 
Overall, the density contributions show that the ring structures are already significantly
decoupled in the vacuum model, with the screening terms only adding slightly to 
this effect.

\begin{figure}
    \centering
    \includegraphics[scale=0.8]{../../Year_3/SpectralDensity/images/screened_density_diff.png}
    \caption{Density contributions of site transitions to exciton states for the
    vacuum (left) and screened (middle) Hamiltonian, with the difference shown in
    on the right.}
    \label{fig:LH2_density}
\end{figure}

The main difference is in the amount of density shared between states localized 
on B800 and B850 sites. The plot of the density difference shows how density is 
localized more on B800 sites for exciton states calculated with the screening
factor. This is also true for the B850 sites. However it does not reduce the delocalization
of intra-ring sites - the amount of density shared between B850 sites with B850 
sites stays effectively the same. Even without the screening factor, most states
are made of either only B800 or B850 character, with little mixing found.

Overall, whilst is was possible to include embedding effects into the chl-xTB exciton
model, they may not change the overall qualitative behavior of the exciton states.
Calculating the chlorophyll system in a vacuum would still be a valid choice of
model, and this model has shown to be been effective in the previous investigations.

\section{Long Timescale Spectral Densities of the LH2 Protein}
\label{sec:sites_states_couplings}

The following section reports on the features of spectral densities calculated using
the chl-xTB and exciton method. A more detailed breakdown of peak positions, heights
and assignments can be found in appendix \ref{app:app02}.

\subsection{Spectral Densities of \Qy Transition Energies at Individual Sites}
\label{subsec:sites}

In order to confirm the spectral densities calculated with Chl-xTB excitons are 
reasonable, spectral densities of monomer chlorophyll \Qy transitions were benchmarked 
against literature data. This benchmarking was done for a B800 site of LH2 chlorophyll,
and the spectrum can be seen in figure \ref{fig:specdens_lit}.

\begin{figure}
    \centering
    \includegraphics[scale=0.8]{../../Year_3/SpectralDensity/images/specdens_literature_comparison.png}
    \caption{Spectral density of the \Qy transition for a chlorophyll in the B800
    ring in LH2. Axis scaling and units are chosen to best reproduce the spectrum
    reported by Mallus \emph{et al.}\cite{Mallus2018}.}
    \label{fig:specdens_lit}
\end{figure}

The use of eV as units of the spectral density and well as frequency is consistent 
with previous literature reports. This first spectrum is compared to the spectral 
density reported by Mallus  \emph{et al.} \cite{Mallus2018}, and uses the same axis 
scaling (max of 0.03 eV for $J\left(\omega\right)$ and 0.1 for $\hbar \omega$). 
Similar features such as the peaks at 0.09, 0.08 and 0.06 eV are found in both. 
The peak at 0.022 eV is not found in the literature comparison. The $J\left(\omega\right)$ 
values of the 0.08 and 0.06 peaks match well, although the 0.09 eV peaks is smaller
in this work.
These small discrepancies can be explained by the different force-fields and response
methods used. It should be noted that this is in the low frequency region of the
spectral density, and the major features are found at much higher frequencies. Features
in this region are attributed to environmental forces \cite{Mallus2018}, and so 
difference in the force-field and MD methods used would be expected to cause small
differences in the spectra. However the good match of peak positions and the generally
important features indicate that using chl-xTB can reproduce spectral densities
from higher level methods quite well.

The spectrum shown in figure \ref{fig:specdens_sites} shows an expanded range of
frequencies up to 0.25 eV, as well as a breakdown by ring type. The three spectra
here are calculated as the average for sites in the B800, B850a and B850b rings.
Whilst many peak positions are similar in all three types, there are differences
in $J\left(\omega\right)$ values that indicate differences in protein environment.
All three show a major feature at 0.165 eV, with magnitudes of around 0.2 eV. The 
next set of largest features in around the 0.2 eV mark. Here differences in the
coupling can be seen. For example the 0.21 eV peak present in B850a and B850b sites,
the largest feature after the 0.165 eV at a magnitude of ~0.185 eV, is not present
in the B800 sites. Similarly the 0.205 eV (height 0.117 eV) present in B800 is not
found in the B850 spectra. A full comparison of peak positions and heights can be
found in appendix \ref{app:app02}. For features found in both ring type sites, it
is generally the case B800 values are higher than B850, which is in line with previous
arguments that the more polar environment around B800 leads to greater variations 
in transition energy \cite{Olbrich2010}.

\begin{figure}
    \centering
    \includegraphics[scale=0.8, angle=90, origin=c]{../../Year_3/SpectralDensity/images/specdens_ring_average.png}
    \caption{Average spectral densities for \Qy transitions at sites in the B800,
    B850a and B850b rings.}
    \label{fig:specdens_sites}
\end{figure}

Generally it can be seen that spectral densities calculated with the chl-xTB method
correspond well with previously reported observations. Major features appear at
previously reported frequencies and magnitudes, and environmental effects are reproduced
well. It is discussed later how the efficiency of the chl-xTB method would allow
better workflows in calculating spectral densities.

\afterpartskip
\subsection{Spectral Densities of Exciton State Energy}
\label{subsec:states}

This section reports on the calculation of the spectral density for exciton states.
The work done here differs from similar work previously reported in the literature
as it was possible to perform electronic structure calculations for every chlorophyll
geometry in every frame of the MD simulation. Due to the strong correlation of geometry
variations to \Qy transition property variations, it is expected that the features 
in the spectral density are representative of the environmental coupling to the 
exciton states, rather than any artificial features from a statistical method. It
was found that the spectral densities of exciton transition energies closely match
the monomer site spectra, implying that the environment coupling mainly affects
intra-chlorophyll properties and not the exciton coupling values.

\begin{figure}
    \centering
    \includegraphics[scale=0.6]{../../Year_3/SpectralDensity/images/abs_prob_by_energy.png}
    \caption{Violin plots of absorption probabilities for exciton states, positioned
    by the average transition energy from the ground state. The width of each distribution 
    indicates the density of values at absorption probabilities, with crosses marking
    the mean value.}
    \label{fig:absorption_probabilities}
\end{figure}

\begin{figure}
    \centering
    \includegraphics[scale=0.8, angle=90, origin=c]{../../Year_3/SpectralDensity/images/specdens_state_spectra.png}
    \caption{Spectral densities of exciton transition energies of LH2, weighted by
    the averaged absorption probability. The highest energy state is indicated in 
    red. This state has a higher fluctuation in energy, causing excess noise in
    the spectral density.}
    \label{fig:specdens_states}
\end{figure}

The exciton spectral density was calculated using the time series of the exciton
transition energies (calculated as the difference between a given state and the
exciton ground state) of each state (bar the ground state). The spectra shown in 
figure \ref{fig:specdens_states} have been weighted in colour by their time-average
absorption probabilities (shown in figure \ref{fig:absorption_probabilities}) to
clearly present the features in the spectra. From figure \ref{fig:absorption_probabilities}
it can be seen that there are two "bright" states, which are the 2nd and 3rd lowest
in transition energy. These absorption probabilities only consider ground to excited 
transitions, and not the transition probability between exciton states. There is
one outlier state in these spectra, marked in red - this corresponds to the highest
energy state, and the increases in $J\left(\omega\right)$ values are attributed 
to increased variations in this state's energy, which would exacerbate the noise 
in the spectrum. Discussion of this state is limited due to this effect, as well 
as the fact that it may not be the most important state to consider as the absorption
probability to this state is the lowest.

Looking at the 2nd lowest energy state, the main features appear at near identical
frequency positions as the site spectra. The largest feature is at 0.165 eV, with
a magnitude of 12.183 meV. Similarly, peaks at 0.211 eV and 0.218 eV have large
amplitudes at 6.775 meV and 8.62 meV respectively. Full assignments can be found
in appendix \ref{app:app02}.

From the change in scale on the y-axis, it is obvious that the environmental effect
on exciton states is much smaller that for sites. This is explained by the lack
of correlation between chlorophyll motions, generally cancelling out any variation
in \Qy transition properties, mostly staying close to the mean. This would reduce
the fluctuation of exciton transition energies, which in turn would decrease the
magnitude of any peak in the spectral density.

The similarity of the exciton state and the site spectra implies that the protein
environment effect on the transition at a single chlorophyll site is much greater 
than any effect on the exciton coupling between chlorophyll sites. What this indicates
is the lack of any large scale environmental effects on multiple chlorophyll sites,
and that all of the environment effects are localized at each site. This localization
of effects does not exclude the possibility that exciton coupling variations can 
be present, just that these would most likely be due to intra-chlorophyll variations, 
and not, for example, a large scale movement of the protein structure to bring chlorophylls
closer together or change the angle between porphyrin planes. The next section looks 
at the variations in coupling values in greater detail.

\afterpartskip
\subsection{Spectral Densities of Exciton State Coupling}
\label{subcsec:coupling}

The spectral densities of the coupling terms were calculated with the same method
as the state and site transition energies, and are shown in figure \ref{fig:specdens_coupling}.
It can be seen that there are far fewer features in the coupling spectral density
than in the site and state transition spectra. Additionally there is a broad feature
at the low frequency end of the spectrum, around 0.01 eV. Intuitively, the magnitude
of the spectral density decreases as the separation of chlorophylls increase, with
anything but nearest neighbours showing significant value against the strongest coupling 
spectrum.

\begin{figure}
    \centering
    \includegraphics[scale=0.8, angle=90, origin=c]{../../Year_3/SpectralDensity/images/specdens_coupling.png}
    \caption{Spectral density of exciton state coupling values (i.e. off diagonal
    elements of the exciton Hamiltonian) of LH2 coloured by average distance between 
    chlorophyll sites.}
    \label{fig:specdens_coupling}
\end{figure}


Taking the largest valued spectrum, where the distance between the two chlorophylls
was the smallest with an average of 15.9 \AA{}, the high frequency features correspond 
to features found in the exciton state and site spectra. The strongest features are 
found at 0.201, 0.203 and 0.205 eV, with strengths of 0.416, 0.457 and 0.372 meV 
respectively. These correspond with major features in the site spectra, implying 
that these are due to intra-chlorophyll variations rather than any change in the
protein scaffold.

Previous arguments about coupling terms have said that the change in distance is
the controlling factor. The effect of distance was also investigated by calculating
the spectral density of the inter-chromophore distance (reported without units as 
these are not physically meaningful). These spectra can be seen in \ref{fig:specdens_distance}. 

\begin{figure}
    \centering
    \includegraphics[scale=0.8, angle=90, origin=c]{../../Year_3/SpectralDensity/images/specdens_distance.png}
    \caption{Spectral density of the separation between nearest neighbour sites
    in LH2.}
    \label{fig:specdens_distance}
\end{figure}

It can be seen that while there is some correspondence to the exciton coupling spectra 
in the low frequency region, there is little correspondence to features in the high
frequency region. The strongest features at 0.169, 0.175, 0.180 and 0.182 eV do
not correspond to frequencies present in other spectra. The largest feature in
this distance spectral density is at 0.003 eV, which is at a similar frequency to a
low frequency peak in the coupling spectra. However, these peaks are qualitatively 
different, with the exciton coupling peak being much broader. This low frequency
peak in the exciton coupling spectra is relatively weak compared to the intra-chlorophyll
motion coupling peaks.

The lack of peaks in the coupling spectra from 0.03 eV to 0.168 eV also supports
the argument that intra-chromophore variations, and not the protein structure movements,
are the determining factor in exciton state transition energy variations. Overall,
it can be seen that whilst some exciton coupling variations may be due to changes 
in the protein scaffold moving chlorophylls, generally this has little affect on
the exciton state spectral densities. The similarity between major features in the
coupling spectral density and site and state spectra imply again that almost all
variation originates at chlorophyll sites.

\afterpartskip
\section{Assigning Specific Motions}
\label{sec:monomer_dimer_assign}

The observation that variations in exciton state energies are primarily due to intra-chromophore 
variations opens the question about which motions are causing these variations. Discussion
of these spectra so far has been limited due to the lack of assignment of specific
motions that are coupling to the environment. This section reports on some of the
tests used to assign features in the spectral densities. These tests include comparing
the LH2 spectral densities to a spectral density of monomer chlorophyll embedded
in diethyl ether, as well as a spectrum constructed from Huang-Rhys factors calculated
with chl-xTB response properties and a GFN1-xTB hessian. Some speculative explanations
of the lack of low frequency features are also given.

\subsection{Spectral Density of the \Qy Transition from Chlorophyll in Diethyl Ether}
\label{subsec:specdens_ether}

So far only the environmental coupling of LH2 has been considered. Whilst some differences
between the ring sites have been observed, it is not clear how other environments 
couple to the \Qy transition. A candidate environment of chlorophyll in diethyl 
ether was investigated, to match the previous chapters. The conclusion from comparing
the spectral density of chlorophyll in diethyl ether would lie between two extremes
- either there is no variation in the spectral density, or major variation. The first
possibility would support the argument that the LH2 protein does not do anything 
discernable to chlorophyll spectral densities, whereas the second would imply that
the protein environment does need to be considered more carefully.

The time series of chlorophyll geometries was taken from an MD simulation of a diethyl 
ether embedded chlorophyll. This simulation was constructed using a solvent box 
made with the \code{packmol} program with a single chlorophyll  molecule in a 64 
$\AA{}$ box with 1054 diethyl ether molecules. The simulation was run with OpenMM 
using parameters for chlorophyll taken from the LH2 force-field, and parameters 
for diethyl ether taken from the General Amber ForceField (GAFF). The system was
equilibrated for 60 ps, with a production workflow of 300 ps run. Structures were 
taken every 2 fs. A Langevin integrator set at 300 K was used with a time step of 2 fs.

The \Qy transition was calculated for every frame of the MD trajectory. The spectral
density of these transition energies was calculated with the same method as the 
previous spectra. The spectral density can be seen in figure \ref{fig:specdens_ether}.

\begin{figure}
    \centering
    \includegraphics[scale=0.8, angle=90, origin=c]{../../Year_3/SpectralDensity/images/specdens_ether.png}
    \caption{Spectral density of the \Qy transition for a chlorophyll in diethyl-ether.}
    \label{fig:specdens_ether}
\end{figure}

The major peaks are found at similar frequencies to the site and state spectra, albeit
with different magnitudes. The strongest feature in the diethyl-ether spectrum is
at 0.21 eV, with a magnitude of 0.241 eV. The next strongest feature is at 0.165 
eV, the frequency of the strongest peaks in the site and state spectra, but at a
magnitude of 0.232 eV. The collection of features at 0.186 eV, 0.191 eV, 0.195 eV
and 0.201 eV are also present with magnitudes of 0.15 eV, 0.069 eV, 0.08 eV and 
0.103 eV.

It can be seen that the features in the spectral density for diethyl ether embedded
chlorophyll are similar to the features from the LH2 chlorophyll and exciton transitions.
This similarity implies that the coupling of environment to energy fluctuations 
is due to mostly intrinsic properties of the chlorophyll geometry. There is still
some variation in the magnitudes of the environmental coupling, however the lack 
of any features at different frequencies or major changes in coupling magnitudes
imply little effect originating from the environment. Overall couplings between
chlorophyll transition energies and the environment are controlled by intrinsic 
chlorophyll properties.

\subsection{Assignment of Molecular Motion with Huang-Rhys Factors}
\label{subsec:hrf}

These intrinsic properties that cause features in the spectral density would most
likely be the coupling of internal vibrational motions of chlorophyll to the \Qy transition. 
A larger coupling of vibrational motion would imply greater variation in the \Qy 
transition energy, which would increase the magnitude of features in the spectral
density. The coupling of normal modes to electron transitions can be calculated 
with Huang-Rhys factors, defined by the difference between minima in the excited 
and ground state energy surfaces along normal mode coordinates. By comparing the 
Huang-Rhys factors of all normal modes of a chlorophyll molecule, it would be possible 
to identify which internal motions are responsible for spectral density features.

The normal modes used to calculate Huang-Rhys factors were calculated with a hessian
calculation on an optimised single chlorophyll structure. It was found that rotations
in the phytol tail caused issues in converging to an optimised geometry, attributed
to the low energy barrier for C-C bond rotation. The phytol tail was removed to
overcome this issue, with a hydrogen atom replacing the phytol group. The optimised
geometry was then used to calculate normal modes with GFN1-xTB.

A scan of excitation energies was calculated for each normal mode, with the chlorophyll
atoms being displaced along the vectors derived from the hessian of the optimised
chlorophyll structure. The coordinate of the scan was defined as 

\begin{equation}
    q_i = \sqrt{\frac{\omega_i}{\hbar}} x^m_i
\end{equation}
%
where $\omega_i$ is the angular frequency of the normal mode $i$ and $x^m_i$ is
the displacement vector in mass weighted coordinates. The \Qy transition energy 
was calculated for a series of structures with successive values of $q_i$. Fits 
of the ground state and excited state energies were made with quadratic functions,
from which it was possible to make estimates of the $q$ value for a minimum ground
state energy ($q_{\text{ground}}$) and excited state energy ($q_{\text{excited}}$).

From these $q$ values it was possible to calculate the Huang-Rhys factors as

\begin{equation}
    d = \frac{\left(q_{\text{excited}} - q_{\text{excited}}\right)^2}{2}
\end{equation}
%
These Huang-Rhys factors were then used to construct a spectral density, using
their absolute value for amplitude and the frequency of the corresponding normal
mode as position in the frequency domain. A plot of this spectrum is shown in figure 
\ref{fig:specdens_hrf}.

\begin{figure}
    \centering
    \includegraphics[scale=0.8, angle=90, origin=c]{../../Year_3/SpectralDensity/images/hrf_spectrum.png}
    \caption{A simulated spectral density of the \Qy transition for chlorophyll 
    constructed from Huang-Rhys factors and the frequency of normal modes. Labeled
    peaks are chosen as modes that may correspond to features in other chlorophyll
    spectral densities.}
    \label{fig:specdens_hrf}
\end{figure}

The low frequency (<0.05 eV) modes in this spectrum are clearly suppressed in LH2
and diethyl-ether environments. As these motions correspond to large scale deformations 
of the porphyrin ring, the high force constants might make these motions unobservable
rather than any effects from the environment. It is fairly clear that these motions
should not correspond to any features in the spectral densities.

The high frequency normal modes' correspondence to spectral density features is 
less clear. Whilst some peaks in the Huang-Rhys spectrum are similar to the full 
spectral density, the overall change in frequency positions make comparison difficult. 
This change could be due to the completely different vibrational modes present, 
however it is more likely that whole sections of normal modes have been shifted 
from the frequencies observed in the spectral densities, due to the differences 
in the force-field and GFN1-xTB method. The motions themselves should be similar,
even if the frequencies are not. Assuming a similarity in motion implies that the
major features in the Huang-Rhys spectrum should correspond with major features 
in other spectral densities. These modes have been labeled in figure \ref{fig:specdens_hrf}.
All of these modes have significant movement in the $N_A$, $N_B$, $N_C$ and $N_D$ 
atoms, corresponding with a symmetry breaking along the \Qy dipole axis. These 
symmetry breaking modes are shown in figure \ref{fig:hrf_vibrations}.

\begin{figure}
    \centering
    \includegraphics[scale=0.7]{../../Year_3/SpectralDensity/images/all_vibrations.png}
    \caption{Motions of the four central nitrogen atoms in bacterial chlorophyll
    for the vibrational modes labelled in figure \ref{fig:specdens_hrf}.}
    \label{fig:hrf_vibrations}
\end{figure}

In total the Huang-Rhys factors do not offer a clean explanation of spectral density
features. It could be that using normal modes calculated with the same forcefield
method might improve the comparison, but this was not possible with the resources
available. However looking at the atomic motions more explicitly could explain some
of the features observed.

\subsection{Assignment of Molecular Motion with Spectral Densities of $N$ Axes Deformation}
\label{subsec:N_axes_deformation}

The motion of the $N_A$-$N_C$ and $N_B$-$N_D$ axes inducing a $D_{4h}$-$C_{s}$ symmetry
break in the vibrational modes with the highest Huang-Rhys factors could suggest
that many of the major spectral density features are due to this motion. As it is
possible to construct a time series of a metric to describe this deformation, it
would be possible to compare the spectral density of these motions to the transition
energy spectral densities. The chosen metric was the ratio of $N_A$-$N_C$ and $N_B$-$N_D$
length, and the spectral density of this property is shown in figure \ref{fig:specdens_N_axes},
again without amplitude units as these would be physically meaningless. In order
to account for all chlorophyll sites, the spectral density was calculated for each
site and then averaged.

\begin{figure}
    \centering
    \includegraphics[scale=0.8, angle=90, origin=c]{../../Year_3/SpectralDensity/images/specdens_N_axes.png}
    \caption{Spectral density of the ratio $\frac{\left\lvert N_A\text{-}N_C \right\rvert}{\left\lvert N_B\text{-}N_D \right\rvert}$,
    averaged over chlorophyll sites in LH2.}
    \label{fig:specdens_N_axes}
\end{figure}

Whilst the major site/state feature at 0.165 eV is also present in this N axes spectrum,
there is little correspondence in other peaks. The majority of peaks are in the 
0.022-0.121 eV range, which is not populated in the transition energy spectra, and
the only two corresponding peaks are at 0.165 eV and 0.205 eV, with the latter only
present in B800 sites. Overall this metric offers little in explanation of which
motions are responsible for transition energy spectral density features, with the
exception of the major 0.165 eV peak. Whilst it is a little surprising that most 
features must be due to other vibrational motions of chlorophyll, it is encouraging 
that the major feature corresponds to a $D_{4h}$-$C_{s}$ symmetry breaking motion. 

Full assignment of the spectral density features is outside the scope of this work,
but it can be seen that there are two possible options for further investigation.
One would be producing hessians from force-field parameters, which would make a more
compelling comparison of Huang-Rhys factors and spectral density features. The second
would be a more thorough look at all atomic motions, creating a large series of 
spectral densities of geometry metrics. The issue with this investigation would 
still be the lack of correspondence between spectral density features and normal
mode vibrations, however it may shed light on which chlorophyll atoms are particularly
important to spectral density features.

\section{Conclusions}
\label{sec:specdens_concs}

The work in this chapter shows that the chl-xTB exciton method can fulfill the criteria
set out in the introduction when describing the scope of this project. Explicitly 
calculating the large number of exciton states necessary for the spectral density 
was achieved  due to the efficiency of using a semi-empirical method. Whilst other 
methods could have been used to achieve this, the accuracy against TD-DFT level 
data was a necessary factor in order to be able to make reliable conclusions about 
response property variance. The conclusions on exciton state spectral densities 
are based upon the good agreement between the LH2 site spectral densities and other 
reported works.

The exciton transition energy and coupling energy spectra clearly show that thermal
environment coupling is mostly based around intra-chlorophyll variations. The coupling
spectra show how low-frequency motions of the protein are a relatively unimportant 
factor in exciton state variation, and that coupling variations originate at the
monomer level than aggregate effects. This extends the observation made in previous
work that the Qy excitation energy of individual chlorophyll sites remains essentially 
constant when intramolecular BChl vibrations are “frozen out” (i.e., the BChl geometry
is artificially fixed during a molecular dynamics simulation) to leave only the 
effect of large-scale protein motion \cite{Claridge2018a}.

Whilst these intra-chlorophyll variations are explored in some detail, it remains
a challenge to fully assign the underlying motions for peaks at specific frequencies.
A known issue with full assignment is also the mismatch between the quantum mechanical
methods used for excited states and the classical force-field methods used for MD \cite{Zuehlsdorff2022}.
The similarity of spectra from an LH2 environment and diethyl-ether environment 
strongly imply that these motions are intrinsic to chlorophyll, and that the environment
only subtly changes the magnitudes of couplings and not the frequencies. The study
into correlation between vibrational motions with significant Huang-Rhys factors 
and spectral density features gave some insight into these motions, but due to the
differences in the hessian and force-field method this is not conclusive. Additionally 
using geometry metrics such as the N axes deformation explain some features but not
all. Studying more of these metrics, especially for atoms where the transition density
is centred, may explain more features.

One obvious way to extend this study would be to calculate exciton states for a 
longer simulation. This would not affect any of the conclusions about the environmental
coupling to exciton states in the high frequency region, but would test the absence
of any low frequency features. Similar to the discussion above, the lack of protein 
motions may be due to the initial low energy crystal structure or a high energy 
barrier, but it could also be that these motions are slower than could be captured 
by the used resolution. Increasing what would count as a reasonable computation 
time would be necessary, for both the exciton states and calculating the autocorrelation 
and fourier transforms, however the workflow would not change.

Using chl-xTB to calculate spectral densities is not necessarily limited to LH2.
For example spectra of the FMO light harvesting complex could be generated with
little alteration to the workflow used here. However when looking at other systems 
some consideration may have to be given to the chl-xTB training data, making sure
to include other conformations or chlorophyll molecule types. Considering the spectral
densities from more systems would contribute to the conclusions about feature origins
discussed here.