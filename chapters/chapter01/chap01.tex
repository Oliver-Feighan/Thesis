%
% File: chap01.tex
% Author: Oliver J. H. Feighan
% Description: Introduction chapter
%
\let\textcircled=\pgftextcircled
\chapter{Introduction}
\label{chap:intro}

\initial {N}aturally occuring light harvesting systems present an interesting 
scientific challenge. With near perfect efficiency, the energy from a photon
will be taken and transfered to a reaction centre, leading to charge transfer 
processes that culminate in powering biological systems. Making models that 
can predict and explain these effects are key to making similarly efficient 
photovoltaic systems.

\section{Electronic structure}
\label{sec:electronic_structure}

\section{Light-Matter Response}
\label{sec:response_theories}

\subsection{Linear Response}
\label{subsec:tddft}

\subsection{$\Delta$-SCF}
\label{subsec{dscf_and_eigdiff}}

\dscf predicts the transition energy $\Delta E$ of a system as the difference of
the single point energy $E_n$ of two states:

\begin{equation}
\Delta E = E_{2} - E_{1}
\end{equation}
%
It is usually assumed that the excited state solution will be in a similar
location to the ground state in the MO coefficient space. The ground state MO 
coefficients are usually used for an initial guess for the excited
state for this reason. In its simplest form, the \dscf method calculates
the ground state with normal DFT or other mean-field methods, and
then calculates the excited state by rerunning the same method with the excited
state occupation numbers. The two sets of MO coefficients give a full description
of both the ground as excited state.

The issue with finding the excited state solution is that the variation principle
and SCF iterative procedure will try to find the global minimum, which is the 
ground state. The excited state is a local minimum, and so often is less reliable
to find as a solution, especially from the standard SAD initial guess.
For this reason it is often found that converging to the \dscf excited state will
fail. Even when using the ground state as an initial guess with excited state
occupations, normal SCF procedure may still collapse back to the ground state.
Usually it is necessary to include additional changes to the SCF procedure,
such as Fock damping, alternative DIIS methods and sometimes intermediate 
initial guess steps.

Initially, the excited state was calculated by relaxing the orbitals which
contain the excited electron and hole in the ground state space, so that the
excited state and ground state are orthogonal \cite{Hunt1969}. However, it was
argued that this procedure would exacerbate the likelihood of collapsing to the ground
state, and that the excited state was not a proper SCF solution \cite{Gilbert2008}.
Alternatively, an SCF like method was proposed, where instead of
populating orbitals according to the Aufbau principle, orbitals which most
resemble the previous iteration's orbitals should be occupied. This is known as 
the maximum overlap method (MOM). In the maximum overlap method, each iteration 
in an SCF procedure produces new molecular orbital coefficients by solving the 
Roothaan-Hall equations \cite{Roothaan1951}, generally given as an eigenvalue problem:

\begin{equation}
\mathbf{F} \mathbf{C}^{\text{n}} = \mathbf{S} \mathbf{C}^{\text{n}} \epsilon
\label{eq:roothaan_hall}
\end{equation}
%
where $\mathbf{C}^{\text{n}}$ are the $n^{\text{th}}$ orbital coefficient solutions, 
$\mathbf{S}$ is the overlap of orbtials, and $\epsilon$ are the orbital energies. 
The Fock matrix $\mathbf{F}$ is calculated from the previous set of orbital 
coefficients,

\begin{equation}
\mathbf{F} = f\left(\mathbf{C}^{n-1}\right)
\end{equation}
%
. The amount of similarity of orbitals can be estimated from their overlap,

\begin{equation}
\mathbf{O} = \left(\mathbf{C}^{\text{old}}\right)^\dagger \mathbf{S} \mathbf{C}^{\text{new}}
\end{equation}
%
and for a single orbital can be evaluated as a projection,

\begin{equation}
p_j = \sum^n_i O_{ij} = \sum^N_\nu \left[\sum^N_\mu\left(\sum^n_i C_{i\mu}^{\text{old}}\right)S_{\mu\nu}\right]C^{\text{new}}_{\nu j}
\end{equation}
%
where $\mu,\nu$ are orbital indices. the set of orbitals with the highest projection
$p_j$ are then populated with electrons.  This method can be used for any
excited state, with the caveat that the orbital solution will most likely be in
the same region as the ground state solution. For a small number of low lying states,
this is generally  true, and so \dscf can be used to calculate a small spectrum of
excited states \cite{Gilbert2008}.

\dscf has been shown to be cheap alternative to TD-DFT and other higher level
methods \cite{Liu2004, Gavnholt2008, Besley2009}, without considerable losses of
accuracy in certain cases, especially for HOMO-LUMO transitions \cite{Kowalczyk2011}.
Additionally, as the excited state is given as solutions to SCF equations,
the gradient of this solution can be given by normal mean-field theory.
These gradients would be much cheaper than TD-DFT or coupled cluster methods, 
which is advantageous for simulatings dynamics \cite{Gavnholt2008}.

\subsection{Eigenvalue Difference}
\label{subsec:eigval_diff}
Another approximation to full response theory is the eigenvalue difference method. 
Here there is assumed to be no response of the orbital energies and shapes when 
interacting with light. This would be recovered from the complete Cassida equation
if the coupling elements in the $\mathbf{A}$ and $\mathbf{B}$ matrices were set to zero.
Within this approximation, the transition energy is just the difference between 
the ground state energy of the orbital an electron has been excited to($\epsilon_{\text{e}}$)
and the orbital has been excited from ($\epsilon_{\text{g}}$),

\begin{equation}
\Delta E = \epsilon_{\text{e}} - \epsilon_{\text{g}}
\end{equation}
%
. Additionally, transition properties can be calculated by constructing transition 
density matrices from the ground state orbitals such that needing only a single 
SCF optimization is required. Generally, eigenvalue difference methods are not 
seen as accurate response methods, but can offer a quick and easy initial value 
\cite{Gimon2009}.

\subsection{Transition Density and Dipole Moments}
\label{subsec:dscf_transition_density}
\dscf transition properties, such as the transition dipole moment, can be calculated from
the SCF solutions for the ground and excited states. The reduced one-particle transition
density matrix $\mathbf{D}^{21}$ can be written as

\begin{equation}
\mathbf{D}^{21} = \ket{\Psi_1} \bra{\Psi_2}
\end{equation}
%
where $\ket{\Psi_n}$ is the Slater determinant of state $n$, constructed from the
set of spin orbitals $\{ \phi_{j}^{\left(n\right)} \} $. Expressed 
in terms of the molecular orbitals coefficients $\mathbf{C}^{\left(n\right)}$, the
transition density matrix is

\begin{equation}
\mathbf{D}^{21} = \mathbf{C}^{\left(2\right)} \text{adj}\left(\mathbf{S}^{21}\right) \mathbf{C}^{\left(1\right) \dagger}
\end{equation}
%
where $\mathbf{S}^{21}$ is an overlap matrix with elements 

\begin{equation}
S^{21}_{jk} = \braket{\phi^2_j|\phi^1_k}
\end{equation}
%
. The depednece on the adjunct of the overlap
can be understood using L{\"o}wdin's normal rules for non-orthogonal determinants \cite{Lowdin1955}.
In the same way, the transition dipole moment is given by

\begin{equation}
\braket{\Psi_2|\hat{\mathbf{\mu}}|\Psi_1} = \sum_{jk} \mathbf{\mu}_{jk}^{21} \text{adj} \left( \mathbf{S}^{21}\right)_{jk}
\end{equation}
%
where $\hat{\mathbf{\mu}}$ is the one-electron transition dipole operator, and
$\mu_{jk}$ is the element of this operator corresponding to orbital indices $j$, $k$.
The determinant of $\mathbf{S^{21}}$ can be defined as the inner product of the 
two states involved in the transition

\begin{equation}
\left\lvert {\mathbf{S}^{21}} \right\rvert = \braket{\Psi_2|\Psi_1}
\end{equation}

The general definition of the transition dipole

\begin{equation}
\mathbf{\mu}^{1\rightarrow2} = \braket{\Psi_2|\hat{\mathbf{\mu}}|\Psi_1}
\end{equation}

can be expressed with this transition density matrix as:

\begin{equation}
\begin{split}    
\braket{\Psi_2|\hat{\mathbf{\mu}}|\Psi_1} &= \text{tr}\left(\hat{\mathbf{\mu}} \ket{\Psi_1} \bra{\Psi_2} \right) \\
&= \text{tr}\left( \hat{\mathbf{\mu}} \mathbf{D}^{21}\right)
\end{split}
\end{equation}


\section{Scaling Theory For Large Systems}
\label{sec:large_systems_theory}


