%
% File: chap01.tex
% Author: Oliver J. H. Feighan
% Description: Introduction chapter
%
\let\textcircled=\pgftextcircled
\chapter{Background Theory}
\label{chap:background_theory}

\initial {T}here are three theory components that are frequently employed when modeling
light harvesting complexes. First is the electronic structure theory, which is almost
exclusively density functional theory (DFT) or density functional tight binding (DFTB). 
Second is the response method, for which the workhorse is often time-dependent DFT (TD-DFT) 
or TD-DFTB. In this work other response methods are discussed, and so these theories
are outlined in this chapter as well. Third is the exciton method, again most frequently
a Frenkel exciton Hamiltonian. In order to establish a basis for the reporting and
discussion of work in later chapters, an outline of each of these three components
are given here.

\section{Electronic structure}
\label{sec:electronic_structure}

\subsection{Density Functional Theory}
\label{subsec:dft}

Density functional theory (DFT) is ubiquitous in electronic structure calculations.
Its application to a wide range of system sizes as well as chemical systems make 
it an ideal choice for many situations, including chlorophyll and light harvesting
systems. A brief overview of DFT is given here to contextualise its use in the results
chapters.

At its heart DFT is based on the two Hohenberg-Kohn theorems. The first states that
the ground state energy $E_{\text{GS}}$, is proven to have a one-to-one mapping 
to a functional of the electron density $\rho_{\text{GS}} \left(r\right)$

\begin{equation}
    E_{\text{GS}} = E \left[ \rho_{\text{GS}} \left(r\right)\right]
\end{equation}
%
where $E \left[ \rho_{\text{GS}} \left(r\right)\right]$ is the functional. The second
theorem is closely related to the variation principle, stating that the \emph{exact}
ground state density also minimises the total energy. This minima corresponds to 
only one electron density. Whilst proven in principle, the exact functional of the
electron density is unknown and so various approximations have be made. One popular
method is the Kohn-Sham approach, where non-interacting electrons are used to generate
the ground state density.  The total energy in the Kohn-Sham approach is the sum
of functionals

\begin{equation}
    \functional{E}{\text{tot}}{\rho\left(r\right)} = \functional{E}{T_S}{\rho\left(r\right)} + \functional{E}{V}{\rho\left(r\right)} + \functional{E}{J}{\rho\left(r\right)} + \functional{E}{X}{\rho\left(r\right)} + \functional{E}{C}{\rho\left(r\right)}
\end{equation}
%
where these terms correspond to the kinetic, (nuclear) potential, bare Coulombic,
exchange and correlation interaction respectively. Due to the Coulombic approximation
not including spin effects, it is necessary to include the exchange and correlation
terms. The solution for the minimum energy would satisfy

\begin{equation}
    \Delta \left[  \functional{E}{\text{tot}}{\rho\left(r\right)}  - \mu \left( \int_{}^{} \rho\left(r\right) \,dr - N \right) \right] = 0
\end{equation}
%
with the constraint that the total number of electrons $N$ is conserved. The value
of $\mu$ is given by

\begin{equation}
    \mu = \frac{\delta  \functional{E}{\text{tot}}{\rho\left(r\right)}}{\delta \rho\left(r\right)}
\end{equation}
%
which can be rewritten in terms of the kinetic energy and energy potentials

\begin{equation}
    \begin{split}
        \mu &= \frac{\delta  \functional{E}{T_S}{\rho}}{\delta \rho\left(r\right)} + \functional{v}{V}{\rho\left(r\right)} + \functional{v}{J}{\rho\left(r\right)} + \functional{v}{X}{\rho\left(r\right)} + \functional{v}{C}{\rho\left(r\right)} \\
            &= \frac{\delta  \functional{E}{T_S}{\rho}}{\delta \rho\left(r\right)} + \functional{v}{KS}{\rho\left(r\right)}
    \end{split}
\end{equation}
%
where the energy potentials are combined into the Kohn-Sham potential $\functional{v}{KS}{\rho\left(r\right)}$ for
convenience. As the electrons are non-interacting, this potential can be used to
solve one-electron Schrödinger equations

\begin{equation}
    \left[ -\frac{1}{2} \nabla^2 + \functional{v}{KS}{\rho\left(r\right)} \right] \psi_i = \epsilon_i \psi_i
\end{equation}
%
where $psi_i$ are one-electron wavefunctions, $\epsilon_i$ are Lagrange multipliers
to ensure orthonormality, and the kinetic energy term $-\frac{1}{2} \nabla ^2$ is
given from the definition of the kinetic energy of non-interacting electrons

\begin{equation}
    \functional{E}{T_S}{\rho\left(r\right)} = -\frac{1}{2} \sum_i^N \braket{\psi_i|\nabla^2|\psi_i}
\end{equation}
%
The total electron density can be constructed from the one-electron wavefunction
solutions

\begin{equation}
    \rho = \sum_i^N \left\lvert \psi_i \right( r \left)^2 \right\rvert 
\end{equation}
%
Again the issue is that the potential functional $\functional{v}{KS}{\rho\left(r\right)}$ is
not known, and so approximations have to be made. Additionally, an initial guess 
of the electron density in needed to start the variational procedure, but this can
readily be given from atomic densities or other methods.

In practice, a range of exchange-correlation functionals are employed, dependant
on the problem at hand. The simplest functionals are employ the local density approximation
(LDA), which assumes the electron density is the same as the uniform electron gas
for all points. More complicated functionals use the density and density gradient,
using the generalised gradient approximation (GGA). Meta-GGAs also use the second
derivative. Hybrid functionals also use some fraction of the Hartree-Fock energy.
Many of exchange-correlation functionals require parameterising against high level
data.

The first DFT calculations were done on periodic systems, where infinite-domain
plane wave functions where used as a basis set, however now most electronic structure
packages use basis functions centred on atomic positions (although this is not always
the case). The one-electron wavefunctions can be written as linear combinations 
of basis functions $\chi_j$

\begin{equation}
    \psi_i = \sum_\mu^n c_{\mu i} \chi_\mu \left( r\right)
\end{equation}
%
where $j$ is the index of the basis function, going up to the total number of functions
$n$, and $c_{\mu i}$ is the molecular orbital (MO) coefficient of the basis function
$\chi_\mu$ (also referred to as an atomic orbital or AO) for orbital (one-electron 
wavefunction) $\psi_i$. Rewritting the electron density gives the density matrix 
$\mathbf{D}$

\begin{equation}
    \rho = \sum_{\mu\nu} P_{\mu\nu} \chi_\mu \chi_\nu
\end{equation}
%
where the elements of the density matrix are the product of MO coefficients

\begin{equation}
    P_{\mu\nu} = \sum_i^N c_{i \mu} c_{i \nu}
\end{equation}
%
Similarly, the Kohn-Sham potential can be written in matrix form 

\begin{equation}
    F_{\mu \nu} = \frac{\delta E_{KS}}{\delta P_{\mu\nu}}
\end{equation}
%
where $E_{KS}$ is the energy from the potential $v_{KS}$. Applying the variational
principle gives the matrix equation

\begin{equation}
    \mathbf{F} \mathbf{C} = \mathbf{S} \mathbf{C} \mathbf{\epsilon}
\end{equation}
%
where $\mathbf{S}$ is the matrix of overlap matrix and $\mathbf{\epsilon}$ is the
diagonal matrix of one-electron wavefunction energies. This equation is similar 
to the Roothaan-Hall equations used to solve Hartree-Fock theory, and so the matrix
$\mathbf{F}$ is commonly referred to as the Fock matrix.

Having a convenient definition of the ground state electron structure is useful
in calculating other properties. As explained in section \ref{subsec:tddft}, the
ground state MO coefficients can be used to calculate vertical excitation energies
as well as excited state and transition electron densities. These properties are
necessary to understand the photochemical processes in light harvesting complexes.

\subsection{Density Functional Tight Binding}
\label{subsec:tight_binding}
In recent years there has been renewed interest in tight-binding methods, including
tight binding methods derived from density functional theory (named density function 
tight binding or DFTB)\cite{Porezag1994}. These methods approximate the density 
functional energy by expanding into a Taylor series, based on the density fluctuations
$\delta\rho$\cite{Koskinen2009}

\begin{equation}
    \begin{split}
    E\left[\rho\right] &= E^{\left(0\right)}\left[\rho_0\right] + E^{\left(1\right)}\left[\rho_0, \delta\rho\right] + E^{\left(2\right)}\left[\rho_0, \left(\delta\rho\right)^2\right] + E^{\left(3\right)}\left[\rho_0, \left(\delta\rho\right)^3\right] + \dots \\
    &= \functional{E}{}{\rho_0\left(r\right)} \\
    &+ \left. \int \frac{\delta \functional{E}{}{\rho\left(r\right)}}{\delta \rho\left(r\right)} \right\rvert_{\rho_0}  \delta \rho \left( r \right) \\
    &+ \frac{1}{2} \left. \int\int \frac{\delta^2 \functional{E}{}{\rho\left(r\right)}}{\delta \rho\left(r\right)\delta \rho\left(r^\prime\right)} \right\rvert_{\rho_0} \delta \rho \left( r \right) \delta \rho \left( r^\prime \right) + \cdots \\
    &+ \frac{1}{p!} \left. \int\int \cdots\int \frac{\delta^p \functional{E}{}{\rho\left(r\right)}}{\delta \rho\left(r\right)\delta \rho\left(r^\prime\right) \cdots \delta \rho\left(r^{\left(p\right)}\right)} \right\rvert_{\rho_0} \delta \rho \left( r \right) \delta \rho \left( r^\prime \right) \cdots \delta \rho\left(r^{\left(p\right)}\right) + \cdots
    \end{split}
\end{equation}
%
with this series usually truncated between the first and third term\cite{Gaus2011}. 
These terms are analogous to some of the terms in the Kohn-Sham functional. The 
first term is called the band-structure energy - as it contains no density fluctuations
is given by the sum of energies of single particle wavefunction

\begin{equation}
    \functional{E}{}{\rho_0\left(r\right)} =  \sum_i \braket{\psi_i|\functional{H}{}{\rho_0}|\psi_i}
\end{equation}
%
where the Hamiltonian $H$ contains the kinetic energy and electron-nuclear potential
(often called the core Hamiltonian). The second order term corresponds to the Coulomb 
and exchange-correlation terms

\begin{equation}
    \frac{1}{2} \left. \int\int \frac{\delta^2 \functional{E}{}{\rho\left(r\right)}}{\delta \rho\left(r\right)\delta \rho\left(r^\prime\right)} \right\rvert {\rho_0} \delta \rho \left( r \right) \delta \rho \left( r^\prime \right) = \frac{1}{2} \left. \int\int \frac{\delta^2 \functional{E}{XC}{\rho\left(r\right)}}{\delta \rho\left(r\right)\delta \rho\left(r^\prime\right)} + \frac{1}{\left\lvert r-r^\prime \right\rvert} \right\rvert_{\rho_0} \delta \rho \left( r \right) \delta \rho \left( r^\prime \right)
\end{equation}
%
with the other Taylor expansion terms collected into what is known as the repulsive
energy term $E^\text{Rep}$. The common expression for DFTB (also known as DFTB2
of SCC-DFTB) energy is then given as the sum

\begin{equation}
    \label{eq:dftb_energies}
    E^{\text{DFTB}} = \sum_i \braket{\psi_i|\functional{H}{}{\rho_0}|\psi_i} + E^\text{XC} + E^\text{Rep}
\end{equation}
%
however it is possible to include other terms from this framework. It is also common
to replace the continuous electron density with a point charge model. This is done
by first approximating the charge fluctuations as a sum of fluctuations centred 
on atomic positions

\begin{equation}
    \delta \rho\left(r\right) = \sum_A \delta \rho_A \left(r\right)
\end{equation}
%
where $\rho_A \left(r\right)$ is the charge fluctuation on atom $A$. These atomic
contributions are then expanded using a multipole expansion, truncated at the first
term

\begin{equation}
    \delta \rho_A \left(r\right) \approx \Delta q_A F_A^{00} \Upsilon^{00}
\end{equation}
%
where $F_A^{00}$ and $\Upsilon^{00}$ are the multipole expansion coefficients - 
$\Delta q_A$ is commonly referred to as the partial charge on atom $A$. As these
charges are not an observable of the system, but an approximation of the electronic
density, it is arbitrary which charge scheme is used. Often these are Mulliken charges 
but other methods can be used as will be seen in section \ref{subsec:stda_xtb}. 
The benefit of using a point charge approximation to electron density is that integrals
are far less expensive to calculate. The exchange-correlation term is then given
by

\begin{equation}
    E^{\text{XC}} = \frac{1}{2}\sum_{A,B} \Delta q_A \Delta q_B \gamma_{AB}
\end{equation}
%
where the function $\gamma_{AB}$ recovers the properties of electron-electron interactions.
For example at large separation this function tends towards the Coulombic $\frac{1}{R_{AB}}$
interaction. At close separations, the $\gamma$ function uses the chemical hardness 
of atoms to damping this interaction. For self-interaction (i.e. $\gamma_{AA}$), 
the energy term is equivalent to the second derivative of the energy with respect
to the atomic partial charge, equal to the Hubbard parameter or twice the chemical
hardness 

\begin{equation}
    \begin{split}
    E^{\text{XC}}_{AA} &= \frac{1}{2} \Delta q_A^2 \gamma_{AA} \\
    &= U_{A} \\
    &= 2 \eta_A
    \end{split}
\end{equation}
%
where $U_A$ and $\eta_A$ are the Hubbard parameter and chemical hardness respectively.
For inter-atomic interactions the $\gamma$ function is scaled by the Hubbard parameters
of both atoms involved.

It is also assumed that density fluctuations will only occur in the valence space
of atoms, and so core atomic orbitals do not need explicit treatment. As such, DFTB
methods usually use a minimal valence basis set\cite{Bannwarth2020}, although again 
this is not always the case. These valence atomic orbitals require orthogonalisation
against the core orbitals of other atoms, usually achieved with a Schmit orthogonalisation.
They are also often calculated from atomic DFT data, however with an additional 
harmonic constraint to prevent the orbitals from becoming too diffuse - the confiment
potential is scaled to be within twice the covalent radius of each atom.

Solving for the ground state wavefunction follows employs a similar Roothaan-Hall 
matrix equation, although with the Fock matrix defined by equation \ref{eq:dftb_energies}.
Whilst in normal DFT the convergence of the potential (or Fock matrix) between iterations
indicates that the ground state solution has been found, many DFTB schemes use the
difference in atomic charges. This scheme is called self-consistent charges (SCC),
and is common in many tight binding models.

Whilst some parameters can be calculated from DFT or \emph{ab initio} calculations,
such as the Hubbard parameters and AO coefficients, many parameters required fitting
against higher level or empirical data. For example the parameters for the repulsive
energy term are often fit to bond length data. Additionally, many elements can be
precomputed. For example the core Hamiltonian can be written in terms of the AO
contributions 

\begin{equation}
    \functional{E}{}{\rho_0\left(r\right)} = \sum_i \sum_{\mu\nu} c_{i\mu}c_{i\nu} H_{\mu\nu}
\end{equation}
%
where the elements of $H_{\mu\nu}$ are precomputed due to not being dependent on
the density fluctuations $\delta \rho$. Similarly elements of the AO overlap matrix
$S_{\mu\nu}$ can be precomputed and tabulated. For many parameters a careful approach
has to be taken to avoid systematic accuracy issues.

Much work has been done on benchmarking DFTB and expanding the formalism. Generally
the accuracy of DFTB is found to be on par with DFT, and higher order derivative 
terms can be used to achieve better results. Most often it is found that the repulsive
energy term is the trickiest to get correct, requiring the most parameterisation
against reference data. Additional energy terms, such as Non-covalent interactions 
and spin-polarisation effects can also be included. The spin-polarisation is especially
useful for excited states. Originally DFTB was formulated in a restricted ansatz,
using doubly occupied orbitals instead of spin-orbitals. This restricted its application
in a linear response framework as the excited states could not be properly treated.
However additional energy terms in the Fock matrix recover effects of using a spin-unrestricted
ansatz, making predictions of excited states much more accurate.

Tight binding methods are usually used in investigations where the scale of the 
system of interest is too large for more usual methods, such as DFT or Hartree-Fock 
(HF) based methods, to be used. Previous methods of dealing with the size of these 
systems has been to turn to force-field methods, which do not use any quantum mechanical 
methods and only use classical methods to evaluate energies and gradients of systems. 
However, it has routinely been shown that these are inaccurate for many systems 
that involve proton transfer or metallic centers or the making and breaking of chemical 
bonds\cite{Salomon-Ferrer2013}, which unfortunately covers many interesting biochemical 
systems, photosynthesis included. For these systems then, using tight-binding methods 
seems to be a good tradeoff between the expense of full DFT methods and the innacuracies 
of classical methods. However, work on making DFT programs quicker, usually with efficient 
massively parallelized codes, is closing the gap where DFTB methods exist\cite{Manathunga2020}.

\subsection{Extended Tight Binding}
\label{subsec:xtb_methods}

Recently the extended tight binding (xTB) family of methods, developed by the Grimme 
group, have been presented as another semi-empirical tight binding solution to investigating 
large chemical systems\cite{Bannwarth2020}\cite{Bannwarth2019}\cite{Grimme2017}\cite{Pracht2019}\cite{Grimme2016}\cite{Spicher2020a}.
Many of these methods have been parameterized for geometry optimizations and frequencies 
of normal modes, and use novel approaches for non-covalent interactions. These are
identified by the GFN prefix (Geometries, Frequencies and Non-covalent). These methods
require far less pair-wise parameterisation whilst remaining efficient and accurate
for large systems. These methods cover a range of detail in treating the electronic
structure, often going in hand with the number of parameters required. For example,
GFN2 uses a quadrupole expansion for electrostatics, that along with a detailed 
dispersion method makes GFN2 completely pairwise parameter free. GFN0 and GFN-FF
on the other hand use very approximate methods and more parameters to increase computation
efficiency.

The energy terms for xTB methods can be characterized by the order of density fluctuations 
they correspond to. For example, the zeroth order terms correspond to dispersion 
(either D3\cite{Grimme2010}, D4\cite{Caldeweyher2020} or a modified D4 method) and
a halogen bonding correction. First order terms are calculated with an extended 
Huckel theory, and second and higher order terms are calculated by isotropic electrostatic
and exchange-correlation terms. In the equations below, these energies are first
labelled with a superscript $\left(n\right)$ to identify the density fluctuation 
order, and a subscript to describe the interaction calculated. The second line in
each equation gives the corresponding label found in the GFN-xTB publications. The
GFN-xTB energies are then summarised as

\newcommand{\orderE}[2]{E^{\left(#1\right)}_{#2}}
\newcommand{\nameE}[2]{E^{#1}_{#2}}
\begin{equation}
\begin{aligned}
E_{\text{GFN1-xTB}} &= \orderE{0}{\text{disp}} + \orderE{0}{\text{rep}} + \orderE{0}{\text{XB}} + \orderE{1}{\text{EHT}} + \orderE{2}{\text{IES+IXC}} + \orderE{3}{\text{IES+IXC}} \\
&= \nameE{\text{D3}}{\text{disp}} + \nameE{}{\text{rep}} + \nameE{\text{GFN1}}{\text{XB}} + \nameE{}{\text{EHT}} + \nameE{}{\gamma} + \nameE{\text{GFN1}}{\Gamma}
\end{aligned}
\end{equation}
%
\begin{equation}
\begin{aligned}
E_{\text{GFN2-xTB}} &= \orderE{0,1,2}{\text{disp}} + \orderE{0}{\text{rep}} + \orderE{1}{\text{EHT}}  + \orderE{2}{\text{IES+IXC}} + \orderE{2}{\text{AES+AXC}} + \orderE{3}{\text{IES+IXC}} \\
&= \nameE{\text{D4'}}{\text{disp}} + \nameE{}{\text{rep}} + \nameE{}{\text{EHT}}  + \nameE{}{\gamma} + \nameE{}{AEC} + \nameE{}{AXC} + \nameE{\text{GFN2}}{\Gamma}
\end{aligned}
\end{equation}
%
\begin{equation}
\begin{aligned}
E_{\text{GFN0-xTB}} &= \orderE{0}{\text{disp}} + \orderE{0}{\text{rep}} + \orderE{1}{\text{EHT}}  + \Delta\orderE{0}{} \\
&= \nameE{\text{D4}}{\text{disp}} + \nameE{}{\text{rep}} + \nameE{}{\text{EHT}}  + \nameE{}{\text{EEQ}} + \nameE{}{\text{srb}}
\end{aligned}
\end{equation}
%
It can be seen that energy terms are treated with different methods even though 
they describe the same interaction. For example the dispersion interactions, which
are all zeroth order with respect to density fluctuations, employ a D3, D4 and D4'
(D4 modified) approach in GFN1-, GFN2- and GFN0-xTB respectively. The functional
form of each of these energy terms is given below.

Common to all of these expressions is the extended Huckel theory energy term ($\nameE{}{\text{EHT}}$),
derived from first order density fluctuations. This energy is given by tracing the
Hamiltonian $H_{\mu\nu}^{\text{EHT}}$ with the valence one-electron density $P_{\mu\nu}$

\begin{equation}
E_{\text{EHT}} = \sum_{\mu\nu} P_{\mu\nu} H_{\mu\nu}^{\text{EHT}}
\end{equation}
%
where $\mu$, $\nu$ are the indices of atomic orbitals. The elements of the Hamiltonian 
are given by 

\begin{equation}
H_{\mu\nu}^{\text{EHT}} = \frac{1}{2} K_{AB}^{ll'}S_{\mu\nu}\left(H_{\mu\mu} + H_{\nu\nu}\right)X\left(EN_A, EN_B\right)\Pi\left(R_{AB}, l, l'\right)Y\left(\eta^A_l\eta^B_{l'}\right)
\end{equation}
%
where $A,B$ are atomic indices, and $l,l'$ are the indices of atomic orbitals on 
atoms $A, B$ respectively (ie $l \in A, l' \in B$), $K_{AB}^{ll'}$ are parameterised global 
scaling terms, $S$ is the atomic orbital overlap, $H_{\mu\mu}$ are diagonal elements
of the Hamiltonian, treating on-site energies, $X$ is an environment-scaled electro-negativity
$EN$ function, $\Pi$ is another distance-dependent function to correct for the distance-scaled
interactions from the overlap matrix and $Y$ corrects for kinetic energy integrals
in GFN2- and GFN0-xTB but is discarded for GFN1-xTB.

Due to the lack of element pair-wise parameters (except for a few special cases 
in the global scaling constants), these terms are readily separable and so can either 
be considered or discarded for future parameterization work. For example, the $\Pi$,
$Y$ and $X$ term deal with interactions that are more described by atomic environments 
rather than particular chemistry between atoms. Hence, for excited state theories they 
would not really have to be changed. This is corroborated by their absence in the 
Hamiltonian for the simplified Tamm-Dancoff (sTDA) xTB\cite{Grimme2016} method, 
which is discussed later.

In the GFN1 and GFN2 methods there is also the $E_\gamma$ and $E_\Gamma^{\text{GFNx}}$ 
terms, which are the energy terms from second and third order density fluctuations. 
The second order term, $E_\gamma$ is common to GFN1 and GFN2, and is given by

\begin{equation}
E_\gamma = \frac{1}{2} \sum^{N}_{A,B} \sum_{l \in A} \sum_{l' \in B} q_l q_{l'} \gamma_{AB, ll'}
\end{equation}
%
where $q_l$ are shell-resolved Mulliken partial charges, $A,B$ are atom indices 
and $l,l'$ are shell indices. The $\gamma$ operator describes short-range Coulombic 
interactions

\begin{equation}
\gamma_{AB, ll'} = \frac{1}{\sqrt{R^2_{AB} + \eta^{-2}_{AB, ll'}}}
\end{equation}
%
where $R_{AB}$ is the internuclear distance between A and B, and $\eta$ is a parameterized 
chemical hardness. The third order term is slightly different for GFN1 and GFN2,
generally given as

\begin{equation}
E_\Gamma = \frac{1}{3}\sum_A^N q_A^3 \Gamma_A
\end{equation}
%
where $q_A$ is the atom partial charge (sum of the shell partial charges on that atom), 
and $\Gamma_A$ is a different operator constructed from atom-wise parameters. These
terms are analogous to the $\gamma$ operators discussed in the general DFTB theory
section. GFN2-xTB also includes higher order multipole interactions in the density-dependent
terms, referred to as anisotropic electrostatic (AES) and anisotropic exchange-correlation
(AXC) terms. The AES term is given by a sum of the monopole-dipole, monopole-quadrupole
and dipole-dipole interactions

\begin{equation}
    E_{\text{AES}} = E_{q\mu} + E_{q \Theta} + E_{\mu\mu}
\end{equation}
%
The monopole-dipole interaction, monopole-quadrupole and dipole-dipole energy terms 
are given by

\begin{equation}
    E_{q\mu} = \frac{1}{2} \sum_{AB} f_3\left(R_{AB}\right) \left[ q_A \left(\mathbf{\mu}_B^T \mathbf{R}_{BA}\right) + q_B \left(\mathbf{\mu}_B^T \mathbf{R}_{AB}\right)\right]
\end{equation}
%
\begin{equation}
    E_{q \Theta} = \frac{1}{2} \sum_{AB} f_5\left(R_{AB}\right) \left[ q_A \mathbf{R}^T_{AB} \mathbf{\Theta}_B \mathbf{R}_{AB} + q_B \mathbf{R}^T_{AB} \mathbf{\Theta}_A \mathbf{R}_{AB} \right]
\end{equation}
%
\begin{equation}
    E_{\mu\mu} = \frac{1}{2} \sum_{AB} f_5\left(R_{AB}\right) \left( \mathbf{\mu}_A^T \mathbf{\mu}_B \right) R_{AB}^2 - 3 \left( \mathbf{\mu}_A^T \mathbf{R}_{AB} \right) \left( \mathbf{\mu}_B^T \mathbf{R}_{AB} \right) 
\end{equation}
%
respectively, where  $\mathbf{\mu}_A$ is the dipole moment on atom $A$ and where 
$\mathbf{\Theta}_A$ is the quadrupole moment. The damping functions $f_n \left(R_{AB}\right)$
follow a similar shceme to the dispersion models below, although with modified parameters.

The AXC term is given by 

\begin{equation}
    E_{\text{AXC}} = \sum_A \left( f_{XC}^{\mu_A} \left\lvert \mathbf{\mu}_A\right\rvert^2  + f_{XC}^{\Theta_A} \left\lvert \left\lvert \mathbf{\Theta}_A \right\rvert\right\rvert^2\right)
\end{equation}
%
where $f_{XC}^{\mu_A}, f_{XC}^{\Theta_A}$ are element specific parameters. It can
be seen that this exchange-correlation term only accounts for changes to the electron
density around atom $A$.

Whilst the energy terms discussed so far have been derived from the Taylor expansion
of electron density, other energy corrections are necessary to account for some 
of the shortcomings of the tight-binding approximations. The repulsion energy term
common to all methods describes nuclear-nuclear interactions,
different to the $E_{\text{rep}}$ term in DFTB. This is defined as

\begin{equation}
    E_{\text{rep}} = \frac{1}{2}\sum_{A,B} \frac{Z^{\text{eff}}_A Z^{\text{eff}}_B}{R_{AB}} e^{-\sqrt{\alpha_A \alpha_B} \left(R_{AB}\right)^{k_f}}
\end{equation}
%
where $Z^{\text{eff}}_A$ is the effective nuclear charge on atom $A$, differing 
from the true nuclear charge $Z_A$ by the core atomic electron density, $k_f$ is
a global parameter and $\alpha_A$ are atom-wise parameters.

The dispersion energy terms, whilst all correcting for the charge-average schemes
introduced by the Kohn-Sham approach, vary in models. The D3 dispersion model gives
this energy as

\begin{equation}
    E^{\text{D3}}_\text{disp} = -\frac{1}{2}\sum_{A,B}\sum_{n=6,8} s_n \frac{C_n\left(CN_A, CN_B\right)}{R^n_{AB}} f^{\left(n\right)}_{\text{BJ-damping}} \left(R_{AB}\right)
\end{equation}
%
where $C_n$ is the dispersion coefficients for dipole-dipole ($n=6$) and dipole-quadrupole
(n=8) interactions, which are functions of the coordination number $CN_A$, $s_n$
is a scaling factor, and the Becke-Johnson damping function is given by

\begin{equation}
    f^{\left(n\right)}_{\text{BJ-damping}}\left(R_{AB}\right) = \frac{R^n_{AB}}{R^n_{AB} + \left(a_1 \sqrt{\frac{C_8^{AB}}{C_6^{AB}}} + a_2 \right)^n} 
\end{equation}
%
where $a_1$, $a_2$ are also global parameters. The D4 model of dispersion differs
by including a charge dependency in the dispersion coefficient function, giving

\begin{equation}
    E^{\text{D4}}_\text{disp} = -\frac{1}{2}\sum_{A,B}\sum_{n=6,8} s_n \frac{C_n\left(q_A, CN_A, q_B, CN_B\right)}{R^n_{AB}} f^{\left(n\right)}_{\text{BJ-damping}} \left(R_{AB}\right)
\end{equation}
%
(here the charges calculated using the EEQ scheme, discussed below). For GFN2-xTB
the D4 model is modified to include a 3-body term, giving

\begin{equation}
    \begin{split}
    &E^{\text{D4}^{\prime}}_\text{disp} = -\frac{1}{2}\sum_{A,B}\sum_{n=6,8} s_n \frac{C_n\left(CN_A, CN_B\right)}{R^n_{AB}} f^{\left(n\right)}_{\text{BJ-damping}} \left(R_{AB}\right) \\
    & + s_9 \sum_{A,B,C} \frac{\left(3\cos\left(\theta_{ABC}\right)\cos\left(\theta_{BCA}\right)\cos\left(\theta_{CAB}\right)+1\right) C_9 \left(CN_A, CN_B, CN_C\right)}{\left(R_{AB}R_{AC}R_{BC}\right)^3} \\
    & \times f^{\left(n\right)}_{\text{damping}} \left(R_{AB}, R_{AC}, R_{BC}\right)
    \end{split}
\end{equation}
%
where $f^{\left(n\right)}_{\text{damping}} \left(R_{AB}, R_{AC}, R_{BC}\right)$ is
a special damping function for the three-body term. It can be seen that again GFN2-xTB 
includes higher order terms to achieve greater accuracy against the test data.

Present also in GFN1- and GFN0-, but not GFN2-xTB, are correctional terms that are
independent with respect to the electron density. For GFN1-xTB this is the halogen
bonding energy term, which for a system of halogen bond acceptor $A$, donor $B$ and
halogen $X$ is given by

\begin{equation}
    E_{XB}^{\text{GFN}1} \sum^{N_{XB}}_{AXB} f_{ABX\text{-damping}} k_X \left[ \left( \frac{k_{XR} R_{\text{cov}, AX}}{R_{AX}}\right)^{12} - k_{X2} \left(\frac{k_{XR} R_{\text{cov}, AX}}{R_{AX}}\right)^6 \right] \left[ \left( \frac{k_{XR} R_{\text{cov}, AX}}{R_{AX}}\right)^{12} + 1\right]^{-1}
\end{equation}
%
where $k_{X2}, k_{XR}$ are global parameters but $k_X$ is a halogen-specific parameter.
The damping term is given as

\begin{equation}
    f_{AXB\text{-damping}} = \frac{1}{2} \left(1 - \frac{1}{2} \frac{R^2_{XA} + R^2_{XB} - R^2_{AB}}{\left\lvert R_{XA} \right\rvert \left\lvert R_{XB} \right\rvert} \right) ^6
\end{equation}
%
Similar to DFT and DFTB, these energy terms (or more accurately their potentials)
are used to construct the Fock matrix that is solved self-consistently in the Roothan-Hall
equations to give ground state MO coefficients and energies. However this is only
true for GFN1- and GFN2-xTB, as these use Mulliken schemes for the partial charges.
The GFN0-xTB uses a non-self consistent charge scheme referred to as electronegativity
equilibration (EEQ), which means that only one diagonalisation of the Fock matrix
is required for a ground state solution. These charges are given by solving the 
matrix equation

\begin{equation}
    \label{eq:eeq_solve}
    \begin{pmatrix}
        \mathbb{A} & \mathbf{1} \\
        \mathbf{1}^T & 0 \\
    \end{pmatrix}
    \begin{pmatrix}
        \mathbf{q} \\
        \lambda \\
    \end{pmatrix}
    \begin{pmatrix}
        \mathbf{X} \\
        q_\text{tot} \\
    \end{pmatrix}
\end{equation}
%
where $q_\text{tot}$ is the total charge of the system, $\mathbf{X}$ is a vector of 
electronegativities given by

\begin{equation}
    X_A = \kappa_A \sqrt{mCN_A} - EN_A
\end{equation}
%
where $mCN_A$ is a modified coordination number, and $\mathbb{A}$ is a charge-charge
interaction matrix that damps interatomic interactions and returns a measure of 
the chemical hardness for intra-atomic elements

\begin{equation}
    \mathbb{A}_{AA} = J_{AA} + \frac{2\gamma_{AA}}{\sqrt{\pi}}
\end{equation}
%
\begin{equation}
    \mathbb{A}_{AB} = \frac{\text{erf}\left(\sigma_{AB} R_{AB}\right)}{R_{AB}}
\end{equation}
%
where $\sigma_{AB}$ is a geometric mean of the atomic radii $\alpha_A$. Solving 
for these charges then gives the energy as 

\begin{equation}
    E_{EEQ} = \mathbf{q}^T \left(\frac{1}{2}\mathbb{A}\mathbf{q} - \mathbf{X}\right)
\end{equation}
%
These charges are used in place of Mulliken charges that would be found in other
energy terms, and as only a single diagonalisation of equation \ref{eq:eeq_solve} 
is required to generate these charges it can be seen that GFN0-xTB does not need
a self-consistent approach to get ground state solutions.

For the most part, the GFN-xTB methods use a minimal basis set, however with some
exceptions for hydrogen atoms. These are constructed from Gaussian functions linearly
combined into Slater type orbitals. The coefficients and number of Gaussian functions
are intrinsic to each GFN-xTB method.

The parameters for the energy terms are either taken from DFT properties (such
as the electronegativities or covalent radii) or by using a "top-down" approach
to optimise the parameters to a set of target properties. This approach is very 
successful for all methods. GFN1-xTB  has a standard relative deviation of around
1.1\% against when compared to geometries from higher level methods, with GFN2-xTB
performing similarly well. Generally the GFN-xTB methods predict non-covalent interaction
energies with a mean absolute deviation of just over 1 kcal/mol, comparable to low
level DFT methods and greatly outperforming other semi-empirical quantum methods.
These successes against the target data shows that using partial charge interactions 
to replace full integrals and global/element-wise parameters to scale interactions
works well in designing efficient but accurate methods.

Another benefit of the xTB approach to tight-binding is the lack of pair-wise parameters.
Often using more approximate but faster energy terms leads to an increase in the
number of fitted parameters, as in order to cover many edge cases it is necessary
to include pair-wise parameters. The issue with these highly specific parameters 
is that they are rarely transferable to other chemical environments and often require
reparameterisation for different target properties. They are often limited to the 
upper parts of the periodic table for this reason. However then GFN-xTB models were
designed to emulate the ZDO type methods that only use element-wise parameters.
This is best demonstrated in the GFN2-xTB method which uses no pairwise parameters.

\section{Light-Matter Response Methods}
\label{sec:response_theories}

\subsection{Linear Response TD-DFT}
\label{subsec:tddft}

Linear response time-dependant DFT (TD-DFT) is a well established method for calculating
excitation energies and transition properties from only ground state information. 
It is formulated from the Runge-Gross theorem\cite{Runge1984}, which states that the
time dependent density of a system can be mapped from the time-dependent external 
potential (for light-matter interactions, the external potential is the light wave), 
and this mapping is unique. This is analogous to the Hohenberg-Kohn theorems used
to derive ground states in section \ref{subsec:dft}. Similar to the Taylor expansion
approach used to derive energy terms for DFTB theory, the density response is usually
expanded in terms of time derivatives. This is valid as long as the time scale for 
response is small, and the excited state density is not too different from the ground
state. The name "linear response" is due to curtailing this expansion at first orders 
terms, which makes TD-DFT practical to calculate \cite{Marques2004}. This is useful
as only the ground state is needed to calculate perturbations to the first order, 
meaning that all transition properties can be calculated from the ground state\cite{Marques2004}.

Response theory is used to predict the changes in the electronic structure of a 
system over a time period, defined by the time-dependent Schrödiner equation

\begin{equation}
    i \frac{\delta}{\delta t} \Psi\left(r, t\right) = \hat{H}\left(r, t\right) \Psi\left(r, t\right)
\end{equation}
%
For excited state, this change is in response to an external light potential, defining
the Hamiltonian as

\begin{equation}
\hat{H}\left(t\right) = \hat{H^0} + v_{\text{ext}}\left(t\right)
\end{equation}
%
where $\hat{H^0}$ is the unperturbed Hamiltonian, $V^{\text{ext}}\left(t\right)$ 
is the potential from the external field, which is usually taken to be an oscillating 
electric field and so is time $t$ dependent. This Hamiltonian can then be used to 
describe a time-dependent set of Kohn-Sham equations\cite{Kohn1952}. The time-dependent
Kohn-Sham Hamiltonian is given by

\begin{equation}
\hat{H}_{KS}\left[\rho\right]\left(t\right) = \hat{H}^0_{KS}\left[\rho\right] + v_H\left[\rho\right]\left(t\right) + v_{\text{XC}}\left[\rho\right]\left(t\right) + v_{\text{ext}}\left(t\right)
\end{equation} 
%
where $V_H$ and $V_{XC}$ are the Kohn-Sham coloumb and exchange-correlation potentials 
respectively. The Runge-Gross theorm states that the density solution to this Hamiltonian
can be uniquely mapped from the potential function, and so the time-dependent density 
can be written as a function of the potential function

\begin{equation}
\rho\left(t\right) = \rho\left[v_{\text{ext}}\right]\left(t\right)
\end{equation}

. The response of the density can be given by the integral of this external potential
with what is referred to as the response function $\chi$

\begin{equation}
    \delta \rho \left(r, \omega \right) = \int d^3 r^\prime \chi \left(r, r^\prime, \omega \right) \delta v_{\text{ext}} \left(r^\prime, \omega \right)
\end{equation}
%
, which calculated the fourier transform of the time series $n\left(r, t\right) - n\left(r, t_0\right)$
which can be written explicitly for a Kohn-Sham system of electrons

\begin{equation}
    \chi_{KS} \left(r, r^\prime, \omega \right) = \sum_{\mu\nu} \left(f_\mu - f_\nu \right) \frac{\psi_\mu\left(r\right)\psi^\ast_\mu\left(r^\prime\right) \psi_\nu\left(r^\prime\right) \psi^\ast_\nu\left(r\right)}{\omega - \left(\epsilon_\mu - \epsilon_\nu\right) + i\eta}
\end{equation}
%
where $f_\mu$ is the occupation number of the orbital $\mu$, and $\eta$ is a positive
infinitesimal number. Again similar to ground-state DFT, this response function 
is used in conjunction with correcting exchange-correlation terms to yield the response
function of the real system. This would calculate the response, and therefore the 
excited state, explicitly. However this issue with the theory given so far is that
it does not identify at which frequencies $\omega$ give any appreciable density 
response i.e. where the excited states are. However the response function can be
written as a discretised set of states $m$

\begin{equation}
    \chi\left(r, r^\prime, \omega \right) = \lim_{\eta \to 0^+} \sum_m \left[ \frac{\braket{0|\hat{n}\left(r\right)|m}\braket{m|\hat{n}\left(r^\prime\right)|0}}{\omega - \left(E_m - E_0\right) + i\eta} - \frac{\braket{0|\hat{n}\left(r^\prime\right)|m}\braket{m|\hat{n}\left(r\right)|0}}{\omega + \left(E_m - E_0\right) + i\eta} \right]
\end{equation}
%
, where $\ket{0}$ is the ground state with energy $E_0$, $\ket{m}$ is an excited
state with energy $E_m$, and $\hat{n}$ is the density operator. It can be seen that
the poles of the response function would be where $\omega = E_m - E_0$ or where 
the frequencies of light match the excitation energies. Eventually it can be shown 
that these poles can be identified by solving the eigenvalue equation

\begin{equation}
\label{full_cassida_eq1}
\left(\begin{matrix}
\mathbf{A} & \mathbf{B} \\
\mathbf{B^*} & \mathbf{A^*}
\end{matrix}\right)
\left(\begin{matrix}
\mathbf{X}\\
\mathbf{Y}
\end{matrix}\right)
=
\left(\begin{matrix}
\omega & 0\\
0 & -\omega
\end{matrix}\right)
\left(\begin{matrix}
\mathbf{X}\\
\mathbf{Y}
\end{matrix}\right)
\end{equation}
%
where $\omega$ are the excitation energies, and the vectors $\mathbf{X}$ and $\mathbf{Y}$ 
describe the electronic transitions in the basis of ground state molecular orbitals. 
The elements of matrices $\mathbf{A}$ and $\mathbf{B}$ are given by

\begin{equation}
A_{ia,jb}\left(\omega\right) = \delta_{ij}\delta_{ab}\left(\epsilon_a - \epsilon_i\right) + \int dr_1 \int dr_2 \psi_i^*\left(r_1\right) \psi_a\left(r_1\right) f_{\text{XC}}\left(r_1, r_2, \omega\right) \psi_i\left(r_2\right) \psi_a^*\left(r_2\right)
\end{equation}
%
\begin{equation}
B_{ia,jb}\left(\omega\right) = \int dr \int dr' \psi_i^*\left(r_1\right) \psi_a\left(r_1\right) f_{\text{XC}}\left(r_1, r_2, \omega\right) \psi_i\left(r_2\right) \psi_a^*\left(r_2\right)
\end{equation}
%
$i$,$j$ and $a$, $b$ are occupied and virtual orbital indices respectively, $\epsilon_i$
are the orbital energies the ground state orbitals, and $\delta$ is the usual kronecker 
delta function. The kernel function $f_{\text{XC}}$ is an exact frequency (excitation energy) 
dependent exchange-correlation functional, and as it is dependent on the excitation 
energy given by the solutions of this eigenvalue equation, can be seen to be self-consistent.

In some cases it is useful to make some approximations. First is that the coupling 
matrix elements are in fact zero. Hence the excitation energies are just the eigenvalue differences, 
giving the eigenvalue difference method discussed below. Second is to say that the 
kernel function is actually frequency independent. This can then give the matrix 
elements in a more computable form. These could be calculated by any (DFT) method
- for example if using a mix of (GGA) density functional and exact exchange, the
matrix elements would be given by

\begin{equation}
A_{ia, jb} = \delta_{ij} \delta_{ab} \left(\epsilon_a - \epsilon_i \right) + \left(ia|jb\right) - \zeta\left(ij|ab\right) + \left(1-\zeta\right)\left(ia|f_{\text{XC}}|jb\right)
\end{equation}
%
\begin{equation}
B_{ia, jb} = \left(ia|jb\right) - \zeta\left(ij|ab\right) + \left(1-\zeta\right)\left(ia|f_{\text{XC}}|jb\right)
\end{equation}
%
where spatial notation has been used for brevity. $\zeta$ here is the amount of 
exact exchange mixing. The density functional term $\left(ia|f_{\text{XC}}|jb\right)$ 
is given by

\begin{equation}
    \left(ia|f_{\text{XC}}|jb\right) = \int dr \int dr^\prime \psi_i \left(r\right)\psi_a \left(r^\prime\right) \frac{\delta^2 E^{\text{GGA}}_{\text{XC}}}{\delta\rho\left(r\right)\delta\rho\left(r^\prime\right)} \psi_i \left(r^\prime\right)\psi_a \left(r\right)
\end{equation}
%
It has been found that some density functionals are better than others for this
approximation. Recently a benchmarking of different density functionals for the 
\Qy transitions in chlorophylls showed that the lowest error is around 0.1 eV \cite{List2013}. 
TD-DFT has become the work-horse of excited state studies. It is on comparable accuracy
to higher level methods, such as coupled cluster methods, but without the expense\cite{Laurent2013}
and can be calculated from just ground state information with the linear response
approximation.

\subsection{$\Delta$-SCF}
\label{subsec{dscf_and_eigdiff}}

\dscf predicts the transition energy $\Delta E$ of a system as the difference of
the single point energy $E_n$ of two states:

\begin{equation}
\Delta E = E_{2} - E_{1}
\end{equation}
%
It is usually assumed that the excited state solution will be in a similar
location to the ground state in the MO coefficient space. Therefore the ground-state
MO coefficients can be used as an initial guess for the excited state. In its simplest
form, the \dscf method calculates the ground-state with normal DFT or other mean-field
methods, and then calculates the excited state by rerunning the same method with 
the excited state occupation numbers. The second set of MO coefficients then give
a full description of excited state.

The issue with finding the excited state solution is that the variation principle
and SCF iterative procedure will try to find the global minimum, which is the 
ground state. The excited state is a local minimum, and so often is less reliable
to find as a solution, especially from the standard SAD initial guess.
For this reason it is often found that converging to the \dscf excited state will
fail. Even when using the ground state as an initial guess with excited state
occupations, normal SCF procedure may still collapse back to the ground state.
Usually it is necessary to include additional changes to the SCF procedure,
such as Fock damping, alternative DIIS methods and sometimes intermediate 
initial guess steps.

Initially, the excited state was calculated by relaxing the orbitals which
contain the excited electron and hole in the ground state space, so that the
excited state and ground state are orthogonal \cite{Hunt1969}. However, it was
argued that this procedure would exacerbate the likelihood of collapsing to the ground
state, and that the excited state was not a proper SCF solution \cite{Gilbert2008}.
Alternatively, an SCF like method was proposed, where instead of
populating orbitals according to the Aufbau principle, orbitals which most
resemble the previous iteration's orbitals should be occupied. This is known as 
the maximum overlap method (MOM). In the maximum overlap method, each iteration 
in an SCF procedure produces new molecular orbital coefficients by solving the 
Roothaan-Hall equations \cite{Roothaan1951}, generally given as an eigenvalue problem:

\begin{equation}
\mathbf{F} \mathbf{C}^{\text{n}} = \mathbf{S} \mathbf{C}^{\text{n}} \epsilon
\label{eq:roothaan_hall}
\end{equation}
%
where $\mathbf{C}^{\text{n}}$ are the $n^{\text{th}}$ orbital coefficient solutions, 
$\mathbf{S}$ is the overlap of orbitals, and $\epsilon$ are the orbital energies. 
The Fock matrix $\mathbf{F}$ is calculated from the previous set of orbital 
coefficients,

\begin{equation}
\mathbf{F} = f\left(\mathbf{C}^{n-1}\right)
\end{equation}
%
. The amount of similarity of orbitals can be estimated from their overlap,

\begin{equation}
\mathbf{O} = \left(\mathbf{C}^{\text{old}}\right)^\dagger \mathbf{S} \mathbf{C}^{\text{new}}
\end{equation}
%
and for a single orbital can be evaluated as a projection,

\begin{equation}
p_j = \sum^n_i O_{ij} = \sum^N_\nu \left[\sum^N_\mu\left(\sum^n_i C_{i\mu}^{\text{old}}\right)S_{\mu\nu}\right]C^{\text{new}}_{\nu j}
\end{equation}
%
where $\mu,\nu$ are orbital indices. the set of orbitals with the highest projection
$p_j$ are then populated with electrons.  This method can be used for any
excited state, with the caveat that the orbital solution will most likely be in
the same region as the ground state solution. For a small number of low lying states
this is generally true, and so \dscf can be used to calculate a small spectrum of
excited states \cite{Gilbert2008}.

\dscf has been shown to be cheap alternative to TD-DFT and other higher level
methods \cite{Liu2004, Gavnholt2008, Besley2009} without considerable losses of
accuracy in certain cases especially for HOMO-LUMO transitions \cite{Kowalczyk2011}.
Additionally, as the excited state is given as solutions to SCF equations, the gradient
of this solution can be given by normal mean-field theory. These gradients would
be much cheaper than TD-DFT or coupled cluster methods, which is advantageous for simulating
dynamics \cite{Gavnholt2008}.

\subsection{Eigenvalue Difference}
\label{subsec:eigval_diff}
Another approximation to full response theory is the eigenvalue difference method. 
Here there is assumed to be no response of the orbital energies and shapes when 
interacting with light. This would be recovered from the complete Cassida equation
if the coupling elements in the $\mathbf{A}$ and $\mathbf{B}$ matrices were set to zero.
Within this approximation, the transition energy is just the difference between 
the ground state energy of the orbital an electron has been excited to($\epsilon_{\text{2}}$)
and the orbital has been excited from ($\epsilon_{\text{1}}$),

\begin{equation}
\Delta E = \epsilon_{\text{2}} - \epsilon_{\text{1}}
\end{equation}
%
. Additionally, transition properties can be calculated by constructing transition 
density matrices from the ground state orbitals such that needing only a single 
SCF optimization is required. Generally, eigenvalue difference methods are not 
seen as accurate response methods, but can offer a quick and easy initial value 
\cite{Gimon2009}.

\subsection{Transition Density and Dipole Moments}
\label{subsec:dscf_transition_density}
\dscf transition properties, such as the transition dipole moment, can be calculated from
the SCF solutions for the ground and excited states. The reduced one-particle transition
density matrix $\mathbf{D}^{21}$ can be written as

\begin{equation}
\mathbf{D}^{21} = \ket{\Psi_1} \bra{\Psi_2}
\end{equation}
%
where $\ket{\Psi_n}$ is the Slater determinant of state $n$, constructed from the
set of spin orbitals $\{ \phi_{j}^{\left(n\right)} \} $. Expressed in terms of
the molecular orbitals coefficients $\mathbf{C}^{\left(n\right)}$, the transition 
density matrix is

\begin{equation}
\mathbf{D}^{21} = \mathbf{C}^{\left(2\right)} \text{adj}\left(\mathbf{S}^{21}\right) \mathbf{C}^{\left(1\right) \dagger}
\end{equation}
%
where $\mathbf{S}^{21}$ is an overlap matrix with elements 

\begin{equation}
S^{21}_{jk} = \braket{\phi^2_j|\phi^1_k}
\end{equation}
%
. The dependence on the adjunct of the overlap is due to the use of L{\"o}wdin's
rules for non-orthogonal determinants \cite{Lowdin1955}. In the same way, the transition
dipole moment is given by

\begin{equation}
\braket{\Psi_2|\hat{\mathbf{\mu}}|\Psi_1} = \sum_{jk} \mathbf{\mu}_{jk}^{21} \text{adj} \left( \mathbf{S}^{21}\right)_{jk}
\end{equation}
%
where $\hat{\mathbf{\mu}}$ is the one-electron transition dipole operator, and
$\mu_{jk}$ is the element of this operator corresponding to orbital indices $j$, $k$.
The determinant of $\mathbf{S^{21}}$ can be defined as the inner product of the 
two states involved in the transition

\begin{equation}
\left\lvert {\mathbf{S}^{21}} \right\rvert = \braket{\Psi_2|\Psi_1}
\end{equation}
%
The general definition of the transition dipole

\begin{equation}
\mathbf{\mu}^{1\rightarrow2} = \braket{\Psi_2|\hat{\mathbf{\mu}}|\Psi_1}
\end{equation}
%
can be expressed with this transition density matrix as

\begin{equation}
\begin{split}    
\braket{\Psi_2|\hat{\mathbf{\mu}}|\Psi_1} &= \text{tr}\left(\hat{\mathbf{\mu}} \ket{\Psi_1} \bra{\Psi_2} \right) \\
&= \text{tr}\left( \hat{\mathbf{\mu}} \mathbf{D}^{21}\right)
\end{split}
\end{equation}
%
\section{Large System Models}
\label{sec:large_systems_theory}

There is nothing inherent in any of the theories above would exclude them from modeling
an entire light harvesting system. Some reservations may be made about the accuracy
of the results, but technically the size of the system is unlimited. However in 
practice system sizes are limited due to two factors - one is the computational 
power available, and the other is the scaling of calculations with respect to system
size. DFT and DFTB for example both have a scaling of , but DFTB drastically decreases
the overhead and scaling prefactors due to the many approximations used. This leads
to DFT being used for systems of around ~$10^1$ atoms at best, whereas DFTB can 
go up to ~$10^2$. This is all dependent on the programs, with some massively parallelised
codes fixing these issues. It is also possible to use a linear scaling DFT formalism,
but this only outperforms systems of huge sizes.
A similar story is found in the response methods. TD-DFT, whilst far less expensive
than coupled cluster or other higher level methods, is still limited to the $10^1$ 
atom range, as is DFT based $\Delta$-SCF and eigenvalue difference methods. Whilst
it may be possible to use tight-binding schemes, there is a better approach to achieve
the required scaling for light harvesting systems. This is done by splitting the
system up into sites where excitations can occur (trivial for naturally occuring 
light harvesting complexes as these sites are just the chlorophyll molecules) and
modeling the excitation energy transfer to generate excited states of the entire
system. This is explained in the next section.

\subsection{Excitation Energy Transfer and Frenkel Exciton Hamiltonians}
\label{subsec:frenkel_exciton_theory}

Excitation energy transfer (EET) stabilizes the absorption of a photon of light,
and it is thought that light harvesting systems have evolved to maximize this stability\cite{Cleary2013}.
In regimes with strong couling (ie large transfer), excitation energy can flow back
and forth between different sites. More explicitly, transfer can be described as
the transition from an initial site to another site

\begin{equation}
\ce{D^* + A -> D + A^*}
\end{equation}

where $D$ and $A$ are the donor and acceptor system respectively, and $*$ denotes
the excited state. The superposition state can be given by the linear combination 
of the inital and final states:

\begin{equation}
\ket{\phi} = c_1 \ket{D^*A} + c_2 \ket{DA^*}
\end{equation}

where $c_n$ are the cofficients of each acceptor-donor state. In a Frenekl-Davydov
model the total Hamiltonian for an aggregate of chromophore sites can be written
as a sum of individual site Hamiltonians and inter-site interactions

\begin{equation}
\hat{H}_{\text{tot}} = \sum^N_m \hat{H}_m + \sum^N_m \sum^N_n \left(\hat{V}_{\text{el-el}} + \hat{V}_{\text{el-nuc}} + \hat{V}_{\text{nuc-nuc}}\right)
\end{equation}

where $N$ is the number of sites, $\hat{H}_m$ is on-site Hamiltonian (ie only treats
electrons for the site $m$). $\hat{V}_{\text{el-el}}$, $\hat{V}_{\text{el-nuc}}$ 
and $\hat{V}_{\text{nuc-nuc}}$ are the electron-electron, electron-nuclear and nuclear-nuclear
interactions between two sites respectively. It is assumed that the site Hamiltonians
can give both the ground and excited states of the chromophore $m$, i.e. it satisfies
the time-independent Schrödinger equation

\begin{equation}
\hat{H}_{a_m} \ket{\phi_{a_m}} = E_{a_m} \ket{\phi_{a_m}}
\end{equation}

where ${a_m}$ is the index of either the ground or an excited state for the chromophore
$m$. There are then three approximations we can make to simplify the total Hamiltonian.
First is to assume that there is only one site excitation, and this is to the first
excited state. This reduces the complexity of any solution to the total Hamiltonian,
and is known as the Heitler-London approximation\cite{Agranovich2000}. Second is 
that the chromophore sites are well separated enough that the wavefunctions do not
overlap\cite{Frenkel1931}. We can then use a Hartree product to construct the aggregate
wavefunctions from a basis set of individual chromophores

\begin{equation}
\ket{\Psi_\text{tot}} = \sum_{a_m} c_{a_m} \ket{\Phi_{a_m}}
\end{equation}

\begin{equation}
\ket{\Phi_a} = \prod^N_m \phi_{a_m}
\end{equation}

where $c_{a_m}$ are the coefficients for each site, given from the eigenvectors 
of the time-independent Schrödiner equation. $\ket{\Phi_{a_m}}$ is the total chromophore
state, and $\phi_{a_m}$ are the one-electron orbitals on the chromophore site $m$.
The third approximation is to neglect the nuclear-electron and nuclear-nuclear interactions.
This can be done if the effects of the surrounding nuclear environment are treated
in the individual site Hamiltonians\cite{Scholes2003} - this includes the other 
chromophore sites as well as any other environments ie. the LH2 protein environment.
Alternatively, this term can just be ignored.

The total Hamiltonian can then just be expressed as the site Hamiltonians and an 
electronic coupling element between two sites. The total Hamiltonian can then be
constructed from the basis of single chromophores. Starting from the excitonic wavefunctions,
the Hamiltonian can be written as

\begin{equation}
\begin{aligned}
\bra{\Phi_a} \hat{H}_{\text{tot}}\ket{\Phi_b} &= \sum^N_m \bra{\Phi_a} \hat{H}_{m}\ket{\Phi_b} + \sum^N_m \sum^N_n \bra{\Phi_a} \hat{V}_{\text{el-el}}\ket{\Phi_b} \\
&= E^{\left(m\right)}_{a_m} \prod^N_l \delta_{al} \delta_{bl} + \sum^N_m \sum^N_n \bra{\Phi_a} \hat{V}_{\text{el-el}}\ket{\Phi_b}
\end{aligned}
\end{equation}

where the first term are the site energies, as the exciton states are orthogonal. 
The second term needs to be expanded into the basis set of individual chromophore 
sites, so we can calculate this term from individual response calculations. Each 
term in the double summation can be given as

\begin{equation}
\begin{aligned}
\bra{\Phi_a} \hat{V}_{\text{el-el}}\ket{\Phi_b} &= \sum_{i \in \mathbf{r}_m} \sum_{j \in \mathbf{r}_n} \bra{\Phi_a} \frac{1}{\left|\mathbf{r_i} - \mathbf{r_j}\right|} \ket{\Phi_b} \\
&= \left(\sum_{i \in m} \sum_{j \in n} \bra{\phi^{\left(m\right)}_{a_m} \phi^{\left(n\right)}_{a_n}} \frac{1}{\left|\mathbf{r_i} - \mathbf{r_j}\right|} \ket{\phi^{\left(m\right)}_{b_m}\phi^{\left(n\right)}_{b_n}}\right)\prod^N_{l\neq m,n}\delta_{a_l, b_l}
\end{aligned}
\end{equation}

where $i,j$ are the indices of electrons on the sites $m,n$, and $\mathbf{r}_i, 
\mathbf{r}_j$ are their positions. With a few extra steps not included for brevity,
this expression can then be written in terms of the transition densities of each 
site\cite{Scholes2003}

\begin{equation}
\bra{\Phi_a} \hat{V}_{\text{el-el}}\ket{\Phi_b} = \int d\mathbf{r}_{m} \int d\mathbf{r}_{n} \frac{\rho_{a_m}\left(\mathbf{r}_{m}\right)\rho_{b_n}\left(\mathbf{r}_{n}\right)}{\left|\mathbf{r}_{m} - \mathbf{r}_{n}\right|}
\end{equation}

where $\rho_{a_m}$ is the transition density of transition $a$ on site $m$. Often
the Frenkel Exciton Hamiltonian is summarised as

\begin{equation}
\hat{H}_{\text{eff}} = \sum^N_{m=1} \epsilon_m + \sum^{N,N}_{m \neq n} V_{mn}
\end{equation}

where the diagonal elements $\epsilon_m$ are site energies of the chromophores, 
and the off-diagonal elements $J_{mn}$ are the coulombic coupling between the tranisiton
density of different sites. It is usual to take reduce the transition density with
a multipole expansion\cite{Steinmann2015}. This could either by done on a whole site
scale or at a more detailed atom-in-site scale. Looking at the coarser-grained whole
site scale first, the first term in the reduction, corresponding to monopoles, is
zero as a local excitation will not produce an overall transition charge (as opposed 
to a non-local charge-transfer excitation, which can be included in the Frenkel 
Hamiltonian in some cases\cite{Li2017}). Using second term in this expansion gives
a dipole-dipole interaction, referred to as the point-dipole method
i.e.

\begin{equation}
    V_{mn} = \frac{\mu_m \mu_n}{R_{mn}^3} - 3 \frac{\left(\mu_m \cdot \mathbf{R}_{mn}\right) \left( \mu_n \cdot \mathbf{R}_{mn} \right)}{R_{mn}^5}
\end{equation}

and usually the expansion is stopped here. Alternatively reducing the transition 
density at the atomic level expansion would give an atom centred multipole expansion.
The first term of which corresponds to transition charges, giving a Coulombic interaction

\begin{equation}
    V_{mn} = \sum_{A>B,A \in m, B \in n}\frac{q_A q_B}{\left\lvert R_A - R_B \right\rvert}
\end{equation}

. Other methods for coupling interactions also exist, such as the extended dipole,
transition charges from electrostatic potential and transition density cube methods
(i.e. a grid approach of discretising the continuous transition density). Intuitively,
using a more detailed description of the site transition density results in more 
accurate couplings. However past the point dipole method, which often overestimates
coupling energies, there are diminishing returns in using more expensive methods.
At large separations (>20 $\AA{}$), all of these methods converge to the same value.

Regardless of the interaction method used, all that is needed to calculate a rough
Frenkel exciton Hamiltonian is the local excitations and transition properties for 
each site. This can be done with many response methods, such as time-dependent density
functional theory (TDDFT) or mean-field methods like \dscf. This is also often done
with either DFT or DFTB based methods, or with statistical approaches. 
 
