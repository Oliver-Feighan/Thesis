%
% File: outline.tex
% Author: Oliver J. H. Feighan
% Description: Thesis outline
% Thesis outline
%
%\let\textcircled=\pgftextcircled
\chapter{Literature Review}
\label{chap:lit_rev}

\initial{T}he issue with designing light harvesting models is the scale of the system.
For LH2, the chlorophyll system contain 3780 atoms, or $N$ atoms for the entire protein.
Including a membrane and solvent can quickly leads to the region of 300,000 atoms.
Generally DFT calculations are not tenable on these types of system sizes, hence 
the use of the Frenkel exciton method. However the method used to construct the 
Frenkel Hamiltonian often compromise between accuracy and efficiency. Using DFT
calculations to calculate the matrix elements for every geometry is still a tall
order, and often statistical approaches are used. These may use accurate properties
as a baseline, but much of the small details can be lost due to averaging. On the
other side, more efficient response methods may be able to construct a greater number
of matrix elements, but the accuracy may not be comparable to higher level methods.
Often depending on the study either efficiency or accuracy is the deciding factor.

The space between these tradeoffs has been subject to recent discussion in the literature,
with novel approaches aiming to achieve accuracy and efficiency. While these approaches
make significant steps in closing this tra

\section{Efficient Response Methods}
\label{sec:efficient_response_methods}

\subsection{sTDA-xTB}
\label{subsec:stda_xtb}
sTDA-xTB ("simplified Tann-Dancoff Approximation - eXtended Tight Binding") is another
method in the family of xTB methods developed by the Grimme group, and is parameterised
for transition properties \cite{Grimme2016}. The accuracy in calculating transition energies with this
method is very good, with the error compared to high-level method, such as SCS-CC2,
being around 0.3 - 0.5 eV.

Similar to other xTB methods, the sTDA-xTB method is a tight-binding method that
uses empirically fitted parameters and a minimal basis set. It was trained on a
test set of highly accurate coupled cluster and density functional theory
excitation energies, as well as atomic partial charges for inter-electronic interactions.

Unlike other xTB methods, coefficients in the basis set for sTDA-xTB are dependent on the D3
coordination number. This makes basis functions far more flexible, which would usually
be achieved with fixed basis functions by using diffuse or other additional orbitals in
the basis set. Additionally, it uses two sets of parameterized basis sets - a
smaller valence basis set (VBS) and an extended basis set (XBS). Whilst this reduces
the cost of having larger basis sets, it makes calculating the gradient of transition
properties much more difficult. This motivates the work on designing an alternative
method with more tractable gradients, instead of using this already established method.

The two basis sets are used to construct formally similar Fock matrix elements,
although in practice they use different global parameters. The core Hamiltonian
is similar to other DFTB methods that use a self-consistent charge (SCC) method, as
opposed to an SCF method, to obtain molecular orbital coefficients. It is given by,

\begin{equation}
\bra{\psi_\mu} H^{\text{EHT, sTDA-xTB}} \ket{\psi_\mu}= \frac{1}{2} \left(k^l_\mu k^{l'}_\nu\right) \frac{1}{2} \left(h^l_\mu h^{l'}_\nu\right) S_{\mu\nu} - k_T \bra{\psi_\mu}\hat{T}\ket{\psi_\nu}
\end{equation}
%
where $\mu,\nu,l,l'$ are orbital and shell indices, $k^l_\mu$ are shell-wise 
H{\"u}ckel parameters, $h$ are effective atomic-orbital energy levels, $S_{\mu\nu}$
is the overlap of orbitals $\mu$ and $\nu$, $k_T$ is a global constant and $\hat{T}$
is the kinetic energy operator. The charges used in the inter-electronic repulsion 
function are given by charge model 5 (CM5) \cite{Marenich2012} charges for the XBS
Fock matrix. These are calculated using Mulliken charges obtained from diagonalising
the Fock matrix with the VBS. The charges for the initial VBS Fock matrix are based
on Gasteiger charges \cite{Gasteiger1978}, modified by the parameterised
electronegativities of atoms in the system.

The whole process for determining molecular orbitals can be summarized as:
\begin{enumerate}
	\item Calculate modified Gasteiger charges for the first initial guess
	\item Diagonalise Fock matrix in the VBS to get the first set of Mulliken charges
	\item Compute CM5 charges
	\item Diagonalise Fock matrix in the VBS again for final set of Mulliken charges.
	\item Recalculate CM5 charges with this final set, and diagonalize the Fock matrix in the XBS. The molecular orbital coefficients from this are then fed to the response theory.
\end{enumerate}

The response theory for this method is based on previous work in the Grimme 
group on the simplified Tamm-Dancoff Approximation \cite{Grimme2013}.
There are several approximations made between full linear response theory and
the sTDA method. First is the Tamm-Danncoff approximation, where the $\mathbf{B}$
matrix is ignored. The second approximation is to use monopole approximations with
Mataga-Nishimoto-Ohno-Klopman (MNOK) operators instead of explicit 2 electron integral
as well as neglecting the density functional term.

Transition charges are used to calculate these MNOK integrals. The charge $q^A_{nm}$
centred on atom $A$ associated withthe transition from $ n \rightarrow m$, are
computed using a Löwdin population analysis:

\begin{equation}
q_{nm}^A = \sum_{\mu \in A} C^\prime_{\mu n} C^\prime_{\mu m}
\end{equation}

where the transformed coefficients $C^\prime_{\mu n}$ are given by orthogonalising
the original MO coefficients $\textbf{C}$:

\begin{equation}
\textbf{C}^\prime = \textbf{S}^{\frac{1}{2}} \textbf{C}
\end{equation}

and $\mu$ is an index that runs over the atomic orbitals (AO). The MO coefficients
are the solution of diagonalising the Fock matrix, similar to equation \ref{eq:roothaan_hall}.

Approximations to full 2 electron integrals are given by charge-charge interaction
damped by the MNOK\cite{Nishimoto1957}\cite{Ohno1964}\cite{Klopman1964} functions.
For exchange and coloumb type integrals, difference exponents are used, along with
an additional free parameter to recover the amount of Fock exchange mixing in
the original matrix element equation. These will be discussed in more detail in
the next chapter, as they are a crucial part of designing a new response method 
for chlorophyll systems.

Third is the truncation of single particle excited space that is used to construct 
the $\mathbf{A}$ matrix. This reduces the number of elements that need to be 
calculated, and so reduces the time taken for diagonalisation, whilst also capturing 
a broad enough spectrum of excitation energies. The sTDA-xTB has many of the same 
goals as this project, except in one respect, which is the gradient theory. As 
the sTDA-xTB method still requires constructing and diagonalizing the $\mathbf{A}$ 
matrix, albeit with a tight-binding method for molecular orbital coefficients, 
the gradient of the transition properties would still be difficult to calculate.

\section{Statistical Approaches}
\label{sec:stats_models}



\section{Hardware}
\label{}

\section{Conclusions}
\label{sec:lit_review_conclusions}
