%
% file: app0A.tex
% author: Oliver J. H. Feighan
% description: program and hardware appendix
%

\chapter{Appendix A}
\label{app:app01}

\initial{T}his appendix covers the common computational details of this work such
as the software packages and hardware used. These are not exhaustive list, and additional 
details are provided in the main chapters.

\section{Electronic Structure Codes}
This project has primarily used the \code{QCORE} software for electronic structure
calculations. This includes all of the chl-xTB and \dxtb calculations, as well as
some of the \dscf calculations. At time of writing the only implementation of chl-xTB
and \dxtb are only found in \code{QCORE}, although other packages can do more general
\dscf calculations.
\code{QCORE} is a program that is part of the \code{ENTOS} project. This is a software 
package for DFT and DFTB electronic structure calculations that has been written 
as a joint venture between the Miller group in California Institute of Technology 
and the Manby group in the University of Bristol. It is now being hosted by Entos Inc. 
It is a novel C++ implementation, with a focus on modularity, functional code and
modern development practices to enable easier, cleaner and more reuseable code.

Other calculations, including TD-DFT and ZINDO calculations, were done with the 
\code{GAUSSIAN} program \cite{Gaussian16}.

\section{Computational Hardware}

Much of the method development and analysis was done on a 2017 MacBook Pro with 
an 2.3 GHz Intel Core i5 processor. Running small scale calculations was also done
on this machine.

Larger calculations (e.g. MD simulations, TD-DFT on full chlorophyll, chl-xTB parameter
optimization) utilized the high performance computing systems at the University 
of Bristol. This work was carried out using the computational facilities of the 
Advanced Computing Research Centre, University of Bristol - http://www.bristol.ac.uk/acrc/.

