%
% File: outline.tex
% Author: Oliver J. H. Feighan
% Description: Thesis outline
% Thesis outline
%
%\let\textcircled=\pgftextcircled
\chapter{Outline}
\label{chap:outline}

This thesis focuses on the problem of having efficient yet extendable methods for
photochemical problems on large systems. The main developments is on how semi-empirical
tight binding methods can be used to predict properties for the LH2 protein. It 
is argued in the introduction and literature review that many methods often compromise
between accuracy of high level methods and efficiency of low level or statistical 
methods. This compromise is exacerbated when the size of systems gets bigger, for
example with the LH2 protein where the chlorophyll system is made up of $\sim 4000$
atoms. The new work presented here includes a new benchmarking of previous methods
and design of novel methods which could offer new insight to these problems. The
novel method is used for systems ranging in size from a single chlorophyll to dimers
and finally a full LH2 chlorophyll system. At appropriate stages it is benchmarked
against relevant data, and is used to make novel arguments that explain chlorophyll
system phenomena.

The work is structured into 7 chapters. \textbf{Chapter \ref{chap:intro}} introduces
the ideas and theory referred to throughout, with \textbf{Chapter \ref{chap:background_theory}} 
being a literature review of how these ideas have been explored in recent work and
where the gaps in this lead to the work reported here. The following chapters report
on the results of the work done.

\textbf{Chapter \ref{chap:dscf}} reports on work done on investigating
mean-field methods, such as \dscf and eigenvalue difference, as a more approximate
method for transition properties and whether it would be a good candidate for large
bulk chlorophyll response properties. It also discusses how the underlying level
of electronic structure theory affects the accuracy of transition properties. The
novel work here is the benchmarking of chlorophyll transition properties with \dscf
and TD-DFT functionals, as well as the comparison of transition properties predicted
with xTB based methods to higher level theories.

\textbf{Chapter \ref{chap:chl_xtb}} then uses the findings of the previous chapter to inform
design choices of a novel response method. This method uses previously used approximations
in a new conjunction to predict response properties. While it is not a general 
purpose method, it's application to the \Qy transition of chlorophyll shows excellent
performance at reproducing values from high level theories with great efficiency.
It fulfills the criteria set out in the introduction and literature review, obtaining
response properties with great efficiency whilst also being accurate to high level
data as well as being extendable.

\textbf{Chapter \ref{chap:excitons}} focuses on dimer chlorophyll systems, using an exciton
framework that can extend to multiple chlorophyll systems such as light harvesting
proteins. Comparison to high level theories show that the new workflow can be expected
to give good properties beyond monomer systems. With this workflow it starts to 
be possible to make novel arguments about light harvesting system phenomena. Specifically,
the new workflow is used to explain how the transition from vertical excitation 
to a charge separation in chlorophyll dimers is suppressed by the LH2 protein scaffold,
more than would be expected for the inter-chromophore separation observed in LH2.
These are argued by calculating the rate constants of this transition for a series
of chlorophyll dimer systems.

\textbf{Chapter \ref{chap:LH2}}, the final results chapter, then reports on applying the
novel method to the whole LH2 chlorophyll system. Using the high level of detail
and efficiency new properties of this chlorophyll system are calculated, and further
claims on the role of the protein scaffold are argued. These are mostly based on
spectral densities of LH2 properties, partially characterised by comparison to other
approaches of investigating the effect of environment on chlorophyll transitions.
Whilst the novel workflow is utilised well, it is argued that with further work
more improvements could be made in these final results.

The last chapter, \textbf{Chapter \ref{chap:discussion}}, discusses the conclusions
from the previous results chapters. Investigations such as applying the novel response
method to systems beyond chlorophyll are discussed, as well as alterations to the
exciton benchmarking and framework. More applications to LH2 and other light harvesting
systems are proposed.

