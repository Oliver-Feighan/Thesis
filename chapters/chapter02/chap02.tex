%
% File: chap02.tex
% Author: Oliver J. H. Feighan
% Description: delta-scf benchmarking, including dft and dftb methods.
% Talk about important factors for modelling LHII
%
\let\textcircled=\pgftextcircled
\chapter{Mean-Field excited states}
\label{chap:dscf}

\initial {P}reamble

%=======
\section{Theory}
\label{sec:dscf_theory}

\subsection{$\Delta$-SCF and eigenvalue difference}
\label{subsec{dscf_and_eigdiff}}

\dscf predicts the excitation energy of a system by comparing the single point
energy of the ground state and the excited state. Finding this excited state
correctly can be an issue, but is usually assumed to be similar to the ground
state. In its simplest form, the \dscf method calculates the ground state, and
then calculates the excited state by rerunning an self-consistent field (SCF) 
with the excited state occupation numbers. This then gives a full description 
of both the ground as excited state from the orbital coefficients output from 
the two SCF procedures.

Initially, the excited state could be calculated by relaxing the orbitals which
contain the excited electron and hole in the ground state space, so that the
excited state and ground state are orthogonal\cite{Hunt1969}. However, it was
argued that this proceedure would exacerbate errors from finding the ground
state, and that the excited state was not a proper SCF solution\cite{Gilbert2008}.
Alternatively, it was proposed that an SCF like method, where instead of
populating orbitals according to the aufbau principle, orbitals which most
resemble the previous iteration's orbitals should be occupied. Each iteration 
in an SCF procedure produces new molecular orbital coefficients by solving the 
Roothaan-Hall equations\cite{Roothaan1951}, generally given as an eigenvalue problem:

\begin{equation}
\mathbf{F} \mathbf{C}^{\text{new}} = \mathbf{S} \mathbf{C}^{\text{new}} \epsilon
\end{equation}

where $\mathbf{C}^{\text{new}}$ are the next orbital coefficient solutions, 
$\mathbf{S}$ is the overlap, and $\epsilon$ are the orbital energies. 
The Fock matrix $\mathbf{F}$ is calculated from the previous set of orbital 
coefficients:

\begin{equation}
\mathbf{F} = f\left(\mathbf{C}^{\text{old}}\right)
\end{equation}

The amount of similarity of orbitals can be estimated from their overlap:

\begin{equation}
\mathbf{O} = \left(\mathbf{C}^{\text{old}}\right)^\dagger \mathbf{S} \mathbf{C}^{\text{new}}
\end{equation}

and for a single orbital can be evaluated as a projection:

\begin{equation}
p_j = \sum^n_i O_{ij} = \sum^N_\nu \left[\sum^N_\mu\left(\sum^n_i C_{i\mu}^{\text{old}}\right)S_{\mu\nu}\right]C^{\text{new}}_{\nu j}
\end{equation}

where $\mu,\nu$ are orbital indices. The population can then be given by the set
of orbitals with the highest projection $p_j$.  This method can be used for any
excited state, with the caveat that the orbital solution is in the same region
as the ground state solution. For a few low lying states, this is generally 
true, and so \dscf can be used to calculate a spectrum of excited states\cite{Gilbert2008}.
The method of using this orbital overlap is called the maximum overlap method (MOM).

\dscf has been shown to be cheap alternative to TDDFT and other higher level
methods, without considerable losses of accuracy in certain cases\cite{Worster2021}.
Additionally, as the excited state is given as solutions to SCF equations,
the gradient of this solution can be given by normal mean-field theory.
These gradients would be much cheaper than TDDFT or coupled cluster methods, and
so would be advantagous for a dynamic simulation of LHII.

The final descent in response theory would be to eigenvalue difference methods. 
Here there is assumed to be no response of the orbital energies and shapes when 
interacting with light. As stated earlier, this would be recovered from the
complete Cassida equation if the coupling elements in the $\mathbf{A}$ and 
$\mathbf{B}$ matrices are set to zero. This means that the difference between 
the excited state energy and the ground state energy is just the difference of
the orbital energies between the orbital an electron has been excited to and the
orbital has been excited from. Additionally, transition properties can be 
calculated by calculating transition density matrices from only the ground state.
Hence, all the information needed can be given by a single SCF optimization. 
Generally, eigenvalue difference methods are not seen as accurate response methods,
but can offer a quick and easy initial value\cite{Gimon2009}.

\subsection{Semi-empirical extensions}
\label{subsec:dscf_xtb}
We tried to extend the range of DFT methods that could be used for \dscf and 
eigenvalue difference methods by investigating whether a tight-binding method
could predict transition properties.
We chose the recently published DFTB method parameterized by the Grimme group for
this. This method has been parameterized for geometries, frequencies and non-covalent
interactions, and uses an extended version of H{\"u}ckel theory. The name they
present is GFN-xTB, standing for "Geometries, Frequencies, Non-Covalent - eXtended 
Tight Binding".
We chose this method for two reasons. The first being that it was already 
implemted


\section{Benchmarking}
\label{sec:benchmarking}

\subsection{Small Systems}
\label{subsec:smalltest}

\subsection{LHII Chlorophyll}
\label{subsec:dscf_chl_tests}

\subsection{GFN methods}
\label{subsec:dscf_gfn_tests}


\section{Extensibility}
\label{sec:dscf_problems}

\subsection{Non-orthogonality}
\label{subsec:dscf_nonorth}

\subsection{Embedding}
\label{subsec:dscf_embedding}

\subsection{Scaling}
\label{subsec:dscf_scaling}