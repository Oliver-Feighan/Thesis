%
% File: chap04.tex
% Author: Oliver J. H. Feighan
% Description: Excitons
%
\let\textcircled=\pgftextcircled
\chapter{Beyond Monomer Chlorophyll}
\label{chap:excitons}

\subsubsection*{Previous Published Work}
Parts of the work presented in this chapter are also included in a paper published 
with Dr Susannah Bourne-Worster. These are reported in section \ref{sec:concentration_quenching}.


\initial {T}his chapter investigates whether chl-xTB is still an accurate method 
for oligomer chlorophyll system excited states, beyond the monomers reported in 
the previous chapter. While monomer transition properties have been shown to be 
well predicted by chl-xTB, it is still necessary to show how this would perform
in exciton frameworks that are used for LHCs.

The work here is in two parts. First is the benchmarking the exciton framework on
dimer systems. Chlorophyll dimer systems were chosen as a test as opposed to the
larger systems found in LHCs as their smaller size makes high-level methods tenable.
Hence while this comparison may not be as relevant as a system closer to LHCs it 
would allow for more detailed comparisons. 

The second part applies the dimer exciton system to predict the rate of transition 
to a charge separation state in chlorophyll dimers, and important mechanism in the
quenching of excited states. This is done to illustrate how the novel response method
designed in the previous chapter can solve the issues presented in the introduction.
Aspects such as the efficiency, accuracy and extendability were all necessary in
order to calculate these rates, and this would have been more difficult without 
the novel chl-xTB response method.

The methods discussed in the introduction are good generators of exciton Hamiltonian
matrix elements. They are workable solutions to the problem of dealing with the
large systems present in LHCs. However it could be argued that they are limited
by some design choices, for example the machine-learning methods not being able
to include embedding effects. It is explored here whether chl-xTB solves some of
these issues or is similarly constrained by design choices (i.e. the density functional 
and conformer space used to construct the training data).

Applications to larger LHCs would be trivial after establishing that the dimer exciton
system would work. This would just be a matter of using a system with more basis
sites, for example the 27 chlorophyll sites in LH2. This exciton system is analyzed
in the next chapter as part of a investigation into LH2 properties that can be reliably
calculated with the chl-xTB system.

%=======
\section{Frenkel Exciton Hamiltonian}
\label{sec:exciton_theory}

\subsection{Exciton States}
\label{subsec:exciton_states}
As stated in the introduction, a large majority of models of light harvesting systems
utilize a Frenkel-Davydov Hamiltonian model of exciton states \cite{Curutchet2016, Cignoni2022}, 
where the weak coupling between monomers mean that a model can be constructed from
properties of the individual sites.

To recap the exciton theory from chapter \ref{chap:background_theory}, the exciton 
states $\ket{\Psi}$ can be constructed as a Hartree product of the states of individual 
sites (the monomers or chromophores in LHCs)

\begin{equation}
    \ket{\Psi} = \Pi_m \ket{\phi_m}
\end{equation}
%
where $\ket{\phi_m}$ is the monomer state on site $m$. These monomer states form
the basis function of the overall excitonic states. As the exciton is modelled to
be localised to a specific site, an exciton state with an exciton at site $i$ is
given by

\begin{equation}
    \ket{\Psi}^i = \ket{\phi_i}^\ast \Pi_{m \neq i}  \ket{\phi_m}
\end{equation}
%
where $\ket{\phi_i}^\ast$ is the excited state of monomer site $i$. States with 
more than one exciton are possible in this framework, however this is not usual 
for many light harvesting system models, as explained in the introduction. The Hamiltonian, 
including the "ground state" where there are no excitons in the system, is given 
by

\begin{equation}
    H = 
    \begin{bmatrix}
        E_0 & V_{0, \left(1, 1\right)} & \cdots & V_{0, \left(N, 1\right)} \\
        V_{0, \left(1, 1\right)} & E_{\left(1,1\right)} & \cdots & V_{\left(1,1\right) \left(N, 1\right)} \\
        \vdots & \vdots & \ddots & \vdots \\
        V_{0, \left(N, 1\right)} & V_{\left(1,1\right) \left(N, 1\right)} & \cdots & E_{\left(N, 1\right)}
    \end{bmatrix}
\end{equation}
%
where $N$ is equal to the total number of individual sites. The diagonal terms $E$
is the sum of the site energies and a point charge interaction

\begin{equation}
    E_0 = \sum_m e_m + \sum_{m \neq n, A \in m, B \in n} \frac{q^A_m q^B_n}{r_{AB}}
\end{equation}
%
where $e_m$ is the energy of site $m$ and $q^A_m$ is the charge centered on atom
$A$ in site $m$. Hamiltonian elements corresponding to a single excitation are similarly
given except with the excited state energy and charges

\begin{equation}
    E_{\left(m,1\right)} = e_m + \delta e_m + \sum_{n \neq m} e_n + \sum_{n, A \in m, B \in n} \frac{q^{\ast A}_m q^B_n}{r_{AB}} + \sum_{n,p \neq m, A \in n, B \in p} \frac{q^{A}_n q^B_p}{r_{AB}}
\end{equation}
%
where $\delta e_m$ is the excitation energy of site $m$, and charges marked $q^\ast$
are the excited state charges. The inter-chromophre term has been replaced to explicitly
show that both interaction of excited state point charges with other sites, as well
as ground-state/ground-state interactions, are included.

The off diagonal elements are the coupling elements between all exciton states.
For coupling to the ground state, these are given by

\begin{equation}
    V_{0, \left(m,1\right)} = \sum_{n, A \in m, B \in n} \frac{q^{\text{tr},A}_m q^B_n}{r_{AB}}
\end{equation}
%
where charges marked $q^{\text{tr}}$ are transition charges. It can be seen that 
this coupling element includes an electrostatic interaction between all sites and
the excited site. This is different to coupling elements between two single exciton
states, given as

\begin{equation}
    V_{\left(m, 1\right), \left(n,1\right)} = \sum_{A \in m, B \in n} \frac{q^{\text{tr},A}_m q^{\text{tr},B}_n}{r_{AB}}
    \label{eq:exciton_coupling}
\end{equation}
%
which just involve sites $m$, $n$ which have local excitations.

This Hamiltonian for exciton states is slightly different to those previously reported 
in the literature, as the ground state, where no excitons are present, is not usually
included. Usually the local excitations are sufficiently high in energy and the
couplings between ground and excited states are weak enough that the ground state 
eigensolution is usually unmixed with the local excitations. The excited state eigensolutions
have very little ground state character, and so the block matrix of just excited
state contributions would return the Hamiltonian used in many other studies.

\section{Same-functional Benchmarking}
\label{sec:exction_v_full_dimer}

It has been well established in many other studies that the Frenkel exciton Hamiltonian
can predict dimer response properties well. This type of study was repeated here
to set a benchmark for the later comparison with chl-xTB. Well known causes of error
between the exciton model and full TD-DFT were explored, such as inter-chromophore
distance and treatment of coupling of excited states. This was done with two studies:
first comparing the transition energies predicting by the exciton model against 
full TD-DFT for a set of chlorophylls from LH2; second with a conformational scan
of chlorophyll dimers along rotational axes, to pinpoint if errors may be due to
any specific interactions.

\subsection{LH2 Dimers}
\label{subsec:LH2_exciton_camb3lyp}

Response properties were calculated for a series of chlorophyll dimers taken from
the LH2 protein, using CAM-B3LYP/Def2-SVP TD-DFT. The chlorophyll dimer systems
were taken from the previously used set of LH2 MD structures \cite{Stross2016},
with the phytol tails removed for ease of calculation. Transition energies were
calculated for the full dimer systems, and compared with energies from the exciton 
model. The exciton model was constructed from monomer TD-DFT calculations on the
two chlorophylls in each dimer. The transition energies for the exciton model were
calculated as the difference between the excited and ground states. The scatter
plot of these transition energies is shown in figure \ref{fig:camb3lyp_excitons}. 

\begin{figure}
    \centering
    \includegraphics[scale=0.5]{../../Year_2/DimerModel/CAMB3LYP_exciton_v_tddft_coupling_and_distance.png}
    \caption{The correlation of lowest excitations in LH2 chlorophyll dimer systems
    predicted by the exciton model constructed from monomer TD-DFT and TD-DFT of
    the full dimer system. Scatter points are coloured by the coupling value between
    exciton states (left) and distance between magnesium centres (right).}
    \label{fig:camb3lyp_excitons}
\end{figure}

A clear relationship can be seen between the full dimer calculations and the exciton
model. The systems in which there is any error can be seen to be correlated to the
distance between the monomers, with a lower distance giving a higher error. There 
is a smaller amount of correlation with the coupling value from the exciton Hamiltonian
(taken as the coupling between the excited states, as generally the coupling value
to the ground state is lower), implying that there are effects that are not included
in the exciton model rather than a systematic error in the coupling. These effects
could be from not including higher energy transitions, such as the $Q_x$ transition,
or charge transfer transitions. However for the large majority of cases the exciton
model predicts full dimer TD-DFT with decent accuracy, within the range of previously
reported exciton models.

\subsection{Rotation}
\label{subsec:rotation}

Scans of dimer conformations were done to further investigate breakdowns in the
exciton model. As explored in the introduction, many previous investigations treated
inter-chromophores coupling with a point dipole interaction:

\begin{equation}
    v_{mn} = \frac{\vec{\mu}_m \cdot \vec{\mu}_n}{R_{mn}}
\end{equation}
%
where $\vec{\mu}_m$ is the dipole on site $m$ and $R_{mn}$ is the inter-chromophore
distance. From this expression is can be seen that the two inter-chromophore coordinates
are the angle and distance between the monomers in the dimer pair. With an idea
of how distance affects exciton model accuracy from the LH2 dimers above, the effect 
of angle on accuracy was investigated.

This was done by artificially constructing dimer systems with controlled angles 
between either the planes of the porphyrin rings or the $N_A$-$N_C$ or $N_B$-$N_D$
axis. All dimers were constructed from two truncated chlorophyll molecules with 
the porphyrin planes initially parallel and overlapping. The axis along the magnesium
atoms was used to defined the separation, and as this axis is approximately the 
cross product of the \Qy ($N_A$-$N_C$ axis) and $Q_x$ ($N_B$-$N_D$ axis) transition
dipoles, this axis is labelled $Q_z$. After separation, the angle between monomers
was increased in increments of 3.6 degrees up to a full 360 degree rotation for 
all of the \Qy, $Q_x$, $Q_z$ axes as axes of rotation. The magnesium atom was the 
centre of rotation. For the $Q_z$ rotation the separation was 7 $\AA{}$, a little 
less than the average separation of 9 $\AA{}$ found in LH2 - this was to done to 
maximise the coupling in the exciton model to exaggerate any errors. For the \Qy
and $Q_x$ axes this separation was around 15 $\AA{}$ to be able to fit a full rotation.
The initial system and the axes of rotation can be seen in figure \ref{fig:dimer_system}.

In order to make assignment of exciton states easier, the geometries of the two 
truncated chlorophyll monomers were altered such that there would be a distinct 
gap in their transition energies. This gap was not wider than the variation in 
transition energies found in LH2 monomers. More detail on assigning transition 
energies to the correct excited state is discussed below.

\begin{figure}
    \centering
    \includegraphics[scale=0.4]{../../Year_2/DimerModel/Scans/dimer_system.png}
    \caption{Diagram of the artificial dimer system, showing the axes of rotations
    in green and the chosen sites for calculating distance between functional groups
    in orange.}
    \label{fig:dimer_system}
\end{figure}

Comparisons of the excited state energy can be seen in figures \ref{fig:camb3lyp_Qz_rotation},
\ref{fig:camb3lyp_Qx_rotation}, \ref{fig:camb3lyp_Qy_rotation}. The same trends 
in the profile of excited state energies can be seen both the full dimer system 
and exciton models. The minima and maxima are found at the same positions in both, 
and the qualitative trends in curvature are also similar. The only major discrepancy
is seen in the 150-200 $^{\circ}$ region in the $Q_x$ profile, where the exciton 
model predicts the higher excited state at a little above the TD-DFT profile. A 
similar effect is seen in the LH2 dimers, where the errors in transition energies 
are almost all overestimates, and is attributed to a higher coupling value in the
exciton model than is present in TD-DFT. This higher coupling could be due to the
asymptotic behavior of a bare point charge interaction. Using a short range damping
operator might correct this, but this would be outside the scope of this work.
It should be noted that the angle between nearest neighbour chlorophylls is either
$\approx 0 ^{\circ}$ in the B850 rings, or $\approx 90 ^{\circ}$ for B800-B850 pairs,
and so much of this space is not explored in the LH2 protein. For other light 
harvesting complexes there would be more variation in angle between chlorophyll 
monomers. Additionally, non-nearest neighbour pairs would also explore more of the 
angle space, however these dimers are further apart so would have much lower coupling
values.

\begin{figure}
    \centering
    \includegraphics[scale=0.5]{../../Year_2/DimerModel/Scans/camb3lyp_rotation_along_Qz.png}
    \caption{The dimer excited state energies predicted by full dimer TD-DFT (left)
    and the exciton model (right) for chlorophyll dimer geometries with one monomer
    rotated around the $Q_z$ axis.}
    \label{fig:camb3lyp_Qz_rotation}
\end{figure}

\begin{figure}
    \centering
    \includegraphics[scale=0.5]{../../Year_2/DimerModel/Scans/camb3lyp_rotation_along_Qx.png}
    \caption{The dimer excited state energies predicted by full dimer TD-DFT (left)
    and the exciton model (right) for chlorophyll dimer geometries with one monomer
    rotated around the $Q_x$ axis.}
    \label{fig:camb3lyp_Qx_rotation}
\end{figure}

\begin{figure}
    \centering
    \includegraphics[scale=0.5]{../../Year_2/DimerModel/Scans/camb3lyp_rotation_along_Qy.png}
    \caption{The dimer excited state energies predicted by full dimer TD-DFT (left)
    and the exciton model (right) for chlorophyll dimer geometries with one monomer
    rotated around the $Q_y$ axis.}
    \label{fig:camb3lyp_Qy_rotation}
\end{figure}

Overall it can be seen that the exciton model predicts TD-DFT excited states and
transition energies well. Reasons for error are due to known issues with the exciton
model. These could be addressed by using more detailed methods for calculating the
coupling values between exciton states, or by including other transitions that may
change the character of these states.

\afterpartskip
\section{chl-xTB Excitons}
\label{sec:chl_xTB_excitons}

\subsection{Full Dimer comparison}
\label{subsec:pbe0_and_chl_xtb_dimer}

\subsubsection{Discarding PBE0}
\label{subsec:state_assign}

While in the previous section is was possible to compare excited state energies
with the same CAM-B3LYP functional for both dimer and monomer properties, it was
found this was not the case for chl-xTB as well as the PBE0/Def2-SVP reference method.
For chl-xTB, the obvious flaw is that the method was parameterised for monomer chlorophyll
properties and not dimers. Additionally it was found that dimer response properties 
had systematic errors, which is attributed to the underlying xTB framework.

Ideally, the next-best comparison to replace dimer chl-xTB would have been dimer
PBE0. However it was found that the transitions from PBE0/Def2-SVP were not well
defined, having charge transfer character that is not present in the CAM-B3LYP
dimer transitions. This effect can be seen in the distribution of transition energies, 
shown in figure \ref{fig:camb3lyp_pbe0_distributions}. Here the individual \Qy and 
higher energy transitions that are well resolved by CAM-B3LYP are not well resolved
when using PBE0. This leads these transitions to be uncorrelated to CAM-B3LYP transitions,
as seen in figure \ref{fig:pbe0_camb3lyp_coupling}. Hence the PBE0 dimer data would 
not make a good comparison for either the PBE0 exciton or the chl-xTB exciton models, 
and so was discarded from benchmarking. For the remaining sections, the exciton
model is compared to CAM-B3LYP TD-DFT.

\begin{figure}
    \centering
    \includegraphics[scale=0.6]{../../Year_2/DimerModel/camb3lyp_pbe0_distributions.png}
    \caption{Distributions of the lowest 5 transition energies of chlorophyll dimers 
    predicted by TD-DFT with the CAM-B3LYP functional (grey) and PBE0 functional (red).}
    \label{fig:camb3lyp_pbe0_distributions}
\end{figure}

\begin{figure}
    \centering
    \includegraphics[]{../../Year_2/DimerModel/PBE0_tddft_v_CAMB3LYP_tddft_coupling.png}
    \caption{Scatter of the two lowest transitions from PBE0 TD-DFT against CAM-B3LYP
    TD-DFT, coloured by the coupling value from the exciton model using PBE0 monomer 
    data.}
    \label{fig:pbe0_camb3lyp_coupling}
\end{figure}

\afterpartskip
\subsubsection{Assignment of States}
\label{subsec:state_assign}

In the CAM-B3LYP comparisons, both TD-DFT and the exciton model gave excited states
in the same energy ordering, making assignment straightforward. However it was found
that for some dimer pairs, the exciton states constructed with PBE0 and chl-xTB
would predict show energy ordering. For example, for an A-B  dimer system,
one model might predict the lower energy for a state with the exciton localized 
on monomer A, whereas CAM-B3LYP might predict the state with the exciton on B as
the lower energy state. It should be noted that the CAM-B3LYP exciton model and CAM-B3LYP 
full dimer data were always consistent, and it was only when comparing to PBE0 or
chl-xTB was this effect apparent.

It was therefore necessary to find the location of the excitons in both dimer and
exciton models. For the TD-DFT dimer result, the exciton location was taken as the
molecule where the transition charge distribution was centered. This centre $\mathbf{c}$
was calculated by taking the average of the charge positions $\mathbf{r}$ weighted
by the absolute transition charge value $\left\lvert q^{\text{tr}} \right\rvert$

\begin{equation}
    \mathbf{c} = \frac{\sum_i \left\lvert q^{\text{tr}}_i \right\rvert \mathbf{r}_i}{\sum_i \left\lvert q^{\text{tr}}_i \right\rvert}
\end{equation}
%
where $i$ is an index for all atoms in the dimer. Whilst the "centre of charge"
is generally a poorly defined property for a system of point charges with zero total
charge, these centres predicted with the above equation were found to be located close
to the Mg centre of the truncated chlorophylls, and so is assumed to be a decent
metric for the "transition centre".

For the Frenkel exciton result, the location was taken as the monomer corresponding
to the dominant character in the eigenvector solution. For example, if the diagonal
elements of the Hamiltonian in order of ground state, transition on A, transition
on B, and the eigenvector solution of the lower energy exciton state had the greatest
value in its second element, the "transition on A" character, the exciton was assigned
as localized on monomer A.

\subsubsection{Comparison}
\label{subsec:state_assign}

Figure \ref{fig:chl_xtb_pbe0_LH2} compares the exciton model, constructed with PBE0
and chl-xTB monomer data, against dimer TD-DFT data with CAM-B3LYP (i.e. the same
domain as figure \ref{fig:camb3lyp_excitons}). Whilst the correlation is not as good
as was found with CAM-B3LYP monomer data, there is still a clear relationship that
supports the hypothesis that using the exciton model with lower level methods can
still reproduce qualitative responses found from higher level data.

The systematic shift observed in both is equivalent to the shift found in monomer
transitions in the previous chapter. The standard deviation of the errors was also
found to be similar to the values reported in the last chapter.
The two possible leading causes of error would be the exciton model or the theory 
used for monomer properties. From the CAM-B3LYP exciton comparison in figure \ref{fig:camb3lyp_excitons},
it was concluded that the correlation of close proximity of monomers, related to 
high coupling values, to the overestimation of transition energy implied that the 
exciton model was the leading cause of error. Hence it is implied that where there
is large separation and low coupling between chromophores, the exciton model would 
have little effect and so any error here would be due to the monomer model used. 
The origin of error being from the monomer method then explains the behavior seen
in the PBE0 and chl-xTB exciton scatters - the error in the low coupling regions
is due to the monomer theory rather than the exciton theory. When there is large
coupling, the error in monomer properties is compounded with error in the exciton
model. It can be seen that for high coupling values there is either a cancellation
of errors, with some points being firmly in the middle of the pack, or addition 
of errors, giving the outliers.

The conclusion from these benchmarks is slightly different to the earlier same-functional
benchmarking. Whereas in the earlier benchmark, improvements in the exciton model
would be expected to give better transition energies, here improvements in the underlying
monomer methods would give the greatest improvement. This is a similar conclusion
to the choice of electronic theory against choice of response theory discussed in
chapter \ref{chap:dscf}. However as the chl-xTB method behaves similarly to PBE0,
it would be hard to justify changes other than changing the functional for the training 
set data. The choice to do this would be almost arbitrary - a PBE0 exciton model has
been shown to work well in past investigations, and so even though there is error
to CAM-B3LYP data it is still good enough for investigations on light harvesting 
complexes.

\begin{figure}
    \centering
    \includegraphics[scale=0.6]{../../Year_2/DimerModel/chl_xtb_pbe0_CAMB3LYP_tddft_coupling.png}
    \caption{Transition energies of the exciton states calculated using PBE0 TD-DFT
    monomer data (left) and chl-xTB monomer data (right) against CAM-B3LYP full
    dimer TD-DFT transition energies, coloured by the coupling values.}
    \label{fig:chl_xtb_pbe0_LH2}
\end{figure}

\subsubsection{chl-xTB Scans}
\label{subsubsec:chl_xtb_scans}

Similar to the CAM-B3LYP scans above, the profile of excited dimer states was a useful
test in detailing the accuracy of chl-xTB excitons. For the same reasons as the
LH2 dimers, this benchmarking could not be done against chl-xTB or PBE0 dimer properties,
and so was compared to CAM-B3LYP again. Similar qualitative reproductions of the
excited states were found, although interestingly with slightly better separations
of the two excited states than before. Apart from differences in the magnitudes 
of the monomer charges, it is hard to attribute this improved accuracy to anything
notable.

\begin{figure}
    \centering
    \includegraphics[scale=0.5]{../../Year_2/DimerModel/Scans/rotation_along_Qz.png}
    \caption{Dimer excited state energies predicted by full dimer TD-DFT (left)
    and the exciton model (right) constructed from chl-xTB monomer data for chlorophyll 
    dimer geometries with one monomer rotated around the $Q_z$ axis.}
    \label{fig:chl_xtb_rot_Qz}
\end{figure}

\begin{figure}
    \centering
    \includegraphics[scale=0.5]{../../Year_2/DimerModel/Scans/rotation_along_Qy.png}
    \caption{Dimer excited state energies predicted by full dimer TD-DFT (left)
    and the exciton model (right) constructed from chl-xTB monomer data for chlorophyll 
    dimer geometries with one monomer rotated around the $Q_y$ axis.}
    \label{fig:chl_xtb_rot_Qy}
\end{figure}

\begin{figure}
    \centering
    \includegraphics[scale=0.5]{../../Year_2/DimerModel/Scans/rotation_along_Qx.png}
    \caption{Dimer excited state energies predicted by full dimer TD-DFT (left)
    and the exciton model (right) constructed from chl-xTB monomer data for chlorophyll 
    dimer geometries with one monomer rotated around the $Q_x$ axis.}
    \label{fig:chl_xtb_rot_Qx}
\end{figure}

In this case where there is little random error from geometry variation, the exciton
model rather than chl-xTB properties is most likely the leading cause of error. 
This does not contradict the LH2 dimer conclusion as these setups test two different
things. For LH2 dimers, the exciton model had to predict correct variations in transition
properties for a set of intra-chromophore geometries, similar to the study in the 
previous chapter. The range of inter-chromophore conformations is fairly narrow,
especially for nearest neighbours. In contrast, in this study the intra-chromophore 
geometry is constant, and the inter-chromophore geometry is changed. This removes
the compound issue between chl-xTB and CAM-B3LYP monomer properties being different
for intra-chromophore geometries, and just focuses on whether a non-varying chl-xTB
transition can supply decent properties for a varying dimer system. One observation
that showcased this difference was that the coupling values predicted by earlier
versions of chl-xTB were much larger than CAM-B3LYP. Observing that the transition
dipoles were also much larger in these earlier versions of chl-xTb, an artifact 
also observed in \dscf and the eigenvalue difference methods of the previous chapter,
the transition density matrix scaling factor was then included. Without removing
the variation due to intra-chromophore geometry this effect may not have been clear.

As the exciton framework has been shown to break down at small separations, it was
postulated that there should be a good correlation between proximity of electron
density and errors in predicting excited states. The places where the two chlorophylls
would have a closest approach would be in the functional groups attached to the 
porphyrin ring. The centres of functional groups which have the most amount of transition
density are shown in figure \ref{fig:dimer_system} with orange spheres. These atoms
were used to calculate the maximum reciprocal distance between the chlorophylls, 
which would correspond to the leading terms in the exciton coupling elements.

As the energy value of the excited states is very different between CAM-B3LYP and
chl-xTB, a better comparison would be in the error of transition energies from the
ground exciton state to the excited states, taken as the difference between the 
states. The profile of the sums of the absolute values of the error in these two
transitions along the angle of rotation, alongside the reciprocal distance of the
closest functional groups on the porphyrin ring is shown in figures \ref{fig:error_along_qz}, 
\ref{fig:error_along_qx} and \ref{fig:error_along_qy}. It can be clearly seen that
peaks in error in transition energy correspond with peaks in the proximity of the
functional groups. It would be expected that a better approximation of electron 
interaction at close separations, such as the MNOK integrals used in the previous
chapter, would improve the behaviour in these regions, but these issues are not 
in the scope of this work. As stated earlier, these separations are artificially 
small to exacerbate differences for clearer observation. Additionally the errors 
in transition energies are on the order of miliHartree, well within a reasonable
range of other functionals.

\begin{figure}
    \centering
    \includegraphics[scale=0.6]{../../Year_2/DimerModel/Scans/error_dist_along_Qz.png}
    \caption{Profiles of the sum of errors between chl-xTB exciton transition energies
    and full CAM-B3LYP TD-DFT (blue), alongside the greatest reciprocal distance
    of the functional group sites at each angle (green) for rotations along the
    $Q_z$ axis.}
    \label{fig:error_along_qz}
\end{figure}

\begin{figure}
    \centering
    \includegraphics[scale=0.6]{../../Year_2/DimerModel/Scans/error_dist_along_Qy.png}
    \caption{Profiles of the sum of errors between chl-xTB exciton transition energies
    and full CAM-B3LYP TD-DFT (blue), alongside the greatest reciprocal distance
    of the functional group sites at each angle (green) for rotations along the
    $Q_y$ axis.}
    \label{fig:error_along_qy}
\end{figure}

\begin{figure}
    \centering
    \includegraphics[scale=0.6]{../../Year_2/DimerModel/Scans/error_dist_along_Qx.png}
    \caption{Profiles of the sum of errors between chl-xTB exciton transition energies
    and full CAM-B3LYP TD-DFT (blue), alongside the greatest reciprocal distance
    of the functional group sites at each angle (green) for rotations along the
    $Q_x$ axis.}
    \label{fig:error_along_qx}
\end{figure}

\afterpartskip
\section{Concentration Quenching}
\label{sec:concentration_quenching}

An ideal use case for the chl-xTB exciton dimer model was found to be reconstructing
potential energy surfaces (PESs) of excited states, where a systematic scan of a
reaction coordinate would not be possible. In these cases the PESs can be calculated
from statistical methods, but these require a large volume of data to be accurate.
This volume prohibits the use of full dimer calculations, as well as high level
monomer TD-DFT that could be used to construct exciton models.

This is the case for calculating rate constants of conversion between the excited 
state (ES) and charge separation (CS) state of a chlorophyll dimer, which is important
for understanding the quenching mechanism of excited chlorophyll dimer states. This
rate constant is dependent on the free energy change $\Delta A$ and reorganization
energy $\lambda$ of transitioning from the ES to CS state, and is given by

\begin{equation}
    k_{\text{CS}} = \frac{2\pi}{\hbar} \left\lvert H_{AB} \right\rvert^2 \frac{1}{\sqrt{4 \pi \lambda k_B T}}\text{exp}\left(\frac{-\left(\lambda + \Delta A\right)^2}{4 \lambda k_B T}\right)
\end{equation}
%
where $H_{AB}$ is the coupling value between the ES and CS states, $k_B$ is the 
usual Boltzmann constant and $T$ is the temperature. $\Delta A$ and $\lambda$ are
calculated from the minima of the ES and CS PESs. These surfaces however are 
dependent the both the internal geometry of the dimers and the solvent system around
it, and it is not possible to reduce this space to simple reaction coordinates to 
perform systematic scans. Instead, the PES can be reconstructed using the distribution 
of energy values that it defines, calculated from a large volume of chlorophyll 
dimer geometries.

Calculating PESs was done for a series of chlorophyll dimers, using the difference
between the ES and CS energies as the reaction coordinate (and the space over which
the $\Delta E$ are  distributed), as this would include all of the internal chlorophyll
coordinates and solvent coordinates. The PESs was taken as a quadratic function
of the excitation energy $\Delta E$

\begin{equation}
    V\left(\Delta E\right) = \frac{kT}{2} \left( \frac{\Delta E - \mu}{\sigma}\right)^2
\end{equation}
%
where $\mu$ and $\sigma$ are the mean and standard deviation of the excitation energies.
The Boltzmann distribution from this PES is a normal distribution

\begin{equation}
    P\left(\Delta E\right) = \text{exp} \left(-\frac{1}{2}\left(\frac{\Delta E - \mu}{\sigma}\right)^2\right)
\end{equation}
%
and so values for $\mu$ and $\sigma$ also define the PES. 

The rate $k_{ES \rightarrow CS}$ was calculated for a series of chlorophyll dimers 
at a constant separation, solvated in diethyl ether. The rate was also calculated 
for a pair of chlorophylls embedded in the LH2 protein. MD simulations were used 
to generate  the thermally distributed geometries of the dimer systems. The diethyl
ether systems were prepared using \code{Packmol}, solvating chlorophyll dimers into 
100 \AA{} diethyl ether boxes, with force-field parameters taken from OpenForceField 
version 1.3.0. For the solvent systems, an additional constraint was used to keep 
the dimer at a constant separation in order to properly inspect the effect of chlorophyll
distance on the charge separation rate. These constraints were set to 8, 10, 12,
14 and 20 \AA{}. Each system was equilibrated for 10 ps at 300 K with a
Langevin Integrator, with a production simulation of 500 ps performed afterwards. 
Geometries were taken every 1 ps for 500 total geometries. The LH2 geometries were
calculated using the same MD method but using the LH2 structure and force-field developed
by Ramos \emph{et al.} \cite{Mennucci2019}.

Dimer excited states were calculated for all of the geometries, using the Frenkel
exciton Hamiltonian and chl-xTB. The lowest energy exciton states were taken for 
the distribution to reconstruct the PES. To account for environmental effects, the
solvent and other chlorophyll(s) were included as point charges (with charge values 
taken from the force-fields used) to polarize the chl-xTB calculations. Fitting normal
distribution values for $\mu$ and $\sigma$ was done with the \code{scipy.stats} 
module. Good fits were achieved with this method, justifying the assumption that 
the PESs follow a quadratic function.

The coupling value $H_{AB} = \braket{ES|\hat{H}|CS}$ was calculated for a set of
dimers using Fragment Orbital DFT with a PBE/3-21++G level of theory. FODFT will
become less accurate as the distances between fragments get larger due to basis
set truncation errors, so instead of calculating separate values for all separations
the 10 $\AA{}$ case was used to better approximate further separations using an
exponential decay. Additionally, truncated chlorophylls with the phytol tail removed
were used to make the FODFT calculations feasible.

The rate constant calculated for the $10\AA{}$ angstrom dimer case was 0.0004 $\text{ns}^{-1}$.
The fluorescence rate $k_f$ is given as 0.2 $\text{ns}^{-1}$, implying that charge
separation is not a competitive pathway for quenching the excited state. However
this value assumes a full relaxation of the geometries to a new equilibrium. Taking
values for $\Delta A$ and $\lambda$ from the reaction coordinate at initial photoexcitation 
(i.e. where the ground state dimer distribution would sit) gives a more competitive 
rate constant of 0.01 $\text{ns}^{-1}$. Other considerations also suggest the rate
constant calculated with this method will be an underestimate, making it even more competitive.

The rate constant also decreases with the distance between the chlorophyll dimers, as can
be seen in the $\lambda$ and $\Delta A$ values for 8 $\AA{}$, 10 $\AA{}$, 12 $\AA{}$,
14 $\AA{}$ and 20 $\AA{}$ shown in figure \ref{fig:ct_pes_distance}. As the systems
get more separated both $\Delta A$ and $\lambda$ get greater, exponentially reducing 
the rate constant. This is contrary to the previously held assumption that there 
is a critical distance where this rate drops to zero \cite{Beddard1976}, although 
the exponential drop off in rate will be sharp. Additionally, the coupling value 
$H_{AB}$ will also decrease with distance. This decrease is due to the dependence 
on overlap between the donor and acceptor orbitals in the electron transfer, which 
exponentially decreases with distance.

\begin{figure}
    \centering
    \includegraphics[scale=0.4]{../../Year_3/CT_figures/PES_by_step.png}
    \caption{Potential energy surfaces of the charge separated and vertical excitation
    states of chlorophyll dimer systems, using the difference in state energies 
    as the coordinate, for a series of separations. Free energy changes ($\Delta A$)
    and reorganization energies ($\lambda$) are shown for each separation.}
    \label{fig:ct_pes_distance}
\end{figure}

Comparing the constrained systems to the LH2 system reveals how the protein structure
biases the rate constant to be even lower. The average separation of chlorophylls
in the LH2 system is around 9 $\AA{}$, which is in the region where transition to
the CS state is most competitive. This is in contrast to the "better designed" FMO
complex, where the average separation of 12 $\AA{}$ would drastically reduce the
quenching pathway. However, as can been seen in figure \ref{fig:ct_pes_lh2}, the
free energy change is much higher for LH2 than for even the 20 $\AA{}$ case, attributed
to the destabilization of the CS state. Additionally, the crossing point for the
electron transfer is to the right of the CS state minima, implying that any transition
from ES to CS would quickly reverse. This biasing of PESs illustrates how the LH2
protein actively reduces the rate constant for quenching which would otherwise be 
more present at such close separations.

\begin{figure}
    \centering
    \includegraphics[scale=0.5]{../../Year_3/CT_figures/LH2_PES.png}
    \caption{Potential energy surfaces of the charge separated and vertical excitation
    states of LH2 chlorophyll dimer systems, using the same coordinate as figure
    \ref{fig:ct_pes_distance}. The large free energy change and movement of the 
    charge separation state PES minimum to the left-hand side of the vertical excitation 
    PES imply the LH2 protein environment inhibits transfer to the charge separated
    state.}
    \label{fig:ct_pes_lh2}
\end{figure}

\section{Conclusions}
\label{sec:exciton_concs}

It has been shown that a Frenkel exciton Hamiltonian constructed using chl-xTB can
give reliable transition properties for aggregate chlorophyll systems. Whilst some
issues were found in benchmarking against relevant PBE0 data, comparisons to CAM-B3LYP 
were still possible and showcase the accuracy of the chl-xTB exciton method. The
quantitative accuracy of transition energies for LH2 dimers is on the order expected
from the training data. As chl-xTB and PBE0 based excitons predict CAM-B3LYP data
with similar levels of accuracy, it is not expected that improvements to chl-xTB
optimization methods would have appreciable effect in this specific test. Instead
the qualitative trends in excited state energies for artificial dimer geometries
imply that chl-xTB excitons can be expected to predict variations due to geometry 
changes well, and this is the main focus of this work.

The benefits of the efficiently calculating chl-xTB excitons is illustrated in 
predicting the rate constants of transitions between excited states in chlorophyll
dimers. This study is used to construct important arguments that explain quenching
behavior in chlorophyll dimers and the effect of the LH2 protein scaffold. This work
shows how the chl-xTB exciton method can be used for light harvesting studies that 
may not be possible with other response methods. Similar to the arguments in section
\ref{sec:chl_conclusions}, it is the ability to bootstrap high level data that is
the main benefit of this kind of method.

Some of the errors in the chl-xTB exciton model may be improved by using other interaction
methods beyond the point charge interaction. This could be done using MNOK operators
similar to the work done in the previous chapter, although with different parameters.

It can be seen how chl-xTB design choices have created a method that can be readily 
applied to light harvesting systems. For this work, chl-xTB had to be shown to be 
able to predict transition properties for a range of LH2 dimer conformations, but 
no explicit consideration is given to other systems. The qualitative trends found 
in the dimer geometry scans imply that this method would work for other light harvesting
complexes. However one outcome of this study could be that the training data range
would need to be extended, similar to the conclusion of the previous chapter. 
