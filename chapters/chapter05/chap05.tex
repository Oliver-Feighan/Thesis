%
% File: chap05.tex
% Author: Oliver J. H. Feighan
% Description: LHII Spectral Density
%
\let\textcircled=\pgftextcircled
\chapter{Light Harvesting Complexes}
\label{chap:LH2}

\initial {T}his chapter reports on the application of the chl-xTB exciton framework
in calculating the spectral density of the exciton states of LH2. By comparing the
this spectral density to spectra from other systems, as well as properties such
as the Huang-Rhys factors for chlorophyll normal modes and chlorophyll separation
force constants, it is argued that the protein scaffold has little effect on exciton
states outside of affecting individual chromophore geometries.

%=======
\section{LH2}
\label{sec:LH2}

In the introduction it was stated that spectral densities of individual sites can
be calculated with an explicit electronic treatment for each frame of a time series.

For larger systems, such as the 27-site exciton system in LH2, the increased volume
of calculation necessary mean that approximations have to be made that averages
over intra-site properties, reducing the level of deatil. Due to the efficiency
of the chl-xTB exciton method, it was possible to calculate the transition properties
for each site for each frame of a time series, and construct and diagonalise the
Frenkel exciton Hamiltonian.

This chapter tries to answer two questions about environmental frequency coupling
to the exciton states. The first is whether the is the presence of low frequency
modes in the spectral density, which would be indicative of a large scale motion
of the protein scaffold. It can be seen on AFM images of the LH2 protein that there
are distinct deformations of the circular symmetry, which would indicate a ring-
breathing mode in the chlorophyll site positions. It is not fully understood whether
this would have an effect on the exciton states. Second would be by how much is 
the exciton state spectral density different to that of the sites. This would be
an indicator of whether the protein structure promotes some special interaction 
between sites, or if the protein only affects intra-chromophore geometry. Whilst
these two questions only probe vague details of the protein structure, there are
other conclusions possible that would be useful for further studies. The amount 
of intra- verses inter-site difference would inform how much detail is needed for
future studies on LH2. It would also help in the design of artificial systems, which
is discussed later.

\subsection{Spectral Density Method}
\label{subsec:spec_dens}

The spectral density describes how the oscillations in the surrounding environment 
can effect a property. Whilst it is not limited to transition energies, it is usually
used when looking at light harvesting complexes to assign functions to the protein
structure. 

There are a several definitions of spectral density. This chapter uses the definitions
laid out by Mallus \emph{et al.}. This describes the spectral density
as the real part of the Fourier transform of the classical autocorrelation function.
The autocorrelation here is the deviation of transition energies from the mean correlated
with itself

\begin{equation}
    C\left(t_j\right) = \frac{1}{N-j} \sum^{N-j}_{k=1} \Delta E \left(t_j + t_k\right) \left(t_k\right)
\end{equation}
%
where $j,k$ are indices in the time series, $N$ the total number of frames in the
time series, and $\Delta E$ the deviation of transition energy to the mean. The
spectral density is then given by

\begin{equation}
    J\left(\omega\right) = \frac{\beta \omega}{\pi} \int^\infty_0 dt \, C\left(t\right) \cos \left(\omega t\right)
\end{equation}
%
where $\omega$ are the frequencies and $\beta$ is the inverse temperature $\frac{1}{k_B T}$.

In practice, the spectral density is usually calculated from a discrete series of 
a property calculated from the frames of an MD simulation. Hence the autocorrelation
and Fourier transforms are also discrete. This places a limit on the frequencies 
that the spectral density can cover, proportional to both the length of the MD
for the low frequency limit and the interval between frames for the high frequency
limit. The fastest vibrations would be due to hydrogen atoms with a timescale of 
about, giving the timestep interval to be about , which is in line with other similar
studies. The lower frequency limit is an open ended limit, dependent on how many
frames it is possible to calculate the exciton model for. This is discussed in more
detail later.

\section{Calculating LH2 Excitons}
\label{sec:MD}

The time series of exciton states was calculated from a series of structures of 
LH2 generated by molecular dynamics. These simulations were done with the OpenMM 
package. The forcefield and geometry was the same as used by \emph{et al.}. For 
each simulation, the system was equilibriated for 60 ps. The production workflow 
length was 300 ps, with frames taken every 2 fs similar to the work done by \emph{et al.}.
A Langevin integrator set at 300 K was used with a timestep of 2 fs. Non-bonded
interactions were treated with a PME method.

Exciton states were calculated with the same method as discussed in the previous 
chapter, extending beyond the dimer system to include all chlorophyll sites. The
elements in exciton Hamiltonian were calculated using chl-xTB properties. It was
necessary to implement a highly parallelised version of the code to calculate chl-xTB
properties with good scaling. It was found that using a parent program to partition
each chlorophyll site to run in serial on a single core had the best performance,
with the data then being collected back in to calculate the exciton Hamiltonian.
As the machines available had around 20 cores, the scaling of each exciton run was
similar to a single chl-xTB calculation.

The main bottleneck in calculating these states was the volume of chl-xTB properties
needed, as it was found construction and diagonlisation of the exciton Hamiltonian
was negligable. The 300 ps MD generated 150,000 individual frames, requiring 4,050,000
individual chl-xTB runs. The time for each chl-xTB calculation is $\approx 1$ second,
so this presented about 46 days of CPU walltime. This was significantly reduced 
by using the highly parallelised program. Whilst have this level of detail is necessary
for spectral density investigations, other properties of the LH2 system that can
be obtained with more statistical methods may still be more efficient use of resources.

The coupling values obtained from the chl-xTB exciton framework match those of previous
studies well.

\begin{figure}
    \centering
    \includegraphics[]{../../Year_3/SpectralDensity/images/coupling_all.png}
\end{figure}

\subsection{Screening and Embedding}
\label{subsec:screening}

Recent studies that calculate exciton models for light harvesting systems employ
screening factors as well as point charge interactions to compensate for some of
the embedding effects of the protein scaffold. These were also tested, but ultimately
not used for the spectral density in favour of a vacuum model.

Point charge embedding was used in the previous chapter for LH2 dimers, with the
non-periodic implementation utilising the open source \code{helPME} library. The
CPU time of performing non-periodic point charge embedding was profiled using the
LH2 MD frames. It was found that due to implementation problems with re-using splines
and grid positions, the time for each chl-xTB calculation was increased from $\approx 1$
second to  $\approx 40$ seconds. Additionally the PME routines required multiple 
cores in order to get the best possible performanc, which competed with the site 
parallelisation of serial chl-xTB runs. As improving PME implementations are out
of the scope of this work (beyond what was already done to achieve the work in the
last chapter), and as the embedding only marginally changes the \Qy transition,
this embedding scheme was not applied to the exciton system used in this chapter.

A screening term was also implemented. This followed the same form as that reported
by \emph{et al.}, where any point-charge interaction has a prefactor screening term
that is dependent on the distance between point charges. For example for a coupling
term between two exciton states, a single element in the sum of chromophore-chromophore
interactions would be given by

\begin{equation}
    V_{mn} = \frac{f}{4\pi\epsilon_0} \sum_{i,j} \frac{q^T_i q^T_j}{R_{ij}}
\end{equation}

where the definition of variables are the same for equation . The scaling factor 
$f$ is given by

\begin{equation}
    f = A \text{exp}\left(-B R_{ij}\right) + f_0
\end{equation}

where $A$, $B$ and $f_0$ are constants. 

With this screening factor implemented, it was was found that the spectral density
was not appreciably different, and the only effect  was to decouple the B800 and
B850 exciton states. This is best demonstrated in the simulated absorption spectra 
for LH2 with and without the screening factor, as well as the breakdown of exciton
states into the density contributions from each site.

\subsubsection{Absorption Spectra}
\label{subsubsec:abs_spec}

Absorption spectra of LH2 is usually calculated by calculating the intensity of 
transitions for each exciton state and plotting these against the wavelengths of
transitions to these states. 

The intensity of transition from the ground exciton state $\Psi_0$ to a (one) exciton
state $\Psi_k$ is given by

\begin{equation}
    I_k \varpropto E_k \left\lvert \braket{\Psi_k| \hat{\epsilon} \cdot \mathbf{\mu} | \Psi_0} \right\rvert^2
\end{equation}
%
where $E_k$ is the energy of the state $\braket{\Psi_k|H|\Psi_k}$, $\hat{\epsilon}$
is a unit vector matching the polarisation of the incident light, which to match
sunlight should follow a random distribution. As the one-exciton states are mostly
at the same energy the energy factor $E_k$ can be neglected. The overlap term can
then be expanded into the monomer basis giving

\begin{equation}
    I_k \varpropto \left\lvert \sum^m_{j=1} c_{kj} \hat{\epsilon \cdot \mathbf{\mu}_j }\right\rvert^2
\end{equation}
%
where $j,m$ are the index and total number of chlorophyll sites respectively, $c_{kj}$
is the eigenvector coefficient of state $k$ at site $j$, and $\mathbf{\mu_j}$ is 
the transition dipole moment of chlorophyll $j$ 

\begin{equation}
    \mathbf{\mu}_j = \braket{\phi_j^{\left(1\right)}|\mathbf{\mu}|\phi_j^{\left(0\right)}}
\end{equation}
%
. As this intensity is dependent on the orientation of the unit vector $\epsilon$,
the average intensity can be found by either calculating the intensities for a large
distribution of randomised unit vectors and taking the average, or by taking the
analytic spherical average. Both these methods were tested and give the same values,
as shown in fig.

The simulated LH2 spectra with and without screening factors, alongside the experimental 
spectrum, are shown in fig. It can be seen that including the screening factor does 
produce better splitting of the B800 and B850 peaks. The poor fit of the B800 peak 
to the experimental spectrum has been well discussed in the literature, and the best
fit is achieved by using a more idealised method to calculated exciton transition
energies.

\begin{figure}
    \centering
    \includegraphics[scale=0.6]{../../Year_3/SpectralDensity/images/absorption_spectra.png}
\end{figure}

\subsubsection{Site Contributions to Exciton Density}
\label{subsubsec:site_dens}

The decoupling of ring structures is also seen in the density contributions of chlorophyll
sites to overall exciton states. The eigenvector solutions contain the amount of
site character in each state, with the square of this being equal to the exciton
density on a specific site. For example, the second element in each eigenvector 
is the corresponding amount of character from an excitation on the first chlorophyll
site (the first element corresponding to the ground state, with no excitation on
any chlorophylls). The average of density values is shown in fig for both exciton
states calculated with and without the screening factor, as well as the difference
between them.

The main difference is the change in the amount of density shared between states
localised on B800 and B850 sites. The plot of the density difference shows how density
is localised more on B800 sites for exciton states calculated with the screening
factor. This is also true for the B850 sites. However it does not reduce the delocalisation
of intra-ring sites - the amount of density shared between B850 sites with B850 
sites stays effectively the same. Even without the screening factor, the contribution
of B800 and B850 sites to the same state was relatively low, and so the exclusion
of the screening factor is argued to not be detrimental to the model.

Overall, whilst is was possible to include embedding effects into the chl-xTB exciton
model, it may not change the overall qualitative behaviour of the exciton states.
Calculating the chlorophyll system in a vacuum would still be a valid decision to
make. The vacuum model has been effective in the past, and captures many of the
important effects. This should also be true for the spectral density as the leading
cause of exciton state energy variation would be in effect of intra-chlorophyll 
geometry on the coupling and transition energy.

\begin{figure}
    \centering
    \includegraphics[]{../../Year_3/SpectralDensity/images/screened_density_diff.png}
\end{figure}

\section{Spectra}
\label{sec:sites_states_couplings}

\subsection{Sites}
\label{subsec:sites}

Similar to other studies, it was possible to produce a spectral density for \Qy 
transitions at individual sites. This is shown in figure , with the same scale as
the spectral density by \emph{et. al}. Similar features such as the peaks at eV
are present in both. The discrepancy between both can be explained by the different
forcefields and response methods used. It could then be expected that the features 
found in the spectral density of chl-xTB exciton states to be reasonable as the 
individual sites show the same features as those reported before in the literature.

\begin{figure}
    \centering
    \includegraphics[scale=0.6]{../../Year_3/SpectralDensity/images/specdens_literature_comparison.png}
\end{figure}

Expanding the frequency domain and averaging the spectra by the ring type clearly
shows the effect of the protein scaffold at different sites. In line with previous
studies, the region around 0.2 eV show the greatest features, and is attributed
to intra-chromophore normal modes \cite{Olbrich2010}. The difference between B800
and B850 lines also indicates different binding site environments, in line with 
other studies.

\begin{figure}
    \centering
    \includegraphics[angle=90, origin=c]{../../Year_3/SpectralDensity/images/specdens_ring_average.png}
\end{figure}

\afterpartskip
\subsection{Exciton states}
\label{subsec:states}

The exciton spectral density was calculated for the time series of the exciton transition
energies (calculated as the difference between a state and the exciton ground state)
of each state bar the ground state. The absorption probability was calculated for
each exciton states at all MD frames. It can be seen that the 2nd and 3rd lowest 
energy exciton states have the highest probability of absorption. 

\begin{figure}
    \centering
    \includegraphics[scale=0.6]{../../Year_3/SpectralDensity/images/abs_prob_by_energy.png}
\end{figure}

It was then possible to plot the spectral densities
of all states, weighted by their average absorption probability. The relative absorption 
probability for a state was found not to vary much, justifying taking the average
over a large time span. This plot is shown in fig.

\begin{figure}
    \centering
    \includegraphics[angle=90, origin=c]{../../Year_3/SpectralDensity/images/specdens_state_spectra.png}
\end{figure}

From the change in scale on the y-axis, it is obvious that the environmental effect
on exicton states is much smaller that for sites. This is explained by the lack
of correlation between chlorophyll motions, generally cancelling out any variation
and so mostly staying close to a mean value. This would reduce the fluctuation of
exciton transition energies, which in turn would decrease the magnitude of any peak 
in the spectral density.

It can also be seen that major features are common in both the exciton and site spectra.
This implies that the protein environment effect on the transition at a single chlorophyll
site is much greater than any effect on the coupling between chlorophyll sites.
Whilst not wholly surprising based on the lack of correlation between atomic positions
in chlorophylls, it is interesting to note that there is not as much constraint
on the coupling as there is for the different ring sites. Many of the exciton states
are delocalised across several states, and so it would be imagined that the coupling
that decides this delocalisation would be important to control. What is found here
is that there is little consideration for the variations in coupling outside of
variations due to inter-chlorophyll angle and distance, which stay relatively constant
throughout the MD trajectory.

The absorption probability weightings offer some more notable observations. For
the most part, exciton states with lower probabilities correspond to lower probabilites.
This does give some credit to the idea that the environment does influence some
states in order to achieve some difference properties. However as these are not
outside of the features seen in the site energies it is hard to say this is bourne
from effects on coupling. Additionally, it is clear to see a lack of B800 features 
in the state spectra. This probably has common cause with the suppression of the
B800 peak in simulated absorption spectra. The most notable phenomena is the outlying
exciton state which, across the entire spectrum domain, has a much larger density
than all other states. This would imply one exciton state in particular has much
larger fluctuations in transition energy than others. This state is probably localised
in the B850 ring by the absence of any B800 features at the highest frequency end
of the spectrum. 

There is also a lack of low frequency modes in the exciton state spectra, outside
of the range of intra-chlorophyll motions. These frequencies would correspond with
large scale movements of the protein scaffold, causing changes in inter-chlorophyll
separations or angles. This would have a pronounced effect on the coupling in the
exciton state. However the lack of these modes in the spectral density imply that
either these motions are not present in LH2, or are on a timescale that is too long
for the MD simulation. Reasons for the lack of appearance of these modes is discussed
later.

\afterpartskip
\subsection{Coupling}
\label{subcsec:coupling}

The spectral density of the coupling terms were also calculated with the same method
as the state and site transition energies. These spectra are collected in figure.

It can be seen that there are far fewer features in the coupling spectral density
than in the site and state transition spectra. Additionally there is a broad feature
at the low frequency end of the spectrum. Intuitively, the magnitude of the spectral 
density decreases as the separation of chloroyphlls increase, with anything but
nearest neighbours almost being undetectable.

Previous arguments about coupling terms have said that the change in distance is
the controlling factor. This was investigated by calculating the spectral density
of the interchromophore distance (reported without units as these aren't physically
meaningful). It can be seen that while there is some correspondence in the low frequency
region, it does not correspond to features in the high frequency region. These are
therefore attributed to intra-chromophore vibrations rather than external motions.

The absence of coupling features in the 0.05 eV to 0.15 eV region also gives credence
to the argument that intra-chromophore variations and not the coupling values are
the determining factor in state transition energy variations.

\begin{figure}
    \centering
    \includegraphics[angle=90, origin=c]{../../Year_3/SpectralDensity/images/specdens_coupling.png}
\end{figure}

\begin{figure}
    \centering
    \includegraphics[angle=90, origin=c]{../../Year_3/SpectralDensity/images/specdens_distance.png}
\end{figure}

\afterpartskip
\section{Assigning Specific Motions}
\label{sec:monomer_dimer_assign}

Reporting the results so far has been limited due to the lack of assignment of which
specific motions are coupling to the environment. This section reports on some of
the tests used to assign features in the spectral densities. This includes comparing
the LH2 spectral densities to a spectral density of monomer chlorophyll embedded
in diethyl ether, as well as a spectrum constructed from Huang-Rhys factors calculated
with chl-xTB response properties and a GFN1-xTB hessian. An explanation of the 
lack of low frequency features is also given by an estimation of the force constant
in the interchromophore separation.

\subsection{Ether system}
\label{subsec:specdens_ether}

While environmental coupling differences for chlorophyll sites just in LH2 can be
seen from the results so far, it is not clear how other environments couple to the 
\Qy transition. The spectral density of chlorophyll in diethyl ether was calculated
to give some light on this question. The two extremes for possible answers would
be that there is almost no variation in the spectral density, implying that the 
features in the spectral density are intrinsic to the chlorophyll geometry, or the
is a large variation and so there is a large berth for different environmental coupling. 

Geometries were taken from an MD simulation of a diethyl-ether embedded chlorophyll, 
using a solvent box made with the \code{packmol} program with a single chlorophyll 
molecule in a 64 $\AA{}$ box with 1054 diethyl-ether molecules. The simulation was
run with OpenMM using parameters for chlorophyll taken from the LH2 forceifield,
and parameters for diethyl ether taken from the General Amber ForceField (GAFF).
The system was equilibriated for 60 ps, with a production workflow of 300 ps run 
afterwards. Structures were taken every 2 fs. A Langevin integrator set at 300 K
was used, with a timestep of 2 fs.

The \Qy transition was calculated for every frame of the MD trajectory. The spectral
density of these transition energies was calculated with equations . A plot of this 
can be seen in fig.

It can be seen that the features in the spectral density for diethy-ether embedded
chlorophyll are similar to the features from the LH2 single chlorophyll and exciton
states. This implies that the coupling of environment to energy fluctuations is 
due to mostly intrinsic properties of the chlorophyll geometry. With this in mind,
a more detailed view of the normal modes in chlorophyll might help assignment of
spectral density features. This is the focus of the next section.

\begin{figure}
    \centering
    \includegraphics[angle=90, origin=c]{../../Year_3/SpectralDensity/images/specdens_ether.png}
\end{figure}

\subsection{Huang Rhys Factors}
\label{subsec:hrf}

In the sections discussing LH2 and diethyl-ether environmental coupling, it is argued
that features in the spectral density are mainly due to intrinsic properties of 
the chlorophyll geometry. These properties would most likely be the coupling of
internal motions of chlorophyll to the \Qy transition. A larger coupling of motion
would imply greater variation in the \Qy transition energy, which would cause larger 
features in the spectral density. The coupling of normal modes to electron transition
can be calculated with Huang-Rhys factors, defined by the difference between minima
in the excited and ground state energy surfaces along normal mode coordinates. The
proposed idea was that modes with high Huang-Rhys factors may correspond to features
in the spectral density, and this correspondence might be observed by comparing a
spectrum constructed from Huang-Rhys factors. This would require calculating these
factors for all normal modes.

The normal modes used to calculate Huang-Rhys factors were calculated with a hessian
calculation on an optimised single chlorophyll structure. In trying to obtain an
optimised geometry for chlorophyll, it was found that rotations in the phytol tail 
were too unconstrained for a converged geometry to be achieved. Instead a truncated
chlorophyll with a hydrogen atom replacing the phytol group was used again. The 
optimised geometry was then used to calculate normal modes with GFN1-xTB.

A scan of excitation energies were then made for each normal mode, with the chlorophyll
atoms being displaced along the vectors derived from the hessian of the optimised
chlorophyll structure. The coordinate of the scan was defined as 

\begin{equation}
    q_i = \sqrt{\frac{\omega_i}{\hbar}} x^m_i
\end{equation}
%
where $\omega_i$ is the angular frequency of the normal mode $i$ and $x^m_i$ is
the displacement vector in mass weighted coordinates. The \Qy transition energy 
was calculated for a series of structures with successive values of $q_i$. Fits 
of the ground state and excited state energies were made with quadratic functions,
from which it was possible to make estimates of the $q$ value for a minimum ground
state energy ($q_{\text{ground}}$) and excited state energy ($q_{\text{excited}}$).

From these it was possible to calculate the Huang-Rhys factors as

\begin{equation}
    d = \frac{\left(q_{\text{excited}} - q_{\text{excited}}\right)^2}{2}
\end{equation}
%
. These Huang-Rhys factors were then used to construct a spectral density, using
their absolute value for amplitude and the frequency of the corresponding normal
mode as position. A plot of this spectra, against the single chlorophyll spectra,
is shown in fig.

\begin{figure}
    \centering
    \includegraphics[angle=90, origin=c]{../../Year_3/SpectralDensity/images/hrf_spectrum.png}
\end{figure}

\begin{figure}
    \centering
    \includegraphics[]{../../Year_3/SpectralDensity/images/all_vibrations.png}
\end{figure}

At the low frequency region it is fairly clear that these modes are suppressed in
LH2 and ether. It could be argued that this is due to the high force constants, 
and so would not be expected to be observable motions. Without further investigation
it is hard to say for what reasons these normal modes are not present but it is 
clear that they do not explain any features in the spectral densities.

The high frequency normal modes correspondence to spectral density features is less
clear. Whilst two peaks in the Huang-Rhys spectrum are similar to the full spectral
density, these may just be coincidence. It is argued that it is more likely that 
whole sections of the normal modes are shifted from the frequencies observed in
the spectral densities, due to the differences in the forcefield and GFN1-xTB methods.
Whilst full assignment of the spectral density features are outside the scope of 
this work, it can be clearly seen that with a method of producing hessians from 
forcefield parameters it would be possible to make a more compelling comparison
of Huang-Rhys factors and the spectral density.

\subsection{N Axes Deformation}
\label{subsec:N_axes_deformation}

It was posited that the deformation of the $N_A$-$N_C$ and $N_B$-$N_D$ axis might
correspond well to the previous spectral densities reported. This is due to the 
appearance of symmetry breaking modes as peaks in the Huang-Rhys spectra, implying
that many features in the spectral density might correspond to these motions. Another
spectral density of the ratio of the lengths of the two nitrogen axes was calculated
in the same fashion, and is shown in figure.

It can be seen that some features in the 0.05 eV to 0.15 eV frequency range are 
common, and the prominent feature of the site/state transition spectra at 0.17 eV
is also present. The main feature in the N axes deformation however is at the low
frequency range (although not at a position to explain any coupling spectra features), 
and this does not appear to have any correspondence to the site and state spectra.
Many high frequency features are also missing, implying that other vibrations are
responsible for site/state transition variations.

\begin{figure}
    \centering
    \includegraphics[angle=90, origin=c]{../../Year_3/SpectralDensity/images/specdens_N_axes.png}
\end{figure}

\section{conclusions}
\label{sec:specdens_concs}

