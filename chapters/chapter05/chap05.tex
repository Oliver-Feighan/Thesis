%
% File: chap05.tex
% Author: Oliver J. H. Feighan
% Description: LHII Spectral Density
%
\let\textcircled=\pgftextcircled
\chapter{Light Harvesting Complexes}
\label{chap:LH2}

\initial {T}his chapter reports on the application of the chl-xTB exciton framework
in calculating the spectral density of the exciton states of LH2. By comparing the
this spectral density to spectra from other systems, as well as properties such
as the Huang-Rhys factors for chlorophyll normal modes and chlorophyll separation
force constants, it is argued that the protein scaffold has little effect on exciton
states outside of affecting individual chromophore geometries.

%=======
\section{LH2}
\label{sec:LH2}

In the introduction it was stated that spectral densities of individual sites can
be calculated with an explicit electronic treatment for each frame of a time series.

For larger systems, such as the 27-site exciton system in LH2, the increased volume
of calculation necessary mean that approximations have to be made that averages
over intra-site properties, reducing the level of deatil. Due to the efficiency
of the chl-xTB exciton method, it was possible to calculate the transition properties
for each site for each frame of a time series, and construct and diagonalise the
Frenkel exciton Hamiltonian.

This chapter tries to answer two questions about environmental frequency coupling
to the exciton states. The first is whether the is the presence of low frequency
modes in the spectral density, which would be indicative of a large scale motion
of the protein scaffold. It can be seen on AFM images of the LH2 protein that there
are distinct deformations of the circular symmetry, which would indicate a ring-
breathing mode in the chlorophyll site positions. It is not fully understood whether
this would have an effect on the exciton states. Second would be by how much is 
the exciton state spectral density different to that of the sites. This would be
an indicator of whether the protein structure promotes some special interaction 
between sites, or if the protein only affects intra-chromophore geometry. Whilst
these two questions only probe vague details of the protein structure, there are
other conclusions possible that would be useful for further studies. The amount 
of intra- verses inter-site difference would inform how much detail is needed for
future studies on LH2. It would also help in the design of artificial systems, which
is discussed later.

\subsection{Spectral Density Method}
\label{subsec:spec_dens}

The spectral density describes how the oscillations in the surrounding environment 
can effect a property. Whilst it is not limited to transition energies, it is usually
used when looking at light harvesting complexes to assign functions to the protein
structure. 

There are a several definitions of spectral density. This chapter uses the definitions
laid out by Mallus \emph{et al.}. This describes the spectral density
as the real part of the Fourier transform of the classical autocorrelation function.
The autocorrelation here is the deviation of transition energies from the mean correlated
with itself

\begin{equation}
    C\left(t_j\right) = \frac{1}{N-j} \sum^{N-j}_{k=1} \Delta E \left(t_j + t_k\right) \left(t_k\right)
\end{equation}
%
where $j,k$ are indices in the time series, $N$ the total number of frames in the
time series, and $\Delta E$ the deviation of transition energy to the mean. The
spectral density is then given by

\begin{equation}
    J\left(\omega\right) = \frac{\beta \omega}{\pi} \int^\infty_0 dt C\left(t\right) \cos \left(\omega t\right)
\end{equation}
%
where $\omega$ are the frequencies and $\beta$ is the inverse temperature $\frac{1}{k_B T}$.

In practice, the spectral density is usually calculated from a discrete series of 
a property calculated from the frames of an MD simulation. Hence the autocorrelation
and Fourier transforms are also discrete. This places a limit on the frequencies 
that the spectral density can cover, proportional to both the length of the MD
for the low frequency limit and the interval between frames for the high frequency
limit. The fastest vibrations would be due to hydrogen atoms with a timescale of 
about, giving the timestep interval to be about , which is in line with other similar
studies. The lower frequency limit is an open ended limit, dependent on how many
frames it is possible to calculate the exciton model for. This is discussed in more
detail later.

\section{Calculating LH2 Excitons}
\label{sec:MD}

The time series of exciton states was calculated from a series of structures of 
LH2 generated by molecular dynamics. These simulations were done with the OpenMM 
package. The forcefield and geometry was the same as used by \emph{et al.}. For 
each simulation, the system was equilibriated for 60 ps. The production workflow 
length was 300 ps, with frames taken every 2 fs similar to the work done by \emph{et al.}.
A Langevin integrator set at 300 K was used with a timestep of 2 fs. Non-bonded
interactions were treated with a PME method.

Exciton states were calculated with the same method as discussed in the previous 
chapter, extending beyond the dimer system to include all chlorophyll sites. The
elements in exciton Hamiltonian were calculated using chl-xTB properties. A highly
parallelised version of the code used to calculated chl-xTB properties was implemented
in order to achieve good scaling. As machines with 20 cores were readily accessible, 
it was found that running sites as a serial chl-xTB calculation whilst inside a 
parent program that setup and collected the chl-xTB results, gave the best performance.

The main bottleneck in calculating these states was the volume of chl-xTB properties
needed, as it was found construction and diagonlisation of the exciton Hamiltonian
was negligable. The 300 ps MD generated 150,000 individual frames, requiring 4,050,000
individual chl-xTB runs. The time for each chl-xTB calculation is $\approx 1$ second,
so this presented about 46 days of CPU walltime. This was significantly reduced 
by using the highly parallelised program. Whilst have this level of detail is necessary
for spectral density investigations, other properties of the LH2 system that can
be obtained with more statistical methods may still be more efficient use of resources.

The coupling values obtained from the chl-xTB exciton framework match those of previous
studies well.

\subsection{Screening and Embedding}
\label{subsec:screening}

Recent studies that calculate exciton models for light harvesting systems employ
screening factors as well as point charge interactions to compensate for some of
the embedding effects of the protein scaffold. These were also tested, but ultimately
not used for the spectral density in favour of a vacuum model.

Point charge embedding was used in the previous chapter for LH2 dimers, with the
non-periodic implementation utilising the open source \code{helPME} library. The
CPU time of performing non-periodic point charge embedding was profiled using the
LH2 MD frames. It was found that due to implementation problems with re-using splines
and grid positions, the time for each chl-xTB calculation was increased from $\approx 1$
second to  $\approx 40$ seconds. Additionally the PME routines required multiple 
cores in order to get the best possible performanc, which competed with the site 
parallelisation of serial chl-xTB runs. As improving PME implementations are out
of the scope of this work (beyond what was already done to achieve the work in the
last chapter), and as the embedding only marginally changes the \Qy transition,
this embedding scheme was not applied to the exciton system used in this chapter.

A screening term was also implemented. This followed the same form as that reported
by \emph{et al.}, where any point-charge interaction has a prefactor screening term
that is dependent on the distance between point charges. For example for a coupling
term between two exciton states, a single element in the sum of chromophore-chromophore
interactions would be given by

\begin{equation}
    V_{mn} = \frac{f}{4\pi\epsilon_0} \sum_{i,j} \frac{q^T_i q^T_j}{R_{ij}}
\end{equation}

where the definition of variables are the same for equation . The scaling factor 
$f$ is given by

\begin{equation}
    f = A \text{exp}\left(-B R_{ij}\right) + f_0
\end{equation}

where $A$, $B$ and $f_0$ are constants. 

With this screening factor implemented, it was was found that the spectral density
was not appreciably different, and the only effect  was to decouple the B800 and
B850 exciton states. This is best demonstrated in the simulated absorption spectra 
for LH2 with and without the screening factor, as well as the breakdown of exciton
states into the density contributions from each site.

\subsubsection{Absorption Spectra}
\label{subsubsec:abs_spec}

Absorption spectra of LH2 is usually calculated by calculating the intensity of 
transitions for each exciton state and plotting these against the wavelengths of
transitions to these states. 

The intensity of transition can be calculated with

. 

The two spectra, with and without screening, with the experimental spectrum of LH2,
is shown in fig. It can be seen that including the screening factor does produce
better splitting of the B800 and B850 peaks. The poor fit of the B800 peak to the
experimental spectrum has been well discussed in the literature, and the best fit
is achieved by using a more idealised method to calculated exciton transition energies.

\subsubsection{Site Contributions to Exciton Density}
\label{subsubsec:site_dens}

The decoupling of ring structures is also seen in the density contributions of chlorophyll
sites to overall exciton states. The eigenvector solutions contain the amount of
site character in each state, with the square of this being equal to the exciton
density on a specific site. For example, the second element in each eigenvector 
is the corresponding amount of character from an excitation on the first chlorophyll
site (the first element corresponding to the ground state, with no excitation on
any chlorophylls). The average of density values is shown in fig for both exciton
states calculated with and without the screening factor, as well as the difference
between them.


The main difference is the change in the amount of density shared between states
localised on B800 and B850 sites. 



While this comparison implies that the screening factor should be employed as it
better reproduces the experimental spectrum, there is little observable evidence
that it affects the spectral density.


Overall, whilst is was possible to include embedding effects into the chl-xTB exciton
model, exciton states were calculated by only considering the chlorophyll sites.
This model has been effective in the past, and captures many of the important effects.
This should also be true for the spectral density as the leading cause of exciton
state energy variation would be in effect of intra-chlorophyll geometry on the 
coupling and transition energy.

\section{Spectra}
\label{sec:sites_states_couplings}

\subsection{Sites}
\label{subsec:sites}

Similar to other studies, it was possible to produce a spectral density for \Qy 
transitions at individual sites. This is shown in figure , with the same scale as
the spectral density by . Similar features such as the peaks at eV are present in
both. The discrepancy between both can be explained by the different forcefields
and response methods used. It could then be expected that the features found in
the spectral density of chl-xTB exciton states to be reasonable as the individual
sites show the same features as those reported before in the literature.

\begin{figure}
    \centering
    \includegraphics[scale=0.3]{../../Year_3/SpectralDensity/images/specdens_literature_comparison.png}
\end{figure}

Expanding the frequency domain and averaging the spectra by the ring type clearly
shows the effect of the protein scaffold at different sites. The main difference
is seen between the B800 and B850 rings, at the 0.2 eV region. There is little difference
between the B850a and B850b spectra.

\begin{figure}
    \includegraphics[angle=90, origin=c, scale=0.3]{../../Year_3/SpectralDensity/images/specdens_ring_average.png}
\end{figure}

\subsection{Exciton states}
\label{subsec:states}

The exciton spectral density was calculated on the time series of the exciton transition
energies. Comparing the exciton spectra to the site spectra, it is clear to see that the major
features in the exciton spectra correspond with the major features in the site spectra.
This implies that the inter-site effects of the protein scaffold do not effect the
exciton states as much as the internal geometry does. Significantly, lower frequency
features are also not observed, implying that large scales motions of the protein
are either not present, or do not effect the exciton states.

The variation in the spectra between the different states is considered in the next
section.

\label{subsubsec:Absorption Probability}

While the exciton state spectra show a significant amount of variation, it is argued
only a couple of state transitions are actually important to consider. The transition
probability was calculated for each exciton states at all MD frames. The time series
of these probabilities is shown in fig. It can be seen that the 2nd and 3rd lowest
energy exciton states have the highest probability of absorption. 

\begin{figure}
    \includegraphics[scale=0.1, angle=90, origin=c]{../../Year_3/SpectralDensity/images/abs_prob_by_time.png}
\end{figure}

\begin{figure}
    \includegraphics[scale=0.1]{../../Year_3/SpectralDensity/images/abs_prob_by_energy.png}
\end{figure}

\section{Assigning Spectral Densities}
\label{sec:monomer_dimer_assign}

It was also investigated whether the features in the spectral density of chlorophyll
and exciton states could be assigned by comparison to other spectral densities and
normal mode analysis. The LH2 spectral densities were compared to monomer chlorophyll 
embedded in diethyl ether, Huang-Rhys factors calculated with chl-xTB and a GFN1-xTB
hessians, as well as an estimation of the force constant in the separation of interchromophore
distances in the LH2 protein. 

\subsection{Ether system}
\label{subsec:specdens_ether}

The spectral density for the \Qy transition for a single chlorophyll molecule was
calculated in a similar fashion to the LH2 sites above. 

Geometries were taken from an MD simulation of a diethyl-ether embedded chlorophyll. 
The solvent box was made with the \code{packmol} program with a single chlorophyll 
molecule in a 64 $\AA{}$ box with 1054 diethyl-ether molecules. The simulation was
run with OpenMM using parameters for chlorophyll taken from the LH2 forceifield,
and parameters for diethyl ether taken from the General Amber ForceField (GAFF).
The system was equilibriated for 60 ps, with a production workflow of 300 ps run 
afterwards. Structures were taken every 2 fs. A Langevin integrator set at 300 K
was used, with a timestep of 2 fs.

The \Qy transition was calculated for every frame of the MD trajectory. The spectral
density of these transition energies was calculated with equations . A plot of this 
can be seen in fig.

It can be seen that the features in the spectral density for diethy-ether embedded
chlorophyll are similar to the features from the LH2 single chlorophyll and exciton
states. The immediate implication is that, to a first approximation, the spectral
density of a single chlorophyll in smaller systems can replace those for larger,
more complicated systems. The other implications is that most of the effect of 
the environment on chlorophyll is exhibited though the same normal modes, most likely
normal modes which were already present in monomer chlorophyll. It is argued that
no external motions are likely to be any of the major features as they are found
in two separate 

\subsection{Huang Rhys Factors}
\label{subsec:hrf}

The coupling of normal modes to electron transition can be calculated with Huang
-Rhys factors. This was done for a single chlorophyll molecule to attempt to assign
the spectral density features to individual motions.

The normal modes used to calculate Huang-Rhys factors were calculated using a GFN1-xTB
hessian calculation of a single chlorophyll molecule. It was found that rotations
in the phytol tail were too unconstrained for a good optimised geometry to be achieved, 
and so a truncated chlorophyll with a hydrogen atom replacing the phytol group was
used again. The optimised geometry was then used to calculate normal modes.

A scan of excitation energies were then made for each normal mode, with the chlorophyll
atoms being displaced along the vectors derived from the hessian of the optimised
chlorophyll structure. For a series of coordinates calculated as 

\begin{equation}
    q_i = \sqrt{\frac{\omega_i}{\hbar}} x^m_i
\end{equation}
%
where $\omega_i$ is the angular frequency of the normal mode $i$ and $x^m_i$ is
the displacement vector in mass weighted coordinates. The \Qy transition energy 
was calculated for a series of structures with successive values of $q_i$. Fits 
of the ground state and excited state energies were made with quadratic functions,
from which it was possible to make estimates of the $q$ value for a minimum ground
state energy ($q_{\text{ground}}$) and excited state energy ($q_{\text{excited}}$).

From these it was possible to calculate the Huang-Rhys factors as

\begin{equation}
    d = \frac{\left(q_{\text{excited}} - q_{\text{excited}}\right)^2}{2}
\end{equation}
%
. These Huang-Rhys factors were then used to construct a spectral density, using
their absolute value for amplitude and the frequency of the corresponding normal
mode as position. A plot of this spectra, against the single chlorophyll spectra,
is shown in fig.

At the low frequency region it is fairly clear that these modes are suppressed in
LH2 and ether. It could be argued that this is due to the high force constants, 
and so would not be expected to be observable motions. Without any further investigation,
it is hard to say for what reasons these normal modes are not present but it is 
clear that they do not explain any features in the spectral densities.

The high frequency normal modes correspondence to spectral density features is less
clear. Whilst two peaks in the Huang-Rhys spectrum are similar to the full spectral
density, these may just be coincidence. It is argued that it is more likely that 
whole sections of the normal modes are at difference frequencies to those observed
in the spectral density.

The most probable reason for this discrepancy is that the forcefield and GFN1-xTB
method are much to different in theory to give the same frequencies for normal modes.
Without normal modes calculated with the same forcefield as the spectral density
MD, it would not be possible to assign vibrational motion to the spectral density
features.

\begin{figure}
    \includegraphics[scale=0.6]{../../Year_3/SpectralDensity/images/CO_vibration.png}
\end{figure}

\subsubsection{Estimating Force Constants for Low Frequency Modes}
\label{subsubsec:force_constants}

While there is evidence for the protein scaffold affecting chlorophyll motion and
subsequent \Qy transition and coupling, there is little evidence for any effects
outside of intra-chlorophyll range. One explanation is that the crystal structure
used for LH2 is a low energy structure, and so there would be little potential energy
in the ring-breathing modes to start any observable motions. Ring-breathing motion
could be promoted by additional force constants, but this would introduce artificial
features into the spectral density.

Another explanation is that the barrier to the chlorophyll separation mode is too 
high. 

\begin{figure}
    \includegraphics[scale=0.6]{../../Year_3/SpectralDensity/images/coupling_matrix.png}
\end{figure}

\section{conclusions}
\label{sec:specdens_concs}