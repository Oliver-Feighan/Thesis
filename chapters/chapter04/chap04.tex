%
% File: chap04.tex
% Author: Oliver J. H. Feighan
% Description: Excitons
%
\let\textcircled=\pgftextcircled
\chapter{Beyond Monomer Chlorophyll}
\label{chap:excitons}

\initial {T}his chapter investigates the applicability of chl-xTB for aggregate 
chlorophyll systems, beyond the monomers reported in the previous chapter. It is
the properties of aggregate chlorophyll that make light harvesting systems so efficient.
It is necessary to show that chl-xTB can be used to construct viable models for
these systems.

%=======
\section{Theory}
\label{sec:exciton_theory}

\subsection{Exciton States}
\label{subsec:exciton_states}
As stated in the introduction, a large majority of models of light harvesting systems
utilises a Frenkel Hamiltonian model of excitonic states, where the weak coupling
between monomers mean that a description can be constructed from properties of the 
individual sites.

To recap the exciton theory from the introduction, the exciton states $\ket{\Psi}$ can
be constructed as a Hartree product of the states of individual sites (also refered
to as monomers or chromophores)

\begin{equation}
    \ket{\Psi} = \Pi_m \ket{\phi_m}
\end{equation}
%
where $\ket{\phi_m}$ is the monomer state on sites $m$. These monomer states form
the basis function of the overall excitonic states. As the exciton is modelled to
be localised to a specific site, an exciton state with an exciton at site $i$ is
given by

\begin{equation}
    \ket{\Psi}^i = \ket{\phi_i}^\ast \Pi_{m \neq i}  \ket{\phi_m}
\end{equation}
%
where $\ket{\phi_i}^\ast$ is the excited state of monomer site $i$. States with 
more than one exciton are possible in this framework, however this is not necessary
for light harvesting systems as stated in the introduction. The Hamiltonian, including
the "ground state" where there are no excitons in the system, is given in by

\begin{equation}
    H = 
    \begin{bmatrix}
        E_0 & V_{0, \left(1, 1\right)} & \cdots & V_{0, \left(N, 1\right)} \\
        V_{0, \left(1, 1\right)} & E_{\left(1,1\right)} & \cdots & V_{\left(1,1\right) \left(N, 1\right)} \\
        \vdots & \vdots & \ddots & \vdots \\
        V_{0, \left(N, 1\right)} & V_{\left(1,1\right) \left(N, 1\right)} & \cdots & E_{\left(N, 1\right)}
    \end{bmatrix}
\end{equation}
%
where $N$ is equal to the total number of individual sites. The diagonal terms $E$
is the sum of the site energies and a point charge interaction

\begin{equation}
    E_0 = \sum_m e_m + \sum_{m \neq n, A \in m, B \in n} \frac{q^A_m q^B_n}{r_{AB}}
\end{equation}
%
where $e_m$ is the energy of site $m$ and $q^A_m$ is the charge centered on atom
$A$ in site $m$. Hamiltonian elements corresponding to a single excitation are similarly
given except with the excited state energy and charges

\begin{equation}
    E_{\left(m,1\right)} = e_m + \delta e_m + \sum_{n \neq m} e_n + \sum_{n, A \in m, B \in n} \frac{q^{\ast A}_m q^B_n}{r_{AB}} + \sum_{n,p \neq m, A \in n, B \in p} \frac{q^{A}_n q^B_p}{r_{AB}}
\end{equation}
%
where $\delta e_m$ is the excitation energy of site $m$, and charges marked $q^\ast$
are the excited state charges. The inter-site second term has been replaced here
to explicitly show that there is both the interaction of excited state point charges
with all other sites, as well as the ground-state ground-state interactions for 
all other non-excited chromophores.

The off diagonal elements are the coupling elements between all exciton states.
For coupling to the ground state, these are given by

\begin{equation}
    V_{0, \left(m,1\right)} = \sum_{n, A \in m, B \in n} \frac{q^{\text{tr},A}_m q^B_n}{r_{AB}}
\end{equation}
%
where charges marked $q^{\text{tr}}$ are transition charges. It can be seen that 
this coupling element includes an electrostatic interaction between all sites and
the excited site. This is different to coupling elements between two single exciton
states, given as

\begin{equation}
    V_{\left(m, 1\right), \left(n,1\right)} = \sum_{A \in m, B \in n} \frac{q^{\text{tr},A}_m q^{\text{tr},B}_n}{r_{AB}}
\end{equation}
%
which just involve sites $m$, $n$ which have local excitations.

This Hamiltonian for exciton states is slightly different those reported in the 
literature, as the ground state where no excitons are present is not usually included.
However, if the local excitations are sufficiently different in energy and the coupling
between ground and excited states is weak, then the eigensolution corresponding 
to the ground state would have mostly ground state character. The other excited
state eigensolutions would have very little ground state character, and so the block
matrix of just excited state contributions can be taken which would return the Hamiltonian
used in many other studies. Calculating the ground state however is necessary if
all the states for the exciton system are required, which is the case for the
spectral density investigation of the next chapter.

\subsection{Embedding}
\label{subsec:embedding}



\section{Truncated Chlorophylls}
\label{sec:trunc_chl}

A series of chlorophyll dimers were used to calculate benchmarks for an exciton
model based on chl-xTB calculations. The exciton predictions were compared to TD-DFT
calculations on the full dimer system. In order to get usable scaling with the DFT
methods, a truncated version of chlorophyll with the phytol tail removed was used.
In the large separation limit the effects of TD-DFT and chl-xTB coupling should 
match, with a greater difference in the two methods being present as the chromophores
get closer together.

For each dimer, TD-DFT data was calculated for the two monomers separately as well
as for the full dimer system. The monomer data was then used to construct the Frenkel
Hamiltonian as described above.
The benchmarked data was first calculated with PBE0/Def2-SVP, however it was found 
that an exciton model constructed from monomer data did not well match it's full 
dimer counterpart (see later section on dimer geometries from LHII). As a replacement,
the CAM-B3LYP functional was used instead. The exciton model matched the full dimer
calculation well.

\subsection{Rotation}
\label{subsec:rotation}

As stated in the introduction, many previous investigations treated inter-chromophores
coupling with a point dipole interaction:

\begin{equation}
    v_{mn} = \frac{\vec{\mu}_m \cdot \vec{\mu}_n}{r_{mn}}
\end{equation}
%
where $\vec{\mu}_m$ is the dipole on site $m$ and $r_{mn}$ is the inter-chromophore
distance. From this expression is can be seen that the two inter-chromophore coordinates
are the angle and distance in a dimer pair. These two coordinates were scanned in
order get a full picture of how the excited states can be predicted by a chl-xTB
exciton Hamiltonian. Due to the large number of calculations necessary, the phytol
tails were again removed and replaced with a hydrogen atom. All dimers were constructed
from two truncated chlorophyll molecules with the porphyrin planes parallel and 
overlapping. The axis along the magnesium atoms was used to defined a separation.
As this axis is approximately the cross product of the \Qy and $Q_x$ transition 
dipoles, this axis is labelled $Q_z$. At each separation, one of the dimers was 
rotated in increments of 2 degrees up to a full 360 degree rotation for all of the
\Qy, $Q_x$, $Q_z$ axes as an axis of rotation. For all axes, the magnesium atom
was the centre of rotation. This could not be done for the \Qy and $Q_x$ axis at
a separation lower than 13 angstrom due to the size of the porphyrin ring. 

In order to make assignment of exciton states easier, the geometries of the two 
trunctated chlorophyll monomers were altered such that there would be a distinct 
gap in their transition energies. This gap was not wider that the variation in 
transition energies found in LHII monomers. More detail on assigning transition 
energies to the correct excited state is discussed below. The plot of the two excited
state energies for both CAM-B3LYP dimer and chl-xTB exciton methods are shown in
fig., where they have been aligned to their average values to remove the systematic
shift.

\begin{figure}
    \centering
    \includegraphics[scale=0.6]{../../Year_2/DimerModel/Scans/dimer_system.png}
\end{figure}

\begin{figure}
    \centering
    \includegraphics[scale=0.6]{../../Year_2/DimerModel/Scans/rotation_along_Qz.png}
    \caption{caption}
\end{figure}

\begin{figure}
    \centering
    \includegraphics[scale=0.6]{../../Year_2/DimerModel/Scans/rotation_along_Qx.png}
    \caption{caption}
\end{figure}

\begin{figure}
    \centering
    \includegraphics[scale=0.6]{../../Year_2/DimerModel/Scans/rotation_along_Qy.png}
    \caption{caption}
\end{figure}

It can be seen that the chl-xTB exciton model can reproduce high level TD-DFT dimer
data with an excellent level of accuracy. The energies for both excited states agree
across the entire range of angles, and much of the detailed variation in the CAM-B3LYP
data is also found in the chl-xTB data.. The two areas where the agreement is worse
is at 180 degrees and 270 degrees, where the coupling is strong and weak respectively.
This is attributed to the proximity of functional groups that are found at the
corners of the porphyrin ring, which are closest at these angles. As these groups
get closer together, the difference between the TD-DFT and chl-xTB treatment would
be expected to cause transition energies to grow more distinct. However it can be
seen that even with these effects the difference is small. Additionally, the angle
of chlorophyll pairs in LHII (for B850 a,b pairs) only varies around 0 by a few 
degrees, so these effects would not be so present.

\begin{figure}
    \centering
    \includegraphics[scale=0.6]{../../Year_2/DimerModel/Scans/error_dist_along_Qz.png}
    \label{fig:errror_along_qy}
\end{figure}

\begin{figure}
    \centering
    \includegraphics[scale=0.6]{../../Year_2/DimerModel/Scans/error_dist_along_Qx.png}
    \label{fig:errror_along_qy}
\end{figure}

\begin{figure}
    \centering
    \includegraphics[scale=0.6]{../../Year_2/DimerModel/Scans/error_dist_along_Qy.png}
    \label{fig:errror_along_qy}
\end{figure}

\afterpartskip

\section{LHII Pairs}
\label{sec:LHII_excitons}

Whilst the rotations and separation scans cover a large range of the orientations
between chlorophyll pairs, the best way to test orientations that would be found
in LHII would be to take pairs from the protein system. This was done for a small 
set, taking every pair from the 27 chlorophylls from 3 frames of the earlier reported
MD, giving 1053 pairs. Every dimer pair was included to scan over B800 and B850a/b
interactions, although many of these after nearest-neighbours would have almost 
zero coupling. Similar to the coordinate scan tests above, both full dimer TD-DFT 
and exciton model transition energies were calculated.

\subsection{Assignment of States}
\label{subsec:state_assign}

A more detailed view of each exciton state was needed to properly assign the two
excited states from the full TD-DFT and exciton models. It was found that for some
dimer pairs, the models would predict different energy ordering. More explicitly, 
for the A-B dimer system, one model might predict a lower energy for a state with
the exciton localised on monomer A, whereas another might predict the other state
with the exciton on B. It should be noted that the CAM-B3LYP exciton model and full
dimer data were always consistent, and it was only when comparing to PBE0 or chl-xTB
was this effect apparent.
It was therefore necessary to find the location of the excitons in both dimer and
exciton models. For the TD-DFT dimer result, the exciton location was taken as the
molecule where the transition charge distribution centered around. This centre was
calculated by the absolute charge weighted average of the atomic position. For the
Frenkel exciton result, the location was taken as the monomer which had the most
amount of character in the eigenvector solution.

\subsection{Comparison}
\label{subsec:comparison}

\subsubsection{CAM-B3LYP}
\label{subsubsec:camb3lyp_excitons}

In the ideal case, the TD-DFT and Frenekl exciton framework should match transition
energies very well, with small differences in the small distance limit. This can
be seen in the CAM-B3LYP data in figures \ref{fig:camb3lyp_tddft_exciton_coupling}
and \ref{fig:camb3lyp_tddft_exciton_coupling} where a scatter plot of the TD-DFT
and exciton transition energies reveal a very strong correlation with little systematic
error. The coupling energies between transitions are used to colour gradient the
points in figure \ref{fig:camb3lyp_tddft_exciton_coupling}, with the distance being
used to colour points in figure \ref{fig:camb3lyp_tddft_exciton_distances}. The
better correlation of distance rather than coupling energy to the error suggests
that there are interactions present in the TD-DFT model that are missing in the
exciton framework. However, this error is small. Using CAM-B3LYP transition properties
for the exciton framework can therefore recover the TD-DFT properties very well.

\begin{figure}
    \centering
    \includegraphics[scale=0.6]{../../Year_2/DimerModel/CAMB3LYP_exciton_v_tddft_coupling.png}
    \label{fig:camb3lyp_tddft_exciton_coupling}
\end{figure}

\begin{figure}
    \centering
    \includegraphics[scale=0.6]{../../Year_2/DimerModel/CAMB3LYP_exciton_v_tddft_distance.png}
    \label{fig:camb3lyp_tddft_exciton_distances}
\end{figure}

\afterpartskip
\subsubsection{PBE0 and Charge Transfer}
\label{subsubsec:pbe0_excitons}

It would have been preferable to benchmark the chl-xTB against PBE0 due to this
being used as to construct the training data. However it was found that many of 
the TD-DFT transitions had charge transfer character between the two dimers, which
should not be present. This skewed many of the transitions to lower energy, as seen
in figure \ref{fig:camb3lyp_pbe0_distributions}.
Here the probability distribution of transition energies from the chlorophyll monomers
and dimers have been plotted. For CAM-B3LYP the dimer distribution matches the
monomer distribution well with slight differences due to exciton coupling. For PBE0
however, a clear skew to lower transition energies can be seen in the dimer distribution,
which from looking at the TD-DFT solution characters was due to charge transfer 
character. It was found that there were two effects on the PBE0 transitions. First
was the closer proximity of charge transfer excitions to vertical excitations in
the TD-DFT solutions, which was not present in the CAM-B3LYP data. These were easier
to assign and remove, as although the transition energies were closer to the vertical
excitations the characters were sufficiently different to allow clear assignment.
The second was the mixing of charge transfer and vertical excitation into transitions,
which made assignment more difficult. As the first effect could be removed due to
their easier assigment, the second is the cause of the continuous skew seen in the
distributions.

\begin{figure}
    \centering
    \includegraphics[scale=0.6, angle=90, origin=c]{../../Year_2/DimerModel/camb3lyp_pbe0_distributions.png}
    \label{fig:camb3lyp_pbe0_distributions}
\end{figure}

\afterpartskip

Due to this charge transfer character, the correlations of both PBE0 exciton to
PBE0 TD-DFT transitions and PBE0 TD-DFT to CAM-B3LYP TD-DFT transitions were found
to be very poor. This can be seen in figures \ref{fig:pbe0_exciton_tddft} and \ref{fig:pbe0_tddft_camb3lyp_tddft}.
It can be seen that as the separation in dimers gets smaller (a larger reciprocal
distance), the error to the reference data gets worse. This is attributed to the
proximity of the monomers promoting the charge transfer effects. The plot of PBE0
transitions against CAM-B3LYP transitions in figure \ref{fig:pbe0_tddft_camb3lyp_tddft}
show how the PBE0 TD-DFT transitions do lead to poor correlatin between PBE0 exciton
and TD-DFT data, and it is not due to any inaccuracy in using the PBE0 monomer data
for the exciton framework.

\begin{figure}
    \centering
    \includegraphics[scale=0.6]{../../Year_2/DimerModel/PBE0_exciton_v_tddft_coupling.png}
    \label{fig:pbe0_exciton_tddft}
\end{figure}

\begin{figure}
    \centering
    \includegraphics[scale=0.6]{../../Year_2/DimerModel/PBE0_tddft_v_CAMB3LYP_tddft_coupling.png}
    \label{fig:pbe0_tddft_camb3lyp_tddft}
\end{figure}

This is more clearly shown in plotting the PBE0 exciton transitions against CAM-B3LYP
TD-DFT data, as shown in figure \ref{fig:pbe0_exciton_camb3lyp_tddft}. Here there 
is a clear correlation from the exciton model. The leading cause of error in this
comparison is attributed to the difference in the two functionals rather than the
exciton framework itself, due to the high correlation seen in figure \ref{fig:camb3lyp_tddft_exciton_coupling}
as well as the RMSE being comparable to that between PBE0 and CAM-B3LYP in the reference 
data of the previous chapter. Additionally, the lack of correlation between coupling
energy and transition energy error also implies the exciton framework is not the
leading cause of this error. The four outliers, which all do have a high coupling
value, can be explained due to this effect, where the two leading causes of error
have compounded.

\begin{figure}
    \centering
    \includegraphics[scale=0.6]{../../Year_2/DimerModel/PBE0_exciton_v_CAMB3LYP_tddft_coupling.png}
    \label{fig:pbe0_exciton_camb3lyp_tddft}
\end{figure}

\afterpartskip
\subsubsection{chl-xTB}
\label{subsubsec:chl_xtb_excitons}

With all of these benchmarks in mind, it can be seen that using the chl-xTB data
to construct the Frenkel Hamiltonian performs about as well as can be expected.
Scatter plots of the chl-xTB exciton transitions against PBE0 excitons and CAM-B3LYP
TD-DFT transitions are shown in figures \label{fig:chl_xtb_PBE0_exciton_coupling}
and \label{fig:chl_xtb_camb3lyp_tddft_coupling} respectively. Both show decent correlation
from the exciton transitions, with the same 4 outlier transitions as seen in the
previous scatter plot. The variance to the best fit is again similar to the RMSE
from the parameter optimisation, and there is little correlation between coupling
and error, suggesting that it is the chl-xTB response theory and not the exciton
framework that is the leading cause of error.

\begin{figure}
    \centering
    \includegraphics[scale=0.6]{../../Year_2/DimerModel/chl_xtb_exciton_v_PBE0_exciton_coupling.png}
    \label{fig:chl_xtb_PBE0_exciton_coupling}
\end{figure}

\begin{figure}
    \centering
    \includegraphics[scale=0.6]{../../Year_2/DimerModel/chl_xtb_exciton_v_CAMB3LYP_tddft_coupling.png}
    \label{fig:chl_xtb_camb3lyp_tddft_coupling}
\end{figure}


\afterpartskip
\section{Conclusions}