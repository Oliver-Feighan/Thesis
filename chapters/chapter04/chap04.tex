%
% File: chap04.tex
% Author: Oliver J. H. Feighan
% Description: Excitons
%
\let\textcircled=\pgftextcircled
\chapter{Beyond Monomer Chlorophyll}
\label{chap:excitons}

\initial {T}his chapter investigates the applicability of chl-xTB for aggregate 
chlorophyll systems, beyond the monomers reported in the previous chapter. It is
the properties of aggregate chlorophyll that make light harvesting systems so efficient.
It is necessary to show that chl-xTB can be used to construct viable models for
these systems.

%=======
\section{Theory}
\label{sec:exciton_theory}

\subsection{Exciton States}
\label{subsec:exciton_states}
As stated in the introduction, a large majority of models of light harvesting systems
utilises a Frenkel Hamiltonian model of excitonic states, where the weak coupling
between monomers mean that a description can be constructed from properties of the 
individual sites.

To recap the exciton theory from the introduction, the exciton states $\ket{\Psi}$ can
be constructed as a Hartree product of the states of individual sites (also refered
to as monomers or chromophores)

\begin{equation}
    \ket{\Psi} = \Pi_m \ket{\phi_m}
\end{equation}
%
where $\ket{\phi_m}$ is the monomer state on sites $m$. These monomer states form
the basis function of the overall excitonic states. As the exciton is modelled to
be localised to a specific site, an exciton state with an exciton at site $i$ is
given by

\begin{equation}
    \ket{\Psi}^i = \ket{\phi_i}^\ast \Pi_{m \neq i}  \ket{\phi_m}
\end{equation}
%
where $\ket{\phi_i}^\ast$ is the excited state of monomer site $i$. States with 
more than one exciton are possible in this framework, however this is not necessary
for light harvesting systems as stated in the introduction. The Hamiltonian, including
the "ground state" where there are no excitons in the system, is given in by

\begin{equation}
    H = 
    \begin{bmatrix}
        E_0 & V_{0, \left(1, 1\right)} & \cdots & V_{0, \left(N, 1\right)} \\
        V_{0, \left(1, 1\right)} & E_{\left(1,1\right)} & \cdots & V_{\left(1,1\right) \left(N, 1\right)} \\
        \vdots & \vdots & \ddots & \vdots \\
        V_{0, \left(N, 1\right)} & V_{\left(1,1\right) \left(N, 1\right)} & \cdots & E_{\left(N, 1\right)}
    \end{bmatrix}
\end{equation}
%
where $N$ is equal to the total number of individual sites. The diagonal terms $E$
is the sum of the site energies and a point charge interaction

\begin{equation}
    E_0 = \sum_m e_m + \sum_{m \neq n, A \in m, B \in n} \frac{q^A_m q^B_n}{r_{AB}}
\end{equation}
%
where $e_m$ is the energy of site $m$ and $q^A_m$ is the charge centered on atom
$A$ in site $m$. Hamiltonian elements corresponding to a single excitation are similarly
given except with the excited state energy and charges

\begin{equation}
    E_{\left(m,1\right)} = e_m + \delta e_m + \sum_{n \neq m} e_n + \sum_{n, A \in m, B \in n} \frac{q^{\ast A}_m q^B_n}{r_{AB}} + \sum_{n,p \neq m, A \in n, B \in p} \frac{q^{A}_n q^B_p}{r_{AB}}
\end{equation}
%
where $\delta e_m$ is the excitation energy of site $m$, and charges marked $q^\ast$
are the excited state charges. The inter-site second term has been replaced here
to explicitly show that there is both the interaction of excited state point charges
with all other sites, as well as the ground-state ground-state interactions for 
all other non-excited chromophores.

The off diagonal elements are the coupling elements between all exciton states.
For coupling to the ground state, these are given by

\begin{equation}
    V_{0, \left(m,1\right)} = \sum_{n, A \in m, B \in n} \frac{q^{\text{tr},A}_m q^B_n}{r_{AB}}
\end{equation}
%
where charges marked $q^{\text{tr}}$ are transition charges. It can be seen that 
this coupling element includes an electrostatic interaction between all sites and
the excited site. This is different to coupling elements between two single exciton
states, given as

\begin{equation}
    V_{\left(m, 1\right), \left(n,1\right)} = \sum_{A \in m, B \in n} \frac{q^{\text{tr},A}_m q^{\text{tr}B}_n}{r_{AB}}
\end{equation}
%
which just involve sites $m$, $n$ which have local excitations.

This Hamiltonian for exciton states is slightly different those reported in the 
literature, as the ground state where no excitons are present is not usually included.
However, if the local excitations are sufficiently different in energy and the coupling
between ground and excited states is weak, then the eigensolution corresponding 
to the ground state would have mostly ground state character. The other excited
state eigensolutions would have very little ground state character, and so the block
matrix of just excited state contributions can be taken which would return the Hamiltonian
used in many other studies. Calculating the ground state however is necessary if
all the states for the exciton system are required, which is the case for the
spectral density investigation of the next chapter.

\subsection{Embedding}
\label{subsec:embedding}



\section{Truncated Chlorophylls}
\label{sec:trunc_chl}

A series of chlorophyll dimers were used to calculate benchmarks for an exciton
model based on chl-xTB calculations. The exciton predictions were compared to TD-DFT
calculations on the full dimer system. In order to get usable scaling with the DFT
methods, a truncated version of chlorophyll with the phytol tail removed was used.
In the large separation limit the TD-DFT and chl-xTB results should match, with
more discrepancy appearing as the chromophores are closer together and the differences
between Frenkel exciton coupling and a full DFT treatment are more apparent.

For each dimer, TD-DFT data was calculated for the two monomers separately as well
as for the full dimer system. The monomer data was then used to construct the Frenkel
Hamiltonian as described earlier.
The benchmarked data was first calculated with PBE0/Def2-SVP, however it was found 
that an exciton model constructed from monomer data did not well match it's full 
dimer counterpart. This is attributed to the lack of range-separation in the
functional.
As a replacement, the CAM-B3LYP functional was used instead. The exciton model matched
the full dimer calculation well.

\subsection{Rotation}
\label{subsec:rotation}

The first benchmark was constructed from a scan of transition energies over the
angle between two parallel chlorophyll molecules. This axis and orientation was
chosen as it would maximise the change in exciton coupling as angle varies. Additionally,
pairs of chlorophyll are nearly parallel in LHII, however they are not direcly overlaid
as they are in this set.

Dimer geometries were taken from 5 degree increments of the angle of rotation up 
to a full 360 degree rotation. The plot of the two excited state energies for both
CAM-B3LYP dimer and chl-xTB exciton methods are shown in fig., where they have been
aligned to their average values to remove the systematic shift.


\begin{figure}
    \includegraphics{../../Year_2/DimerModel/Scans/rotating_dimer.png}
    \caption{caption}
\end{figure}

It can be seen that the chl-xTB exciton model can reproduce high level TD-DFT dimer
data with an excellent level of accuracy. The energies for both excited states agree
across the entire range of angles, and much of the detailed variation in the CAM-B3LYP
data is also found in the chl-xTB data.. The two areas where the agreement is worse
is at 180 degrees and 270 degrees, where the coupling is strong and weak respectively.
This is attributed to the proximity of functional groups that are found at the
corners of the porphyrin ring, which are closest at these angles. As these groups
get closer together, the difference between the TD-DFT and chl-xTB treatment would
be expected to cause transition energies to grow more distinct. However it can be
seen that even with these effects the difference is small. Additionally, the angle
of chlorophyll pairs in LHII (for B850 a,b pairs) only varies around 0 by a few 
degrees, so these effects would not be so present.

\subsection{Distance}

The same scan was made on dimer pairs of with different separations. The same 
agreement was found in each case.

\label{subsec:distance}
\begin{figure}
    \includegraphics{../../Year_2/DimerModel/Scans/angle_and_distance.png}
    \caption{caption}
\end{figure}

\section{LHII Pairs}
\label{sec:LHII_excitons}

A number of full dimer properties were also calculated for chlorophyll pairs taken
from LHII. The chlorophyll geometries for these pairs also included the phytol tail
for a more complete test.
Similar to the coordinate scan tests above, both full dimer TD-DFT and exciton model
transition energies were calculated.

\subsection{Assignment of States}
\label{subsec:state_assign}

A more detailed view of each exciton state was needed to properly assign the two
excited states from the full TD-DFT and exciton models. It was found that for some
dimer pairs, some models would predict different energy ordering. More explicitly, 
for the A-B dimer system, one model might predict a lower energy for a state with
the exciton localised on monomer A, whereas another might predict the other state
with the exciton on B. It should be noted that the CAM-B3LYP exciton model and full
dimer data were always consistent, and it was only when comparing to PBE0 or chl-xTB
was this effect apparent.

It was therefore necessary to find the location of the excitons in both dimer and
exciton models. For the TD-DFT dimer result, the exciton location was taken as the
molecule where the transition charge distribution centered around. This centre was
calculated by the absolute charge weighted average of the atomic position. For the
Frenkel exciton result, the location was taken as the monomer which had the most
amount of character in the eigenvector solution.

\subsection{Comparison}
\label{subsec:comparison}
\begin{figure}
    \includegraphics{../../Year_2/DimerModel/excitons_against_dimer.png}
\end{figure}

\begin{figure}
    \includegraphics{../../Year_2/DimerModel/histogram_of_errors.png}
\end{figure}

There are a few conclusions to unpack from the comparison of Frenel exciton models
against full TD-DFT data. First from the CAM-B3LYP based exciton, it can be seen
that the exciton model and the full TD-DFT model agree well. There is a systematic
cluster of points that are seen to follow the $y=x$ line. These are dimers at large
separations where both the TD-DFT and exciton model predict no coupling and so
the individual site transition energies are recovered, which are the same for both
dimer and monomer calculations. At closer separations, the difference from using
the Frenkel model can be observed. Overall, the difference is small compared to 
the total range of transition energies.
The PBE0 exciton model has a large varience in transition energies. As the CAM-B3LYP
data shows that the exciton framework provides accurate predictions of full dimer
properties, this variance is attributed to the differences between the PBE0 and 
CAM-B3LYP functionals rather than full dimer vs. exciton models. Whilst these variances
are large, it is still within the range of errors from in monomer transition energies
reported in the previous chapter.
Therefore the variance in the chl-xTB exciton results to CAM-B3LYP data is more
due to training to the PBE0 functional rather than any inherent unsuitability to
the exciton framework.

\section{Conclusions}