%
% File: outline.tex
% Author: Oliver J. H. Feighan
% Description: Thesis outline
% Thesis outline
%
%\let\textcircled=\pgftextcircled
\chapter{Introduction}
\label{chap:intro}

\initial{P}hotosynthesis is the bedrock of life on this planet. It is often the 
first rung in the food chain, establishing ecosystems from a near unlimited source 
of sunlight. The oldest photosynthetic organisms are purple bacteria, appearing
on Earth X billion years ago. These organisms employ light harvesting complexes (LHCs) 
to absorb and stabilise light which is eventually transferred to reaction centres.
The steps of creating biomass from base materials occurs at these reaction centres,
using the photonic energy for charge transfer, however it is the light harvesting
complexes that are often investigated as they are nearly 100\% efficient at converting
photons to electronic energy.

Many types of complexes exist, differing by the type of chlorophyll pigments as 
well as structural features. Most often these complexes are formed of repeated units.
For example the LH2 complex found in \emph{acidophilus} is formed of a trimer unit
with two bacterial chlorophyll \emph{a} (BChla) chromophore in close proximity, 
aligned parallel along the porphyrin plane, with a third chromophore perpendicular
and further apart. This unit is then repeated to form a circular structure with
between an 8-10 fold symmetry depending on the environment in which the bacteria
are grown. It is thought that the efficiency of these LHCs is mostly due to these
structures, and the conformations of the chromophores.

Detailed computational study of these structures has been possible since the first
crystal structures were produced, and has given a wealth of analysis investigating 
the properties of these complexes. Most recently these include the effects of non-covalent
bonding on the chlorophyll pigments, etc. These studies show that the effects of
the protein occur on the atomistic level as well as on the order of the whole protein 
structure.

The physical properties of LHCs can be obtained by studying and constructing a Hamiltonian 
for the system. However usually it is necessary to reduce the number of degrees 
of freedom in these Hamiltonians. This is due to the sheer size of the complexes.
For LH2, the chlorophyll system alone contains 3780 atoms, with an additional 
~6000 atoms for the entire scaffold. Including a membrane and solvent can quickly
lead to the region of 300,000 atoms.

It can be seen that the usual electronic structure methods that describe molecular 
excited states are not tenable for LHCs. However a Frenkel-Davydov model can be 
utilised due to the weak coupling, which recaptures the delocalisation of excited
states over pigment sites. This model constructs the Hamiltonian from intra-site energies and
inter-site interactions, reducing the degrees of freedom of the Hamiltonian from
the entire state to just single chromophores.

The Frenkel exciton Hamiltonian can either be constructed from static parameters,
fit to experimental data or calculated theoretical values, or as functions of LHC 
geometry and/or time. To recover a truly atomistic treatment of LHCs it is necessary 
to take the second approach, which if looking at dynamic properties (i.e. requiring
a time series of LHC geometries) can be far more expensive to calculate.

Often LHC models of this kind fall into the same pattern. Electronic structure calculations
are used to produce excited state properties for individual sites, which in turn 
are used to construct Frenkel exciton Hamiltonians, ultimately giving the excited 
states of the whole LHC. It is obvious then where the model design choices - which 
electronic structure and response method to use to calculate intra-site properties,
and how to use these properties to construct the exciton Hamiltonian. The decisions
on these choices is highly dependent on the level of detail required for the exciton 
system (i.e. atomistic to coarse grain), and the volume of Hamiltonians required 
(i.e. how many frames of molecular dynamics are used). 

For a truly atomistic approach often density functional theory (DFT) and linear 
response methods (time-dependent DFT or TD-DFT) are used to calculate single site
properties. Due to the size of chlorophyll molecules, using high level methods such 
as coupled cluster and EOM-CCSD would not be reasonable. Even with TD-DFT methods
only a limited number of LHC geometries could have explicit Hamiltonians can be 
constructed before the cost is too high. If a large number of LHC geometries need 
to be calculated then further approximations are necessary.

This outlines the main issue explored in recent models of LHCs, in that detailed 
models are required but the electronic structure and response calculations necessary
are too expensive. Some solutions to this problem are offered in the literature, 
discussed below, however each have limitations. This thesis focuses on investigating
new solutions to this problem with the hope that the conclusions will help inform
future LHC models.

\section{Efficient Response Methods}
\label{sec:efficient_response_methods}

The most obvious solution to this problem is to make the electronic structure and
response calculations more efficient. This is often at the expense of accuracy.
Many recent studies use tight binding methods such as TD-DFTB or ZINDO methods\dots


A recent development in efficient response methods is found in the sTDA-xTB method,
which due to it's relevancy to later sections in this thesis is discussed in detail
here.

\subsection{sTDA-xTB}
\label{subsec:stda_xtb}
sTDA-xTB ("simplified Tann-Dancoff Approximation - eXtended Tight Binding") is another
method in the family of xTB methods developed by the Grimme group, and is parameterised
for transition properties \cite{Grimme2016}. The accuracy in calculating transition energies with this
method is very good, with the error compared to high-level method, such as SCS-CC2,
being around 0.3 - 0.5 eV.

Similar to other xTB methods, the sTDA-xTB method is a tight-binding method that
uses empirically fitted parameters and a minimal basis set. It was trained on a
test set of highly accurate coupled cluster and density functional theory
excitation energies, as well as atomic partial charges for inter-electronic interactions.

Unlike other xTB methods, coefficients in the basis set for sTDA-xTB are dependent on the D3
coordination number. This makes basis functions far more flexible, which would usually
be achieved with fixed basis functions by using diffuse or other additional orbitals in
the basis set. Additionally, it uses two sets of parameterized basis sets - a
smaller valence basis set (VBS) and an extended basis set (XBS). Whilst this reduces
the cost of having larger basis sets, it makes calculating the gradient of transition
properties much more difficult. This motivates the work on designing an alternative
method with more tractable gradients, instead of using this already established method.

The two basis sets are used to construct formally similar Fock matrix elements,
although in practice they use different global parameters. The core Hamiltonian
is similar to other DFTB methods that use a self-consistent charge (SCC) method, as
opposed to an SCF method, to obtain molecular orbital coefficients. It is given by

\begin{equation}
\bra{\psi_\mu} H^{\text{EHT, sTDA-xTB}} \ket{\psi_\mu}= \frac{1}{2} \left(k^l_\mu k^{l'}_\nu\right) \frac{1}{2} \left(h^l_\mu h^{l'}_\nu\right) S_{\mu\nu} - k_T \bra{\psi_\mu}\hat{T}\ket{\psi_\nu}
\end{equation}
%
where $\mu,\nu,l,l'$ are orbital and shell indices, $k^l_\mu$ are shell-wise 
H{\"u}ckel parameters, $h$ are effective atomic-orbital energy levels, $S_{\mu\nu}$
is the overlap of orbitals $\mu$ and $\nu$, $k_T$ is a global constant and $\hat{T}$
is the kinetic energy operator. The charges used in the inter-electronic repulsion 
function are given by charge model 5 (CM5) \cite{Marenich2012} charges for the XBS
Fock matrix. These are calculated using Mulliken charges obtained from diagonalising
the Fock matrix with the VBS. The charges for the initial VBS Fock matrix are based
on Gasteiger charges \cite{Gasteiger1978}, modified by the parameterised
electronegativities of atoms in the system.

The whole process for determining molecular orbitals can be summarized as:
\begin{enumerate}
	\item Calculate modified Gasteiger charges for the first initial guess
	\item Diagonalise Fock matrix in the VBS to get the first set of Mulliken charges
	\item Compute CM5 charges
	\item Diagonalise Fock matrix in the VBS again for final set of Mulliken charges.
	\item Recalculate CM5 charges with this final set, and diagonalize the Fock matrix in the XBS. The molecular orbital coefficients from this are then fed to the response theory.
\end{enumerate}

The response theory for this method is based on previous work in the Grimme 
group on the simplified Tamm-Dancoff Approximation \cite{Grimme2013}.
There are several approximations made between full linear response theory and
the sTDA method. First is the Tamm-Danncoff approximation, where the $\mathbf{B}$
matrix is ignored. The second approximation is to use monopole approximations with
Mataga-Nishimoto-Ohno-Klopman (MNOK) operators instead of explicit 2 electron integral
as well as neglecting the density functional term.

Transition charges are used to calculate these MNOK integrals. The charge $q^A_{nm}$
centred on atom $A$ associated withthe transition from $ n \rightarrow m$, are
computed using a Löwdin population analysis

\begin{equation}
q_{nm}^A = \sum_{\mu \in A} C^\prime_{\mu n} C^\prime_{\mu m}
\end{equation}

where the transformed coefficients $C^\prime_{\mu n}$ are given by orthogonalising
the original MO coefficients $\textbf{C}$

\begin{equation}
\textbf{C}^\prime = \textbf{S}^{\frac{1}{2}} \textbf{C}
\end{equation}

and $\mu$ is an index that runs over the atomic orbitals (AO). The MO coefficients
are the solution of diagonalising the Fock matrix, similar to equation \ref{eq:roothaan_hall}.

Approximations to full 2 electron integrals are given by charge-charge interaction
damped by the MNOK\cite{Nishimoto1957}\cite{Ohno1964}\cite{Klopman1964} functions.
For exchange and coloumb type integrals, difference exponents are used, along with
an additional free parameter to recover the amount of Fock exchange mixing in
the original matrix element equation. These will be discussed in more detail in
the next chapter, as they are a crucial part of designing a new response method 
for chlorophyll systems.

Third is the truncation of single particle excited space that is used to construct 
the $\mathbf{A}$ matrix. This reduces the number of elements that need to be 
calculated, and so reduces the time taken for diagonalisation, whilst also capturing 
a broad enough spectrum of excitation energies. The sTDA-xTB has many of the same 
goals as this project, except in one respect, which is the gradient theory. As 
the sTDA-xTB method still requires constructing and diagonalizing the $\mathbf{A}$ 
matrix, albeit with a tight-binding method for molecular orbital coefficients, 
the gradient of the transition properties would still be difficult to calculate.

The sTDA-xTB method is reported as having excellent accuracy against benchmarked
data, and has been used to generate absorption spectra and other properties for 
large systems. At first glance, it would seem that this method would solve the issue
of calculating many Frenkel exciton systems. However this is not given for two reasons.
First is that much of the data expressing sTDA-xTB accuracy has been performed on
a range of systems, and does not concern smaller variations of a single system. 
The latter is more important for LHCs, as the variations of in chlorophyll geometries
give rise to variations in the exciton system. Without proper treatment of the former
it is hard to expect good predicts in the latter. It would be better to start from
methods that are expected to have accurate correlations with system geometries, 
such as TD-DFT, and work down to retain this accuracy. The issue of keeping accurate
transitions properties variations with respect to geometry variations is a key theme
in this work.

\section{Statistical Methods}
\label{sec:stats_methods}

As well as making response approximations, using more approximate methods of constructing 
the exciton framework is also possible. One of the simplest ways of doing this is
by using static parameters fit from experimental or calculated data. Using these
static Hamiltonians negates any variation in intra-chromophore or protein scaffold 
geometry, but can still produce good predictions of physical phenomena.

If using a long timescale, where the full conformation space is well sampled, exciton
Hamiltonians can be constructed from distributions of chromophore response properties
(i.e. excitation energies and transition densities). These properties would be distributed
along a normal distribution when taken from a set of uncorrelated structures. The
mean and standard deviations can then be used to define a distribution function 
which can be sampled to construct Hamiltonians without the need for explicit calculations
on structures. The Hamiltonians could utilise functions that take into account inter-chromophore
geometries, or use distributions coupling values.

These methods are mostly based on static parameters, such as the mean and standard
deviation or the Hamiltonian elements themselves, and are not functions of time.
This means they would be ill-suited for dynamic studies where structures at different
times are correlated. However recently machine-learning methods has been reported
that would give time-dependent Hamiltonians still without the need for explicit
calculations.

Machine-learning models have been used in many areas of computational chemistry,
especially in areas where both large amounts data and numerical metrics make it 
easy to train these methods. At their heart, these methods are similar to the static
Statistical methods that have already been used for LH2 exciton systems, as they
rely on parameters fit to high level data. However these new models use machine-
learning techniques to incorporate atomic geometry information.

In 2016 H\"{a}se \emph{et al.} reported on a multi-perceptron (also referred to
as a neural network (NN)) model that predicts the \Qy transition for chlorophyll
molecules. Using this model, as well as fitted parameters for the exciton coupling,
it was possible to calculate exciton population dynamics, as well as spectral densities
for chlorophyll sites in the FMO light harvesting complex. Similar to other models
a Coloumb matrix is used as descriptor of the chlorophyll systems, defined as 

\begin{equation}
	M_{AB} = 
	  \begin{cases}
		\frac{1}{2} Z^{2.4} \text{ for } A = B\\
		\frac{Z_A Z_B}{\left\lvert \mathbf{R}_A - \mathbf{R}_B\right\rvert} \text{ for } A \neq B
	  \end{cases}
\end{equation}

where $Z_A$ is some measure of the atomic charge on atom $A$, and $R_A$ is the position
vector. It can be seen that the off-diagonal elements are simply the Coloumbic interactions,
and diagonal elements are a polynomial of the atomic charges. This descriptor is
popular due to the similarity in information that an electronic structure calculation
would start from, namely the positions and nuclear charges of atoms.

This matrix is used as an input for a neural network with two hidden layers. Briefly,
a neural network is a series of matrix multiplications applied to input data that
act as a non-linear function. They are formed of an input and output layer, with 
any number of "hidden" layers in-between. A layer is a distinct vector of values.
For example, flattening the Coloumb matrix into a 1D vector gives the input layer,
and the \Qy transition energy is a the output vector (of 1 element). A vector at
layer $n$, is multiplied with a coefficient matrix $c$ to give the next layer

\begin{equation}
	\mathbf{V}_{n+1} = \mathbf{c}_n \mathbf{V}_n	
\end{equation}

where the coefficient matrices $\mathbf{c}_n$ are fitted to give the smallest deviations
of the output layer against target data (i.e. error in predicted \Qy energies in 
this case). Often additional functions are used to modify these values, such as
sinusoidal, sigmoid, linear ($y=mx$) or rectified linear unit ($y=\text{max}\left(0, mx \right)$)
functions. In this way they are conceptually similar to biological neurons which
take in electrical signals, and through some mechanism are can be activated to send
out a different signal. The coefficient matrices $\mathbf{c}_n$, and any other parameters
in the activation functions, are fit by a back-propagation method, which also uses
parameters - these are referred to as hyper-parameters, and using a systematic grid-search
can find the best hyper-parameters in which to learn the best coefficient matrix
(although other hyper-parameter searching method can be used). Other considerations
such as over-fitting also need to be taken into account.

This model predicted \Qy transition energies with around a 0.3 meV error for all
of the 8 sites in the FMO complex. This is exceptionally accurate, supporting the
idea that atomic positions and nuclear charges contain all the information necessary
to predict transition energies. It is noted in this work that the Nitrogen root
mean squared deviations (RSME) correlates well with excited state properties. Exciton
properties, such as the time series of exciton populations, were also well reproduced
using these \Qy energies, although the coupling parameters were taken from other
fits and not from a NN method.

Another method, developed by Farahvash \emph{et al.}, utilised both neural networks
and kernel ridge regression (KRR) to predict both site energies as well as exciton 
coupling parameters, giving a completely time dependent exciton Hamiltonian. KRR
is another machine-learning method, that can be understood as two processes. First
is the ridge regression, which is similar to a linear regression model but with
an additional factor to account for co-linear properties of inputs. Regression models
are multivariate linear models that follow the form

\begin{equation}
	f^\prime\left(\mathbf{X}\right) = \mathbf{X} \mathbf{\beta}
\end{equation}

where $f^\prime\left(\mathbf{X}\right)$ are the predicted values of some metrics $f\left(\mathbf{X}\right)$ 
(i.e. \Qy energy or exciton coupling value), $\mathbf{X}$ is the matrix of information
used to predict the value $f$ (i.e. flattened Coloumb matrix, referred to as the feature
matrix) and $\beta$ is a set of fitted coefficients that minimise the difference 
$\left\lvert f^\prime \left( \mathbf{x}\right) - f \left(\mathbf{x}\right)\right\rvert$ k.
The matrix $\mathbf{\beta}$ can be found by minimising the squares of this difference, 
known as the least-squares method, however this can lead to expensive terms such
as the inner product of the feature matrix. Here the kernel trick is used to make
these terms easier to calculate, 

squares approach used in optimising the regression model, some terms in these regression
models can become expensive to calculate. This leads to the second part of the model, 
where the "kernel trick" is employed. This rearranges the minimisation of regression
coefficients so that the inner products are not required, but requires a new function
that compares the similarity of features. This gives the function that predicts 
properties as 

\begin{equation}
	f_{\text{KRR}}^\prime = \sum^{N_x}_j \beta_j k\left(\mathbf{x}, x_j\right)
\end{equation}

where the $N_x$ is the size of the feature vector $\mathbf{x}$ (which is the linear
model is stacked to form the matrix $\mathbf{X}$), and $k$ is the kernel function.
Often this is a gaussian function of the feature vector

\begin{equation}
	k\left(\mathbf{x}, x_j\right) = \text{exp}\left(\frac{-\left(\mathbf{x}-x_j\right)^2}{2\sigma^2}\right)
\end{equation}

where $\sigma$ is a fitted parameter. Again these parameters are optimised by searching
through values, similar to the grid search referenced before.

A KRR model was developed for both the excitation energies and exciton coupling 
parameters. However it was found that a NN model predicted exciton couplings with
greater accuracy.


\section{Hardware}
\label{sec:hardware}

\section{Applications of LH2 Exciton Systems}
\label{sec:lit_review_conclusions}
